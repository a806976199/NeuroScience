\chapter*{Introduction}
\label{Introduction}
%\begin{multicols}{2}
 % \setlength{\columnseprule}{0.2pt}

Computational neuroscience is an approach to understanding the information content of neural signals by modeling the nervous system at many different structural scales, including the biophysical, the circuit, and the systems levels. Theoretical analysis and computational modeling are important tools for characterizing what nervous systems do, determining how they function, and understanding why they operate in particular ways.
The questions what, how, and why are addressed by descriptive, mechanistic, and interpretive models, each of which we discuss in the following chapters.
\begin{defn}
  \emph{Descriptive models} summarize large amounts of experimental data compactly yet accurately, thereby characterizing what neurons and neural circuits do.
\end{defn}

\begin{defn}
   \emph{Mechanistic models} address the question of how nervous systems operate on the basis of known anatomy, physiology, and circuitry.
\end{defn}

\begin{defn}
  \emph{Interpretive models} use computational and information-theoretic principles to explore the behavioral and cognitive significance of various aspects of nervous system function, addressing the question of why nervous systems operate as they do.
\end{defn}

%\end{multicols}

%%% Local Variables:
%%% mode: latex
%%% TeX-master: "../notesOnFluidMechanics"
%%% End:
