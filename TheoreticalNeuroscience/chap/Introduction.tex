\chapter*{Introduction}
\label{Introduction}
%\begin{multicols}{2}
 % \setlength{\columnseprule}{0.2pt}
  
  Computational neuroscience is an approach to understanding the information content of neural signals by modeling the nervous system at many different structural scales, including the biophysical, the circuit, and the systems levels. Theoretical analysis and computational modeling are important tools for characterizing what nervous systems do, determining how they function, and understanding why they operate in particular ways.
  \con [\emph{Descriptive Models}]Summarizing large amounts of experimental data compactly yet accurately, thereby characterizing what neurons and neural circuits do.
  \con [\emph{Mechanistic Models}]Addressing the question of how nervous systems operate on the basis of known anatomy, physiology, and circuitry.
  \con [\emph{Interpretive Models}]Using computational and information-theoretic principles to explore the behavioral and cognitive significance of various aspects of nervous system function, addressing the question of why nervous systems operate as they do.

%\end{multicols}

%%% Local Variables:
%%% mode: latex
%%% TeX-master: "../notesOnFluidMechanics"
%%% End:
