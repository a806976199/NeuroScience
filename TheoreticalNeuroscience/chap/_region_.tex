\message{ !name(../notesOnNeuroScience.tex)}
\message{ !name(NeuralEncoding1/Part3.tex) !offset(368) }
rul:whiteNoiseApprox}
  An approximation to white noise can be generated by choosing each $s_m$ independently from a probability distribution with mean $0$ and variance $\sigma_s^2 / \Delta{t}$.
\end{rul}
\begin{rem}
  Any reasonable probability function satisfying two conditions in Rule \ref{rul:whiteNoiseApprox} can be used to generate the stimulus values within each time bin. The factor of $1/\Delta{t}$ in the variance indicates that the variability must be increased as the time bins get smaller.
\end{rem}

\begin{exm}
  \label{exm:whiteNoiseApproxExm}
  A special class of white-noise stimuli, Gaussian white noise, results when the probability distribution used to generate the $s_m$ values is a Gaussian function.
\end{exm}


\begin{rem}
  Although Equations \ref{equ:1.40} and \ref{equ:1.44} are both sound, they do not provide a statistically efficient method of estimating the power spectrum of discrete approximations to white-noise sequences generated by the methods described in this chapter.
\end{rem}

\begin{rul}
\message{ !name(../notesOnNeuroScience.tex) !offset(-21) }
