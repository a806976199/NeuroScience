\section{Encoding and Decoding}
\label{sec:Encoding and Decoding}

\begin{ntn}
  \label{ntn:probablity}
  Several different probabilities and conditional
  probabilities enter into our discussion. The probabilities we need are:
  \begin{itemize}
  \item $P[s]$, the probability of stimulus $s$ being presented,
    often called the prior probability.
  \item $P[\mathbf{r}]$, the probability of response $\mathbf{r}$
    being recorded independent of what stimulus was used.
  \item $P[\mathbf{r},s]$, the probability of stimulus $s$ being
    presented and response $\mathbf{r}$ being recorded. This is called the
    joint probability.
  \item $P[\mathbf{r}|s]$, the conditional probability of evoking response $\mathbf{r}$, given that
    stimulus $s$ was presented.
  \item $P[s|\mathbf{r}]$, the conditional probability that stimulus $s$ was presented,
    given that response $\mathbf{r}$ was recorded.
  \end{itemize}
  Here $\mathbf{r}=(r_1,r_2,\cdots,r_N)$ for $N$ neurons is a list of spike-count fring rates.
\end{ntn}

\begin{ntn}
  \label{encoding and decoding}
  The \emph{encoding} is characterized by the set of probabilities
  $P[\mathbf{r}|s]$ for all stimuli and responses, with which we
  considered the problem of predicting neural responses to known
  stimuli. The \emph{decoding} a response is to determine the probabilities
  $P[s|\mathbf{r}]$ which reflects what is going on in the real world from neuronal spiking patterns.
\end{ntn}

\begin{thm}
  \label{Bayes theorem}
  \emph{Bayes theorem} relating $P[s|\mathbf{r}]$ to $P[s|\mathbf{r}]$:
  \begin{equation*}
    \label{eq:Bayes}
   P[s|\mathbf{r}]=\frac{P[\mathbf{r}|s]P[s]}{P[\mathbf{r}]}.
 \end{equation*}
\end{thm}

\begin{rem}
  According to Bayes theorem,  $P[s|\mathbf{r}]$ can be obtained from $P[\mathbf{r}|s]$, but the stimulus probability $P[s]$ is also needed. As a result, decoding requires knowledge of the statistical properties of experimentally or naturally occurring stimuli.
\end{rem}

\begin{rem}
  We sometimes treat the response fring rates or the stimulus values as continuous variables. In this case, the probabilities listed must be replaced by
the corresponding probability densities,
$p[\mathbf{r}]$, $p[\mathbf{r}|s]$, etc.

\end{rem}


%%% Local Variables:
%%% mode: latex
%%% TeX-master: "../notesOnFluidMechanics"
%%% End:
