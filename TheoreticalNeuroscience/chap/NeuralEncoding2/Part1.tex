
%\begin{rem}
 % The spike-triggered average stimulus introduced in chapter 1
  %is a standard way of characterizing the selectivity of a neuron.
%\end{rem}

\begin{ntn}
  \label{ntn:reverse-correlation}
  \emph{Reverse-correlation} is a technique for studying
  how sensory neurons add up signals from different locations
  in their receptive fields, and also how they sum up stimuli
  that they receive at different times, to generate a response.
\end{ntn}

\begin{rem}
  The goal of the reverse-correlation technique is to find a function $r = f(s)$ that maps from the stimulus $s$ to the neuronal response $r$, where the stimulus is a function dependent on spatial location and time $s = s(x,y,z,t)$. 
\end{rem}

\begin{rem}
  The reason that this technique is called "reverse" is that we align the time origin with the neuron's response and then reverse the timeline to find what stimulus ($t<0$) triggered the neuron's response at the current moment ($t=0$).
\end{rem}

\begin{ntn}%%?????whether can I do this notation?
  \label{ntn: typesOfModels}
  \emph{Descriptive Models} approximate descriptions of neural responses,
  and do not explain how visual responses arise from the synaptic,
  cellular, and network properties of neoral circuits.
  Nevertheless, they provide an important framework for characterizing response selectivities,
  a reference point for identifying and characterizing novel effects,
  and a basis for building mechanistic models,
  some of which are discussed at the end of this chapter.
\end{ntn}

%%% Local Variables:
%%% mode: latex
%%% TeX-master: "../notesOnFluidMechanics"
%%% End:
