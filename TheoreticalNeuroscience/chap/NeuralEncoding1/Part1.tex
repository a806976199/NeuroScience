\documentclass[letterpaper,oneside]{book}

\usepackage{geometry}
% make full use of A4 papers
\geometry{margin=1.5cm, vmargin={0pt,1cm}}
\setlength{\topmargin}{-1cm}
\setlength{\paperheight}{29.7cm}
\setlength{\textheight}{25.1cm}

% auto adjust the marginals
%\usepackage{marginfix}

\usepackage{amsfonts}
\usepackage{amsmath}
\usepackage{amssymb}
\usepackage{amsthm}
\usepackage{CJKutf8}   % for Chinese characters
\usepackage{enumerate}
\usepackage{graphicx}  % for figures
\usepackage{layout}
\usepackage{multicol}  % multiple columns to reduce number of pages
\usepackage{mathrsfs}
\usepackage{fancyhdr}
\usepackage{subfigure}
\usepackage{tcolorbox}
\usepackage{tikz-cd}
\usepackage{gensymb}
\usepackage{upgreek}
\usepackage{dsfont}

%------------------
% common commands %
%------------------
\newcommand{\dif}{\mathrm{d}}
\newcommand{\Dim}{\mathrm{D}}
\newcommand{\avg}[1]{\left\langle #1 \right\rangle}
\newcommand{\xibold}{\boldsymbol{\xi}}
\newcommand{\varphibold}{\boldsymbol{\varphi}}
\newcommand{\psibold}{\boldsymbol{\psi}}
\newcommand{\RE}{{\text{Re}}}

% this environment is for solutions of examples and exercises
\newenvironment{solution}%
{\noindent\textbf{Solution.}}%
{\qedhere}
% the following command is for disabling environments
%  so that their contents do not show up in the pdf.
\makeatletter
\newcommand{\voidenvironment}[1]{%
  \expandafter\providecommand\csname env@#1@save@env\endcsname{}%
  \expandafter\providecommand\csname env@#1@process\endcsname{}%
  \@ifundefined{#1}{}{\RenewEnviron{#1}{}}%
}
\makeatother

%----------------------------------------
% theorem and theorem-like environments %
%----------------------------------------
\numberwithin{equation}{chapter}
\theoremstyle{definition}

\newtheorem{thm}{Theorem}[chapter]
\newtheorem{alg}[thm]{Algorithm}
\newtheorem{asm}[thm]{Assumption}
\newtheorem{axm}[thm]{Axiom}
\newtheorem{coro}[thm]{Corollary}
\newtheorem{defn}[thm]{Definition}
\newtheorem{exm}[thm]{Example}
\newtheorem{exc}[thm]{Exercise}
\newtheorem{frm}[thm]{Formula}
\newtheorem{lem}[thm]{Lemma}
\newtheorem{ntn}{Notation}
\newtheorem{prop}[thm]{Proposition}
\newtheorem{rem}{Remark}[chapter]
\newtheorem{rul}[thm]{Rule}
\newtheorem{prin}[thm]{Principle}
%\newtheorem{ter}[thm]{Term]

\begin{document}
\pagestyle{empty}
\pagenumbering{roman}

%\tableofcontents
%\clearpage

\pagestyle{fancy}
\fancyhead{}
\lhead{Yang Li}
\chead{Computational and Mathematical Modeling of Neural Systems}
\rhead{2020}

%\setcounter{chapter}{-1}
\pagenumbering{arabic}
% \setcounter{page}{0}

% --------------------------------------------------------
% uncomment the following to remove these environments
%  to generate handouts for students.
% --------------------------------------------------------
%\begingroup
%\voidenvironment{rem}%
%\voidenvironment{proof}%
%\voidenvironment{solution}%

% each chapter is factored into a separate file.

\chapter{Neural Encoding I:\\
Firing Rates and Spike Statistics}
\label{cha:equations-motion}

\begin{multicols}{2}
\setlength{\columnseprule}{0.2pt}
\section{1.1 Introduction}
\subsection{A brief introduction to Computational Neuroscience:}
\ntn Computational neuroscience is an approach to understanding the information content of neural signals by
modeling the nervous system at many different structural scales,including the biophysical,the circuit,and the systems levels.
\ntn Theoretical analysis and computational modeling are important tools for characterizing what nervous systems do,determining how they function,and understanding why they operate in particular ways.
\subsection{The modeling types about what,how and why:}
\ntn (Descriptive models)Descriptive models summarize large amounts of experimental descriptive models data compactly yet accurately, thereby characterizing what neurons and neural circuits do.
\ntn (Mechanistic models)Mechanistic models, on the other hand, address the question of how nervous systems operate on the basis of known anatomy, physiology,and circuitry.
\ntn (interpretive models)Interpretive models use computational and information-theoretic principles interpretive models to explore the behavioral and cognitive significance of various aspects of nervous system function, addressing the question of why nervous systems operate as they do.
\subsection{Explanation of some terms:}
\ntn (Neurons) Neurons are highly specialized for generating electrical signals in response to chemical and other inputs, and transmitting them to other cells.
\ntn (Dendrites)Receiving information inputs from other neurons.
\ntn (Axon)Carrying the neuronal output to other cells.
\ntn (Ion channels) Ion channels control the flow of ions across the cell membrane by opening and closing in response to voltage changes and to both internal and external signals.
\ntn (Membrane Potential)The potential difference between two solutions separated by membranes,generally refers to the electrical phenomenon accompanying the life activities of cells,which exists on both side of cells.
\rem Under resting conditions,the potential inside the cell membrane(mainly $K^+$) is negative,outside the cell membrane(mainly $Na^+$) is positive, and the cell is said to be polarized.
\ntn (Hyperpolarization) Current in the form of positively charged ions flowing out of the cell (or negatively charged ions flowing into the cell) through open channels makes the membrane potential more negative, a process called hyperpolarization.
\ntn (Depolarization)Current flowing into the cell changes the membrane potential to less negative or even positive values. This is called depolarization.
\ntn (Action Potential)If a neuron is depolarized sufficiently to raise the membrane potential above a threshold level, a positive feedback process is initiated, and the neuron generates an action potential.
\rem The generation of action potential is the result of depolarization and hyperpolarization,which is the result of $K^+$ and $Na^+$ transmembrane flow.When $V_{K^+}=V_{Na^+}$,said there is a spike.
\rem Neurons typically respond by producing complex spike sequences.
\ntn (Absolute Refractory Period)For a few milliseconds just after an action potential has been fired, it may be virtually impossible to initiate another spike.
\ntn (Relative Refractory Period)After the absolute refractory period,the excitability of cells gradually recovers.After stimulation,excitement can occur,but the stimulation must be greater than the original threshold intensity.
\defn Neural encoding refers to the map from stimulus to response.
\exm We can catalog how neurons respond to a wide variety of stimuli, and then construct models that attempt to predict responses to other stimuli.
\defn Neural decoding refers to the reverse map, from response to stimulus.
%\input{sec/introduction}
\rem . The complexity and trial-to-trial variability of action potential sequences make it unlikely that we can describe and predict the timing of each spike deterministically. Instead, we seek a model that can account for the probabilities that different spike sequences are evoked by a specific stimulus.





%\input{sec/fluidKinematics}

%\input{sec/materialDerivative}

%\input{sec/ReynoldsTransportThm}

%\input{sec/incompressibility}

%\input{sec/conservationOfMass}

%\input{sec/EulerEquations}

%\input{sec/CartesianTensors}

%\input{sec/CauchyEquation}

%\input{sec/angularMomentum}

%\input{sec/constitutiveEquation}

%\input{sec/NavierStokesEq}

%\input{sec/dimensionalAnalysis}

%\input{sec/GePUP}

\end{multicols}

%\input{sec/GePUPSpatialDiscretization}

\clearpage

\appendix


%\bibliography{bib/numericalApprox}
%\bibliographystyle{abbrv}
%\bibliographystyle{abbrvnat}
%\setcitestyle{authoryear,open={[},close={]}}

\end{document}

<<<<<<< HEAD
=======
% wo asasdasdasn
% wobuai bu xi zao de songjianjian
>>>>>>> 7d36cbc66190f50176a2cc9353c273f68faf923b

%%% Local Variables:
%%% mode: latex
%%% TeX-master: t
%%% End:

