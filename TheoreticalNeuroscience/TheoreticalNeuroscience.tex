\documentclass[letterpaper,oneside]{book}

\usepackage{geometry}
% make full use of A4 papers
\geometry{margin=1.5cm, vmargin={0pt,1cm}}
\setlength{\topmargin}{-1cm}
\setlength{\paperheight}{29.7cm}
\setlength{\textheight}{25.1cm}

% auto adjust the marginals
\usepackage{marginfix}

\usepackage{amsfonts}
\usepackage{amsmath}
\usepackage{amssymb}
\usepackage{amsthm}
\usepackage{CJKutf8}   % for Chinese characters
\usepackage{enumerate}
\usepackage{graphicx}  % for figures
\usepackage{layout}
\usepackage{multicol}  % multiple columns to reduce number of pages
\usepackage{mathrsfs}  
\usepackage{fancyhdr}
\usepackage{subfigure}
\usepackage{tcolorbox}
\usepackage{tikz-cd}
\usepackage{gensymb}
\usepackage{upgreek}
\usepackage{dsfont}

%------------------
% common commands %
%------------------
\newcommand{\dif}{\mathrm{d}}
\newcommand{\Dim}{\mathrm{D}}
\newcommand{\avg}[1]{\left\langle #1 \right\rangle}
\newcommand{\xibold}{\boldsymbol{\xi}}
\newcommand{\varphibold}{\boldsymbol{\varphi}}
\newcommand{\psibold}{\boldsymbol{\psi}}
\newcommand{\RE}{{\text{Re}}}

% this environment is for solutions of examples and exercises
\newenvironment{solution}%
{\noindent\textbf{Solution.}}%
{\qedhere}
% the following command is for disabling environments
%  so that their contents do not show up in the pdf.
\makeatletter
\newcommand{\voidenvironment}[1]{%
  \expandafter\providecommand\csname env@#1@save@env\endcsname{}%
  \expandafter\providecommand\csname env@#1@process\endcsname{}%
  \@ifundefined{#1}{}{\RenewEnviron{#1}{}}%
}
\makeatother

%----------------------------------------
% theorem and theorem-like environments %
%----------------------------------------
\numberwithin{equation}{chapter}
\theoremstyle{definition}

\newtheorem{thm}{Theorem}[chapter]
\newtheorem{alg}[thm]{Algorithm}
\newtheorem{asm}[thm]{Assumption}
\newtheorem{axm}[thm]{Axiom}
\newtheorem{coro}[thm]{Corollary}
\newtheorem{defn}[thm]{Definition}
\newtheorem{exm}[thm]{Example}
\newtheorem{exc}[thm]{Exercise}
\newtheorem{frm}[thm]{Formula}
\newtheorem{lem}[thm]{Lemma}
\newtheorem{ntn}{Notation}
\newtheorem{prop}[thm]{Proposition}
\newtheorem{rem}{Remark}[chapter]
\newtheorem{rul}[thm]{Rule}
\newtheorem{prin}[thm]{Principle}

\begin{document}
\pagestyle{empty}
\pagenumbering{roman}

%\tableofcontents
%\clearpage

\pagestyle{fancy}
\fancyhead{}
\lhead{Yang Li}
\chead{Notes on Fluid Mechanics}
\rhead{2020}

%\setcounter{chapter}{-1}
\pagenumbering{arabic}
% \setcounter{page}{0}

% --------------------------------------------------------
% uncomment the following to remove these environments 
%  to generate handouts for students.
% --------------------------------------------------------
%\begingroup
%\voidenvironment{rem}%
%\voidenvironment{proof}%
%\voidenvironment{solution}%

% each chapter is factored into a separate file.

\chapter{ Neural Encoding I: Firing Rates and Spike Statistics}
\label{cha:Neural Encoding I}

\begin{multicols}{2}
\setlength{\columnseprule}{0.2pt}  

\section{Introduction}
\subsection{The Explanation of Some Terms}
\rem \emph{Neurons} are highly specialized for generating electrical signals in response to chemical and other inputs, and transmitting them to other cells.
\rem \emph{Dendrites} receives information inputs from other neurons.
\rem \emph{Axon} carries the neuronal output to other cells.
\begin{center}
    \label{fig:1.1}
    \includegraphics[scale = 0.35]{png/Figure1-1}\\
\end{center}

\rem \emph{Ion channels} control the flow of ions across the cell membrane by opening and closing in response to voltage changes and to both internal and external signals.
\begin{center}
  \label{fig:1.2}
  \includegraphics[scale = 0.55]{png/Figure1-2}\\
\end{center}
  
\con [Membrane Potential]The potential difference between two solutions separated by membranes, generally refers to the electrical phenomenon accompanying the life activities of cells, which exists on both side of cells.
\rem Under resting conditions,the potential inside the cell membrane(mainly K$^+$) is negative, outside the cell membrane(mainly Na$^+$) is positive, and the cell is said to be \emph{polarized}.
\defn [Action Potential] \emph{Action potential} is the characteristic electrical pulses or, more simply, spikes that can travel down nerve fibers.
\con [Hyperpolarization]Current in the form of positively charged ions flowing out of the cell (or negatively charged ions flowing into the cell) through open channels makes the membrane potential more negative, a process called \emph{hyperpolarization}.
\con [Depolarization]Current flowing into the cell changes the membrane potential to less negative or even positive values. This is called \emph{depolarization}.
\rem If a neuron is depolarized sufficiently to raise the membrane potential above a threshold level, a positive feedback process is initiated, and the neuron generates an \emph{action potential}.
\con [Absolute Refractory Period]For a few milliseconds just after an action potential has been fired, it may be virtually impossible to initiate another spike.
\con [Relative Refractory Period]After the absolute refractory period, the excitability of cells gradually recovers. After stimulation, excitement can occur, but the stimulation must be greater than the original threshold intensity.
\rem \emph{Absolute refractory period} and \emph{relative refractory period} are two basic phenomena in the process of neural response.
\subsection{Recording Neuronal Responses}
\exm Intracellular and extracellular methods for recording neuronal responses electrically
\begin{center}
    \label{fig:1.3}
    \includegraphics[scale = 0.35]{png/Figure1-3}\\
\end{center}

\begin{enumerate}[(i)]
  \item The top trace represents a recording from an intracellular electrode connected to the soma of the neuron.
  \item The middle trace is a simulated extracellular recording.
  \item The bottom trace represents a recording from an intracellular electrode connected to the axon some distance away from the soma.
\end{enumerate}




\subsection{From Stimulus to Response}
\rem  Neurons typically respond by producing complex spike sequences that reflect both the intrinsic dynamics of the neuron and the temporal characteristics of the stimulus.
\defn Neural encoding refers to the map from stimulus to response.
\exm We can catalog how neurons respond to a wide variety of stimuli, and then construct models that attempt to predict responses to other stimuli.
\defn Neural decoding refers to the reverse map, from response to stimulus.
\rem The complexity and trial-to-trial variability of action potential sequences make it unlikely that we can describe and predict the timing of each spike deterministically. Instead, we seek a model that can account for the probabilities that different spike sequences are evoked by a specific stimulus.


%%% Local Variables:
%%% mode: latex
%%% TeX-master: t
%%% End:


\section{Discrimination}
\label{sec:Single-cell decoding}
\begin{exm}
  \label{exm:mokey experiment}
  In the experiments performed by Britten et al. (1992).,
a monkey was trained to discriminate between two directions of motion
of a visual stimulus which was a pattern of dots on a video monitor. The percentage of dots that move together in the fixed direction is
called the coherence level. By varying the degree of coherence shown by
pictures, the task of detecting the movement direction can be made more or less
diffcult.
\begin{center} 
 \includegraphics[scale = 0.3]{./png/3-1}
\end{center}
\end{exm}
\begin{defn}[plus and minus]
  \label{defn:plus and minus}
  The preferred direction was called \emph{plus} (\rm{or} $+$ ) 
direction that produced the maximum response
in that neuron, and  its opposite direction is called the \emph{minus}
 (\rm{or} $-$ ) direction.
\end{defn}
\begin{exm}
  \label{fig:random-dot motion-
discrimination task}
  During the same experiment in Example \ref{exm:mokey experiment}, the judgment
  accuracy of the monkey and the optic nerve coding signal activity in
  the MT area were recorded. The experimental results show that: first,
  the coding of MT neural activity is basically sufficient for judging
  the direction; second, at high coherence levels, the fring-rate distributions
corresponding to the two directions are fairly well separated, while
at low coherence levels, they merge.
\begin{center}
  \centering
 \includegraphics[scale = 0.4]{png/3-2AB}
  \label{fig:3.2A}
\end{center}

\end{exm}

\begin{rem}
  Although spike count rates take only discrete values, it is more
  convenient to treat $r$ as a continuous variable for our
  discussion. Treated as probability densities, these two
  distributions are approximately Gaussian with the same variance, $\sigma_{r}^{2}$, but different means,
$\left\langle r \right\rangle_{+}$ for the plus direction and $\left\langle r \right\rangle_{-}$ for the minus direction.
\end{rem}

\begin{defn}[discriminability]
  \label{defn:discriminability}
  A convenient
measure of the separation between the distributions is the
\emph{discriminability}
\begin{equation}
  \label{eq:3.4}
  d'=\frac{\left\langle r \right\rangle_{+}-\left\langle r \right\rangle_{-}}{\sigma_{r}}.
\end{equation}
\end{defn}


\begin{rem}
 
Decoding involves using the neural
response to determine in which of the two possible directions the
stimulus moved for Example \ref{exm:mokey experiment}. A simple decoding procedure in this case is to determine the fring rate
$r$ during a trial and compare it to a threshold number $z$. If $r
\geq z$, we report “plus”; otherwise we report “minus”.
\end{rem}

\begin{defn}[size and power]
  \label{defn:size and power}
    Below are the probabilities of answering plus for both given the conditions:
  \begin{enumerate}[(i)]
  \item The probability that it will give the answer “plus”
when the stimulus is moving in the plus direction is the conditional probability that $r\geq z$ given a plus
stimulus, $\alpha(z)=P[r\geq z|+]$, called \emph{size} or \emph{false
  alarm} rate of the test.
\item The probability that it will give the answer “plus”
when the stimulus is actually moving in the minus direction (called a false
alarm) is similarly $\beta(z)=P[r\geq z|-]$, called \emph{power} or \emph{hit} rate of the test.
\end{enumerate}

These two probabilities completely determine the performance of the decoding procedure because the probabilities for the other two cases
\begin{center}
  \begin{tabular}[h]{|c|cc|}
\hline
         & \multicolumn{2}{c|}{probablity}          \\ \hline
stimulus & \multicolumn{1}{c|}{correct} & incorrect \\ \hline
$+$        & \multicolumn{1}{c|}{$\beta$}        &$1-\beta$       \\ \hline
$-$       & \multicolumn{1}{c|}{$1-\alpha$}     & $\alpha$         \\ \hline
\end{tabular}
\end{center}
\end{defn}


\subsection{ROC Curves}
\begin{defn}
  \label{def:ROC curves}
  The \emph{receiver operating characteristic} (\textbf{ROC}) curve is
  traced out as a function of the threshold $z$. Each point on an ROC
  curve corresponds to a different value of $z$. The $x$ coordinate of
  the point is $\alpha$, the size of the test for this value of $z$
 and the $y$ coordinate is $\beta$. ROC curve provides a way of
 evaluating how test performance depends on the choice of  $z$ and
 indicates how the size and power of a test trade off as the threshold
 is varied.
\end{defn}

\begin{exm}
  \label{exm:ROC curves}
The figure shows ROC curves computed by Britten et al. for several different values of the stimulus coherence.
\begin{enumerate}[(i)]
\item At high coherence levels, when
the task is easy, the ROC curve rises rapidly from $\alpha(z)=0$,
$\beta(z)=0$ as the threshold is lowered from a high value, and the
probability $\beta(z)$ of a correct “plus” answer quickly approaches
$1$ without a concomitant increase in $\alpha(z)$. As the threshold is
lowered further, the probability of giving the answer “plus” when
the correct answer is “minus” also rises, and $\alpha(z)$ increases.
\item At lower high coherence levels, when the task is difficult, the
  curve rises more slowly as $z$ is lowered.
\item At quite low coherence levels, the task is impossible, in that
  the test merely gives random answers, the curve will lie along the diagonal $\alpha=\beta$, because the probabilities of answers being correct and incorrect are equal.
\end{enumerate}
\begin{center}
    \includegraphics[scale = 0.5]{png/3-3}
 \end{center}
\end{exm}

\begin{rem}
  Examination of Example \ref{exm:ROC curves} suggests a relationship between
  the area under the ROC curve and the level of performance on the
  task. When the ROC curve in Example \ref{exm:ROC curves} lies along the diagonal, the
  area underneath it is $1/2$, which is the probability of a correct
  answer in this case (given any threshold). When the task is easy and
  the ROC curve hugs the left axis and upper limit, the area under it approaches $1$, which is again the
  probability of a correct answer (given an appropriate threshold). The
  area underneath the ROC curve is the probability of a correct answer
  in the most cases (given an appropriate threshold).

  However, the precise relationship between task performance and the area under
the ROC curve is complicated by the fact that different threshold values can be
used. This ambiguity can be removed by considering a slightly different
task, called \emph{two-alternative forced choice}.
\end{rem}

\begin{ntn}
  \label{ntn:two-alternative forced choice}
  For \emph{two-alternative forced choice}, the stimulus is presented
twice, once with motion in the plus direction and once in the minus di-
rection. The task is to decide which presentation corresponded to the plus
direction, given the fring rates on both trials, $r_1$ and $r_2$. A natural exten-
sion of the test procedure we have been discussing is to answer trial 1 if
$r_1 \geq r_2 $ and otherwise answer trial 2. This removes the threshold variable
from consideration.
\end{ntn}

\begin{prop}
  \label{prop:two-alternative correct probability}
 In the two-alternative force-choice task, the value of $r$ on one trial serves
as the threshold for the other trial. Then the probability of getting the correct answer
\begin{equation}
  \label{eq:3.6}
  P[\rm{correct}]=\int_{0}^{\infty}{p[z|-]\beta(z)dz}.
\end{equation}
 \begin{proof}
 For example, if the order of stimulus presentation is plus, then minus, the comparison procedure we have
outlined will report the correct answer if $r_1 \geq z$ where $z=r_2$,
and this has probability $P[r_1\geq z|+]=\beta(z)$ with $z=r_2$. For
small $\Delta z$, the probability that $r_2$ takes a value in the range
between $z$ and $z+\Delta z$ when the second trial has a minus
stimulus is $p[z|-]\Delta z$, where $p[z|-]$ is the conditional
fring-rate probability density for a fring rate $r=z$. Integrating over all values of $z$
gives the answer.
 \end{proof}
\end{prop}


\begin{prop}
  \label{prop:size}
The probability of getting the correct answer in the
Equation \ref{eq:3.6} can be transformed into
\begin{equation}
  \label{eq:3.5}
  P[\rm{correct}]=\int_0^1\beta d\alpha.
\end{equation}

 \begin{proof} 
   $\alpha(z)$ mentioned in definition \ref{defn:size and power}, can
be written as an integral of the conditional fring-rate probability density
$p[r|-]$,
\begin{equation}
  \label{eq:3.7}
  \alpha(z)=\int_{z}^{\infty}{p[r|-]dr}.
\end{equation}
   Taking the derivative of this equation with respect to $z$, we find
   that
   \begin{equation*}
       \label{eq:3.8}
       \frac{d\alpha}{dz}=-p[z|-].
     \end{equation*}
     This allows us to make the replacement $p[z|-]dz\rightarrow
     -d\alpha$ in the integral of Equation (\ref{eq:3.6}) and to change
     the integration variable from $z$ to $\alpha$. Noting that
     $\alpha=1$ when $z=0$ and $\alpha=0$ when $z=\infty$, we infer it.
\end{proof}
\end{prop}
\begin{rem}
  The ROC curve is just $\beta$ plotted as a function of $\alpha$, so
this integral is the area under the ROC curve. Thus, the area under
the ROC curve is the probability of responding correctly in the two-alternative forced-choice test.
\end{rem}

\begin{exc}
  Prove that suppose that $p[r|+]$ and $p[r|-]$ are both Gaussian functions with means
$\left\langle r \right\rangle_{+}$ and $\left\langle r
\right\rangle_{-}$, and a common variance $\sigma_r^{2}$. The reader
is invited to show that, in this case,
\begin{equation}
  \label{eq:3.10}
  P[\text{correct}]=\frac{1}{2}\text{erfc} \Big(\frac{
    \left<r\right>_{+}-\left<r\right>_{-}}{2\sigma_{r}}\Big)=\frac{1}{2}\text{erfc} \Big( -\frac{d'}{2} \Big),
\end{equation}
where $d'$ is the discriminability defined in equation (\ref{eq:3.4})
and $\rm{erfc}(x)$ is the complementary error function (which is an integral of a Gaussian distribution) defined as
\begin{equation*}
  \label{eq:3.11}
  \text{erfc}(x)=\frac{2}{\sqrt{\pi}}\int_x^{\infty}\text{exp}(-y^{2})dy.
\end{equation*}
\end{exc}
\begin{rem}
  $P[\rm{correct}]$ and $d'$ are positively correlated, that is to
  say,  the greater the difference in their firing rates, the greater
  the probability of accurate judgment. And in the case where the
  distributions are equalvariance Gaussians, the relationship between
  the discriminability and the area under the ROC curve is invertible because the complementary error function is monotonic.
\end{rem}

\subsection{ROC Analysis of Motion Discrimination}
\begin{rem}
  To interpret the experiment as a two-alternative forced-choice task, Brit
ten et al. imagined that, in addition to being given the fring rate of the
recorded neuron during stimulus presentation, the observer is given the
fring rate of a hypothetical “anti-neuron” having response
characteristics exactly opposite from the recorded neuron.
 In reality, the responses of this
anti-neuron to a plus stimulus were just those of the recorded neuron to a
minus stimulus, and vice versa. The idea of using the responses of a single
neuron to opposite stimuli as if they were the simultaneous responses of
two different neurons will also reappear in our discussion of spike-train decoding. An observer predicting motion directions on the basis of just these
two neurons at a level equal to the area under the ROC curve is termed an
ideal observer.
\end{rem}
\begin{rem}
   The figure A in Example \ref{fig:random-dot motion-
discrimination task} shows a typical result for the performance of an ideal observer
using one recorded neuron and its anti-neuron partner. The open circles in
figure were obtained by calculating the areas under the ROC curves
for this neuron. Amazingly, the ability of the ideal observer to perform
the discrimination task using a single neuron/anti-neuron pair is equal to
the ability of the monkey to do the task. This seems
remarkable because the monkey presumably has access to a large
population of neurons, while the ideal observer uses only two. 
\end{rem}



\subsection{The Likelihood Ratio Test}
\begin{lem}
  The discrimination test we have considered compares the fring rate
  to a fixed threshold value. The Neyman-Pearson lemma shows that it is
  optimal to choose the test function the ratio of
  probability densities (or probabilities),which also can be seen function of the fring rate
  \begin{equation}
    \label{eq:3.12}
    l(r)=\frac{p[r|+]}{p[r|-]},
  \end{equation}
  which is known as the \emph{likelihood ratio}.
  
  \begin{proof}
Consider the difference $\beta$ in the power of two tests that have identical
sizes $\alpha$. One uses the likelihood ratio $l(r)$, and the other uses a different
test function $h(r)$. For the test $h(r)$ using the threshold $z_{h}$,
\begin{equation}
  \label{eq:3.61}
  \begin{aligned}
    \alpha_h(z_h)&=\int {p[r|-]\Theta(h(r)-z_h)dr},\\
    \beta_h(z_h)&=\int {p[r|+]\Theta(h(r)-z_h)dr}.
  \end{aligned}
\end{equation}
Similar equations hold for the $\alpha_l(z_{l})$ and $\beta_l(z_l)$
values for the test $l(r)$ using
the threshold $z_{l}$. We use the $\Theta$  function, which is $1$ for positive and $0$ for
negative values of its argument, to impose the condition that the test is
greater than the threshold. Comparing the $\beta$ values for the two tests, we
find
\begin{equation}
  \label{eq:3.62}
  \begin{aligned}
    \nabla\beta&=\beta_l(z_l)-\beta_h(z_h)\\
                 &=\int{p[r|+]\Theta(l(r)-z_l)dr}-int {p[r|+]\Theta(h(r)-z_h)dr}.
  \end{aligned}
\end{equation}
The range of integration where $l(r)\geq z_l$ and also  $h(r)\geq z_{h}$  cancels between
these two integrals and use the definition $ l(r)=p[r|+]/{p[r|-]}$,
we can replace ${p[r|+]}$ with $l(r){p[r|-]}$ in this equation, giving
\begin{equation}
  \label{eq:3.64}
  \begin{split}
     \nabla\beta=\int{
    l(r)p[r|-]\left(\Theta(l(r)-z_l)\Theta(z_{h}-h(r))\right)dr}\\
    -\int {l(r)p[r|-]\left(\Theta(z_{l}-l(r))\Theta(h(r)-z_h) \right)dr}.
  \end{split}
  \end{equation}
  Then, due to the conditions imposed on $l(r)$ by the $\Theta$ functions within the
integrals, replacing $l(r)$ by $z$ can neither decrease the value of the integral
resulting from the first term in the large parentheses, nor increase the value
arising from the second. This leads to the inequality
\begin{equation}
  \label{eq:3.65}
  \begin{split}
    \nabla\beta\geq z\int {p[r|-]
    \Theta(l(r)-z_l)\Theta(z_{h}-h(r))dr}\\
    -z\int {p[r|-]\Theta(z_{l}-l(r))\Theta(h(r)-z_h)dr}.
  \end{split}
  \end{equation}
  Putting back the region of integration that cancels between these two
  terms (for which $l(r)\geq z_{l}$ and $h(r)\geq z_{h}$), we find
  \begin{equation}
    \label{eq:3.66}
    \nabla\beta\geq z\Big[\int {p[r|-]\Theta(l(r)-z_l)dr}-\int {p[r|-]\Theta(h(r)-z_h)dr}\Big].
  \end{equation}
  By definition, these integrals are the sizes of the two tests, which are equal
by hypothesis. Thus $\beta\geq 0$, showing
likelihood ratio $l(r)$, at least in the sense of maximizing the power for a
given size.
\end{proof}\qedhere
\end{lem}

\begin{rem}
  The test function $r$ used above is
not equal to the likelihood ratio. However, if the likelihood is a
monotonically increasing function of $r$, the fring-rate threshold test
is equivalent to using the likelihood ratio and is also indeed
optimal. Similarly, any monotonic function of the likelihood ratio will
provide as good a test as the likelihood itself, and the logarithm is frequently used.
\end{rem}



\begin{prop}
   There is a direct relationship between the likelihood ratio and the ROC
  curve. As in Equation \ref{eq:3.7} and (\ref{eq:3.8}), we can
  write
  \begin{equation}
  \label{eq:3.13}
  \beta(z)=\int_z^{\infty}p[r|+]dr \quad\text{so} \quad \frac{d\beta}{dz}=-p[z|+].
\end{equation}
Combining this result with Equation \ref{eq:3.8}, we find that
\begin{equation}
  \label{eq:3.14}
  \frac{d\beta}{d\alpha}=\frac{d\beta}{dz}\frac{dz}{d\alpha}=l(z),
\end{equation}
so the slope of the ROC curve is equal to the likelihood ratio.
\end{prop}

\begin{rem}
  Another way of seeing that comparing the likelihood ratio to a threshold
value is an optimal decoding procedure for discrimination uses a \emph{Bayesian
approach} based on associating a cost or penalty with getting the wrong answer.
\end{rem}

\begin{defn}
  The penalty associated with answering “minus” when
the correct answer is “plus” is quantifed by the \emph{loss parameter} $L_{-}$. Similarly, quantify the loss for answering “plus” when the correct answer is
“minus” as $L_{+}$.
\end{defn}

\begin{thm}
  The probabilities that the correct answer is
“plus” or “minus”, given the fring rate $r$, are $P[+|r]$ and $P[-|r]$ respectively. These probabilities are related to the conditional fring-rate probability densities by Bayes theorem,
\begin{equation}
  \label{eq:3.15}
  P[+|r]=\frac{p[r|+]P[+]}{p[r]}\quad \text{and}\quad P[-|r]=\frac{p[r|-]P[-]}{p[r]}.
\end{equation}
\end{thm}


\begin{prop}
  The average loss expected for a “plus” answer when the fring rate is $r$ is
the loss associated with being wrong times the probability of being wrong,
$\rm{Loss}_+=L_{+}P[-|r]$. Similarly, the expected loss when answering “minus”
is $\rm{Loss}_-=L_{-}P[+|r]$. A reasonable strategy is to cut the losses, answering
“plus” if $\rm{Loss}_{+}\le \rm{Loss}_{-} $ and “minus”
otherwise. Using Equation
\ref{eq:3.15}, we find that this strategy gives the response
“plus” if
\begin{equation}
  \label{eq:3.16}
  l(r)=\frac{p[r|+]}{p[r|-]}\geq \frac{L_+P[-]}{L_-P[+]}.
\end{equation}
This shows that the strategy of comparing the likelihood ratio to a threshold is a way of minimizing the expected loss.
\end{prop}


\begin{exm}
  If the conditional probability densities $p[r|+]$ and $p[r|-]$ are Gaussians
with means $r_{+}$ and $r_{-}$ and identical variances $\sigma_r^2$ ,
and $P[+]=P[-]=1/2$, the probability $P[+|r]$ is a sigmoidal function
of $r$,
\begin{equation}
  \label{eq:3.17}
  P[+|r]=\frac{1}{1+\exp(-d'(r-r_{ave})/\sigma_r)},
\end{equation}
where $r_{\rm{ave}}=(r_++r_-)/2$.
\end{exm}

\begin{exm}
  We have thus far considered discriminating between two quite distinct
stimulus values, plus and minus. Often we are interested in discriminating
between two stimulus values $s+\nabla s$ and $s$ that are very close to one another.
In this case, the likelihood ratio is
\begin{equation}
  \begin{aligned}
    \frac{p[r|s+\nabla s]}{p[r|s]} &\approx \frac{p[r|s]+\nabla
                                     s\partial p[r|s]/\partial s}{p[r|s]}
                                     &=1+\nabla \frac{\partial \ln
                                       p[r|s]}{\partial s}.
  \end{aligned}
\end{equation}
For small $\nabla s$, a test that compares
\begin{equation}
  \label{eq:3.19}
  Z(r)=\frac{\partial\ln p[r|s]}{\partial s},
\end{equation}
to a threshold $(z-1)/s$ is equivalent to the likelihood ratio test.
\end{exm}


%%% Local Variables:
%%% mode: latex
%%% TeX-master: "../notesOnFluidMechanics"
%%% TeX-master: t
%%% End:


\section{Entropy and Information for Spike Trains}
\label{sec:Entropy and Information for Spike Trains}


\begin{rem}
  Computing the entropy or information content of a neuronal response characterized by spike times is much more difficult than computing these quantities for responses described by firing rates. Nevertheless, these computations are important, because firing rates are incomplete descriptions that can lead to serious underestimates of the entropy and information.
\end{rem}

\begin{fac}
  \label{fac:spikeTrainInformation}
  Spike-train entropy calculations are typically based on the study of long-duration recordings consisting of many action potentials. The longer the total length of a spike train, the more information it contains.
\end{fac}

\begin{rem}
  By Fact \ref{fac:spikeTrainInformation}, the entropy and mutual information of spike trains are reported as entropy or information rates.
\end{rem}

\begin{defn}
  \label{def:entropyInformationRates}
  The \emph{entropy rate} and \emph{information rate} are defined as the total entropy and information divided by the duration of the spike train, respectively. Alternatively, entropy and mutual information can be divided by the total number of action potentials and reported as bits per spike rather than bits per second.
\end{defn}

\begin{ntn}
  We write the entropy rate as $\dot{H}$ rather than $H$.
\end{ntn}

\begin{fac}
  The temporal pattern of a group of action potentials can be specified by listing either the individual spike times or the sequence of intervals between successive spikes.
\end{fac}

\begin{rem}
  The entropy and mutual information calculations we present are based on a spike-time description, but as an initial example we consider an approximate computation of entropy using interspike intervals.
\end{rem}
\subsection{Based on Interspike Intervals}
\begin{rem}
  The interspike interval is a continuous variable.
\end{rem}
\begin{ntn}
  The probability of an interspike interval falling in the range between $\tau$ and $\tau+\Delta\tau$ is given in terms of the interspike interval probability density by $p[\tau]\Delta\tau$, where $\Delta\tau$ is the resolution.
\end{ntn}

\begin{prop}
  \label{prop:independentInterspike}
  If the different interspike intervals are statistically independent and identically distributed, the entropy associated with the interspike intervals in a spike train of average rate $\left<r\right>$ and of duration $T$ is
  \begin{displaymath}
    H = -\left<r\right>T\int_0^{\infty}p[\tau]\log_2(p[\tau]\Delta\tau)d\tau,
  \end{displaymath}
  where $\left<r\right>T$ is the number of intervals. In this case, the entropy rate is
  \begin{displaymath}
    \dot{H} = -\left<r\right>\int_0^{\infty}p[\tau]\log_2(p[\tau]\Delta\tau)d\tau.
  \end{displaymath}
\end{prop}
\begin{proof}
  These are directly from definitions of the entropy and entropy rate.
\end{proof}

\begin{exm}
  If a spike train is described by a homogeneous Poisson process with rate $\left<r\right>$, we have
  \begin{displaymath}
    p[\tau] = \left<r\right>e^{-\left<r\right>\tau}
  \end{displaymath}
  and the interspikes are statistically independent (Chapter \ref{cha:Neural Encoding I}). Thus,
  \begin{equation}
    \label{equ:4.53}
    \dot{H} = \frac{\left<r\right>}{\ln 2}(1-\ln\left<r\right>\Delta\tau).
  \end{equation}
  In fact,
  \begin{displaymath}
    \begin{aligned}
      \dot{H} &= -\left<r\right>\int_0^{\infty}\left<r\right>e^{-\left<r\right>\tau}\log_2(\left<r\right>e^{-\left<r\right>\tau}\Delta\tau)d\tau\\
      &= -\left<r\right>\int_0^{\infty}e^{-\tau}\log_2(\left<r\right>e^{-\tau}\Delta\tau)d\tau\\
      &= -\left<r\right>\left(\log_2(\left<r\right>\Delta\tau) + \int_0^{\infty}\frac{-e^{-\tau}}{\ln 2}d\tau\right) \\
      &= \frac{\left<r\right>}{\ln 2}(1-\ln\left<r\right>\Delta\tau),
    \end{aligned}
  \end{displaymath}
  where the second step follows from the variable substitution $\tau = \left<r\right> \tau$ and the third step from the integration by parts.
\end{exm}

\begin{defn}
  \label{PossionEntropyRate}
  Equation \ref{equ:4.53} is called the \emph{Poisson entropy rate}.
\end{defn}

\begin{thm}
  In general, the entropy rate $\dot{H}$ for a spike train with interspike interval distribution $p[\tau]$ and average rate $\left<r\right>$ satisfies
  \begin{equation}
    \label{equ:4.52}
    \dot{H} \leq -\left<r\right>\int_0^{\infty}p[\tau]\log_2(p[\tau]\Delta\tau)d\tau.
  \end{equation}
\end{thm}
\begin{proof}
  Correlations between different interspike intervals reduce the total entropy, so the result obtained by assuming independent intervals provides an upper bound on the true entropy of a spike train.
\end{proof}

\subsection{General Computations}

\begin{frm}
  To make entropy calculations practical, a long spike train is broken into statistically independent subunits, and the total entropy is written as the sum of the entropies for the individual subunits.
\end{frm}

\begin{exm}
  In the case of Proposition \ref{prop:independentInterspike}, the subunit was the interspike interval.
\end{exm}

\begin{rem}
  If interspike intervals are not independent, and we wish to compute a result and not merely a bound, we must work with larger subunit descriptions.
\end{rem}

\begin{ntn}
  The variable $T_s$ is used below to denote the duration of the spike sequence being considered, while $T$, which is much larger than $T_s$, is the duration of the entire spike train.
\end{ntn}

\begin{frm}
  Denote these basic subunits by spike sequences of duration $T_s$. A spike sequence can be obtained as follows.
  \begin{enumerate}[(i)]
  \item Divide time $T_{s}$ into discrete bins of size $\Delta t$, which is small enough so that not more than one spike appears in a bin.
  \item Label each bin by a 0 (no spike) or a 1 (spike), depending on whether or not a spike occurred within it.
  \item Represent a spike sequence defined over a block of duration $T_s$ by a string of $T_s/\Delta t$ zeros and ones.
  \end{enumerate}
  We denote such a sequence by $B(t)$, where $B$ is a $T_s/\Delta t$ bit binary number, and $t$ specifies the time of the first bin in the sequence being considered. Both $T_s$ and $t$ are integer multiples of the bin size $\Delta t$.
\end{frm}

\begin{ntn}
  The probability of a sequence $B$ occurring at any time during the entire response is denoted by $P[B]$.
\end{ntn}

\begin{rem}
  $P[B]$ can be obtained by counting the number of times the sequence $B$ occurs anywhere within the spike trains being analyzed (\emph{including overlapping cases}).
\end{rem}

\begin{prop}
  The spike-train entropy rate implied by the distribution that is characterized by $P[B]$ is
  \begin{equation}
    \label{equ:4.54}
    \dot{H} = -\frac{1}{T_{s}}\sum\limits_{B}P[B]\log_{2}P[B],
  \end{equation}
  where the sum is over all the sequences $B$ found in the data set, and we have divided by the duration $T_{s}$ of a single sequence to obtain an entropy rate.
\end{prop}
%\begin{proof}
 % Here $B$ is a possible value of the sequence over a block of duration $T_{s}$, which is the only one random variable involved.
%\end{proof}

\begin{prop}
  If the spike sequences in nonoverlapping intervals of duration $T_{s}$ are independent and identically distributed, the full spike-train entropy rate is also given by Equation \ref{equ:4.54}.
\end{prop}
\begin{proof}
  By the independence,
  \begin{displaymath}
    \begin{aligned}
      \dot{H} &= \frac{-T/T_{s}\sum\limits_{B}P[B]\log_{2}P[B]}{T} \\
      &= -\frac{1}{T_{s}}\sum\limits_{B}P[B]\log_{2}P[B],
    \end{aligned}
  \end{displaymath}
  which completes the proof.
\end{proof}

\begin{thm}
  For small $T_{s}$ such that the spike sequences are not independent, Equation \ref{equ:4.54} provides an upper bound on the true entropy rate, that is,
  \begin{equation}
    \label{eq:upper1}
    \dot{H} \leq -\frac{1}{T_{s}}\sum\limits_{B}P[B]\log_{2}P[B].
  \end{equation}
\end{thm}
\begin{proof}
  Any correlations between successive intervals (if $B(t+T_{s})$ is correlated with $B(t)$, for example) reduce the total spike-train entropy, causing Equation \ref{equ:4.54} to overestimate the true entropy rate.
\end{proof}

\begin{rem}
  If $T_{s}$ is too small, $B(t+T_{s})$ and $B(t)$ are likely to be correlated, and the overestimate may be severe. As $T_{s}$ increases, we expect the correlations to get smaller, and Equation \ref{equ:4.54} should provide a more accurate value.
\end{rem}

\begin{rem}
  For any finite data set, $T_{s}$ cannot be increased past a certain point, because there will not be enough spike sequences of duration $T_{s}$ in the data set to determine their probabilities. Thus, in practice, $T_{s}$ must be increased until the point where the extraction of probabilities becomes problematic, and some form of extrapolation to $T_{s}\to\infty$ must be made.
\end{rem}

\begin{asm}
  \label{asm:finite-true-relationship}
  Statistical mechanics arguments suggest that the difference between the entropy rate for finite $T_{s}$ and the true entropy rate for $T_{s}\to\infty$ should be proportional to $1/T_{s}$ for large $T_{s}$.
\end{asm}

\begin{prop}
  The true entropy rate can be estimated by linearly extrapolating a plot of the entropy rate versus $1/T_{s}$ to the point $1/T_{s} = 0$.
\end{prop}
\begin{proof}
  This is directly from Assumption \ref{asm:finite-true-relationship}.
\end{proof}

\begin{rem}
  To compute the mutual information rate for a spike train, we must subtract the full noise entropy rate from the full spike-train entropy rate.
\end{rem}

\begin{ntn}
  $P[B(t)]$ is the probability of finding a given sequence $B$ at time $t$ within a set of spike trains obtained on trials using the same stimulus. In contrast, $P[B]$, used in the spike-train entropy rate calculation, is the probability of finding the sequence $B$ at any time within these trains.
\end{ntn}

\begin{lem}
  \label{lem:noiseEntropyRate-t}
  If the same stimulus is used in repeated trials, the noise entropy rate at time $t$ satisfies
  \begin{displaymath}
    \dot{H}_{t} = -\frac{1}{T_{s}}\sum\limits_{B}P[B(t)]\log_{2}P[B(t)].
  \end{displaymath}
\end{lem}
\begin{proof}
  %Definition \ref{def:noiseEntropyRate-t} makes sense.
  The noise entropy rate is determined from the probabilities of finding various sequences $B$, given that they were evoked by the same stimulus. %This is done by considering sequences $B(t)$ that start at a fixed time $t$.
  If the same stimulus is used in repeated trials, sequences $B(t)$ that begin at time $t$ in every trial are generated by the same stimulus. Therefore, the conditional probability of the response, given the stimulus, is in this case the distribution $P[B(t)]$ for response sequences beginning at time $t$. This is obtained by determining the fraction of trials on which $B(t)$ was evoked.
\end{proof}

\begin{rem}
  Determining $P[B(t)]$ for a sufficient number of spike sequences may take a large number of trials using the same stimulus.
\end{rem}

\begin{prop}
  \label{prop:fullNoiseEntropyRate}
  The full noise entropy rate can be computed by averaging the noise entropy rate at time $t$ over all $t$ values, that is,
  \begin{equation}
    \label{equ:4.55}
    \dot{H}_{noise} = -\frac{\Delta t}{T}\sum\limits_{t}\left(\frac{1}{T_{s}}\sum\limits_{B}P[B(t)]\log_{2}P[B(t)]\right),
  \end{equation}
  where $T/\Delta t$ is the number of different $t$ values being summed.
\end{prop}
\begin{proof}
  In this case, the average over $t$ plays the role of the average over stimuli in Equation \ref{equ:4.6}. Then, Lemma \ref{lem:noiseEntropyRate-t} completes the proof.
\end{proof}

\begin{thm}
  If Equation \ref{equ:4.55} is based on finite-length spike sequences, it provides an upper bound on the noise entropy rate, that is,
  \begin{equation}
    \label{equ:upper2}
    \dot{H}_{noise} \leq -\frac{\Delta t}{T}\sum\limits_{t}\left(\frac{1}{T_{s}}\sum\limits_{B}P[B(t)]\log_{2}P[B(t)]\right).
  \end{equation}
\end{thm}

\begin{prop}
  The true noise entropy rate is estimated by performing a linear extrapolation in $1/T_s$ to $1/T_s = 0$.
\end{prop}
\begin{proof}
  As was done for the spike-train entropy rate.
\end{proof}

\begin{exm}
  Entropy and noise entropy rates for the H1 visual neuron in the fly responding to a randomly moving visual image are shown in the following picture. (i) The filled circles in the upper trace show the full spike-train entropy rate computed for different values of $1/T_s$. The straight line is a linear extrapolation to $1/T_s = 0$, which corresponds to $T_s\to \infty$. (ii) The lower trace shows the spike train noise entropy rate for different values of $1/T_s$. The straight line is again an extrapolation to $1/T_s = 0$.
  \begin{center}
    \includegraphics[scale=0.45]{./png/entropyRateEst}
  \end{center}
  Both entropy rates increase as functions of $1/T_s$, and the true spike-train and noise entropy rates are overestimated at large values of $1/T_s$. At $1/T_s\approx 20/s$, there is a sudden shift in the dependence. This occurs when there is insufficient data to compute the spike sequence probabilities. 
  % The difference between the $y$ intercepts of the two straight lines plotted is the mutual information rate.
  By linearly extrapolating the linear part of the series of computed points spike trains had an approximate entropy rate of 157 bits/s and an appeoximate noise entropy rate of 79 bits/s when the resolution was $\Delta t = 3$ ms. The information rate is obtained by taking the difference between the extrapolated values for the spiketrain and noise entropy rates. The result is an information rate of 157 - 79 = 78 bits/s or 1.8 bits/spike.
\end{exm}

\begin{rem}
  Both the spike-train and noise entropy rates depend on $\Delta t$. The leading dependence, coming from the $\log_{2}\Delta t$ term discussed previously, cancels in the computation of the information rate, but the information can still depend on $\Delta t$ through nondivergent terms. This reflects the fact that more information can be extracted from accurately measured spike times than from poorly measured spike times. Thus, we expect the information rate to increase with decreasing $\Delta t$, at least over some range of $\Delta t$ values.  At some critical value of $\Delta t$ that matches the natural degree of noise jitter in the spike timings, we expect the information rate to stop increasing. This value of $\Delta t$ is interesting because it tells us about the degree of spike timing accuracy in neural encoding.
\end{rem}

\begin{rem}
  The information conveyed by spike trains can be used to compare responses to different stimuli and thereby reveal stimulus-specific aspects of neural encoding.
\end{rem}


 



%%% Local Variables:
%%% mode: latex
%%% TeX-master: "../notesOnFluidMechanics"
%%% End:




\begin{asm}
  \label{asm:surfaceForceViscous}
  From now on, 
  assume that
  \begin{equation}
    \label{equ:surfaceForceSigma}
    \text{force on $S$ per unit area} = -p(\mathbf{x}, t)\mathbf{n}+\mathbf{n}\cdot\boldsymbol\sigma(\mathbf{x}, t), 
  \end{equation}
  where $\boldsymbol\sigma$ is the \emph{(deviatoric) stress tensor} and
  $\mathbf{n}$ is the unit outward normal of $S$.
\end{asm}

%%% Local Variables:
%%% mode: latex
%%% TeX-master: "../notesOnFluidMechanics"
%%% End:
\newpage
\section{Spike-Train Statistics}
\label{sec:1.4}


\begin{rem}
        A complete description of the stochastic relationship between a stimulus and a response would require us to know the probabilities corresponding to every sequence of spikes that can be evoked by the stimulus.    
\end{rem}

\begin{lem}
    The probability that $z$ takes a value between $z$ and $z+ \Delta z$, for small $\Delta$(strictly speaking, as $\Delta z \to 0$), is equal to $p[z]\Delta z$, where $p[z]$ is called a probability density.
\end{lem}

\begin{ntn}    
    Throughout this book,  we use the notation $P$[\ ] to denote probabilities and $p$[\ ] to denote probability densities.
\end{ntn}    


\begin{thm}
    The probability of a spike sequence appearing is proportional to the probability density of spike times,  $p[t_1, t_2, ..., t_n]$. In other words, the probability $P[t_1,t_2,...,t_n]$ that a sequence of n spikes occurs with spike $i$ falling between times $t_i$ and $t_i+\Delta t$ for $i= $1,2,...,n is given in terms of this density by the relation 
    \begin{equation}
        P[t_1,t_2,...,t_n]=p[t_1,t_2,...,t_n](\Delta t)^n.        
    \end{equation}
    % This relationship is a special case of Equation \ref{equ:1.37} derived below.
    \begin{proof}
        \small
        $$P[t_1,t_2,...,t_n]=\int... \int p[s_1,s_2,...,s_n]dS\\$$
        $$=\int^{t_n+\Delta t/2}_{t_n-\Delta t/2} 
        \int^{t_{n-1}+\Delta t/2}_{t_{n-1}-\Delta t/2} ...\int^{t_1+\Delta t/2}_{t_1-\Delta t/2} p[s_1,s_2,...,s_n]ds_1 ...ds_{n-1}ds_{n}\\$$
  \\ According to the integral mean value theorem ( $\Delta t \to 0  $ )\\
        $\Rightarrow  P[t_1,t_2,...,t_n]=p[t_1,t_2,...,t_n](\Delta t)^n.        $
        
    \end{proof}
\end{thm}

\begin{defn}[\emph{point process}]
    A stochastic process that generates a sequence of events, such as action potentials ,is called a point process.     
\end{defn}

\begin{rem}
    In general, the probability of an event occurring at any given time could depend on the entire history of preceding events. 
\end{rem}

\begin{defn}[\emph{renewal process}]
    If this dependence extends only to the immediately preceding event, so that the intervals between successive events are independent, the point process is called a renewal process.
\end{defn}

\begin{defn}
    The Poisson process provides an extremely useful approximation of stochastic neuronal firing.
    To make the presentation easier to follow, we separate two cases, the homogeneous Poisson process, for which the firing rate is constant over time, and the inhomogeneous Poisson process, which involves a time-dependent firing rate.
\end{defn}

\subsection{The Homogeneous Poisson Process}

\begin{ntn}
    We denote the firing rate for a homogeneous Poisson process by r$(t)=$r, because it is independent of time.
\end{ntn}

\begin{defn}[\emph{probality of $n$ spikes occuring}]
     The probality that an arbitrary sequence of exactly $n$ spikes occurs within a trial of duration $T$ is $P_T[n]$.
\end{defn}

\begin{thm}
    For a homogeneous Poisson process, the Poisson distribution is 
    \begin{equation}
        P_T[n]=\frac{(rn)^n}{n}exp(-rT).
        \label{equ:1.29}
    \end{equation}
    \begin{proof}
        To compute $P_T[n]$, we divide the time T into M bins of size $\Delta t =T/M$. We assume that $\Delta t$ is small enough so that we never get two spikes within any one bin because, at the end of the calculation,we take the limit $\Delta t \to 0$.\\
        $P_T[n]$ is the product of three factors: \\
            (a)\ The probability of generating $n$ spikes within a  specified set of the $M$ bins,$\frac{M!}{(M-n)!n!}$;\\
            (b)\ The probability of not generating spikes in the remaining $M - n$ bins,$(r\Delta t)^n$;\\
            (c)\ A combinatorial factor equal to the number of ways of putting $n$ spikes into $M$ bins,$(1-r\Delta t)^{M-n}$; \\
            \text{    To sum up,}            
            \begin{equation}
            \label{equ:1.27}
            P_T[n]=\lim_{\Delta t \to 0}\frac{M!}{(M-n)!n!}(r\Delta t)^n(1-r\Delta t)^{M-n}.
        \end{equation}
        As $\Delta t \to 0, M$ grows without bound because $ M\Delta t=T$. Because n is fixed, we can write $M-n\approx M=T/\Delta t$. Using this approximatin and defining $\epsilon=-r\Delta t$, we find that 
        \begin{equation}
            \lim_{\Delta t \to 0}(1-r\Delta t)^{M-n}=\lim_{\epsilon\to 0}(((1+\epsilon)^{\frac{1}{\epsilon}})^{-rT}=\exp(-rT)
        \end{equation}
        For large $M,\ \frac{M!}{(M-n)!}\approx M^n=(T/\Delta t)^n$, so
        \begin{equation}            
            P_T[n]=\frac{(rn)^n}{n}exp(-rT).
        \end{equation}
    \end{proof}
\end{thm}

\begin{exm}
    The probabilities $P_T[n]$, for a few $n$ values, are plotted as a function of $rT$ in the following firgue. Note that as $n$ increase, the probability reaches its maximum at larger $T$ values and that large $n$ values are more likely than small ones for large $T$.
\end{exm}    

\begin{center}
    \label{fig:1.11}                
        \includegraphics[scale = 0.36]{png/Figure1-11-A}\\        
\end{center}

\begin{exm}
    The following figure shows the probabilities of various numbers of spikes occurring when the average number of spikes is $10$. For large $rT$, which corresponds to a large expected number of spikes, the Poisson distribution approaches a Gaussian distribution with mean and variance equal to $rT$. This figure shows that this approximation is already quite good for $rT = 10$.
\end{exm}    

\begin{center}
    \label{fig:1.12}            
    \includegraphics[scale = 0.36]{png/Figure1-11-B}\\    
\end{center}

\begin{thm}
    The probability $P[t_1,t_2,...,t_n]$ can be expressed in terms of another probability function $P_T[n]$, which is the probality that an arbitrary sequence of exactly $n$ spikes occurs within a trial of duration $T$. Assuming that the spike times are ordered $0\leq t_1\leq t_2\leq ...\leq t_n\leq T$, so that, the relationship is 
    \begin{equation}
        P[t_1,t_2,...,t_n]=n!{P_T[n]\left (\frac{\Delta t}{T}\right )^n}.
        \label{equ:1.26}
    \end{equation}
    \begin{proof}
        % represents
        The probability of docking is $ n!(\frac{\Delta t}{T})^n $ in a specific time order $(t_1,t_2,...,t_n).$  so,
       \begin{align}       
         &P[t_1,t_2,...,t_n]={P_T[n]}(n(\frac{\Delta t}{T})(n-1)(\frac{\Delta t}{T})...1(\frac{\Delta t}{T}))\\
        &=n!{P_T[n]\left(\frac{\Delta t}{T}\right)^n}
    \end{align}
    \end{proof}
\end{thm}

\begin{coro}
    We can compute the variance of spike counts produced by a Poisson process from the probabilities in Equation \ref{equ:1.29}. The spike count is 
    \begin{equation}
        \sigma^2_n = \langle n^2 \rangle -\langle n  \rangle ^2=rT.
    \end{equation}
    \begin{proof}
    The average number of spikes generated by a Poisson process with constasnt rate $r$ over a time $T$ is 
    \begin{equation}
        \langle n\rangle=\sum_{n=0}^\infty nP_T[n]=\sum_{n=0}^\infty\frac{n(rT)^n}{n!}\exp(-rT).
        \label{equ:1.45}
    \end{equation}
    and the variance in the spike count is
    \begin{equation}
        \sigma_n^2(T)=\sum_{n=0}^\infty n^2P_T[n]-\langle n\rangle^2=\sum_{n=0}^\infty\frac{n^2(rT)^n}{n!}\exp(-rT)-\langle n\rangle^2.
        \label{equ:1.46}
        \end{equation}
        To compute the quantities,we need to calculate the two sums appearing in these Equations.A good way to do this is to compute the moment-generating function
        \begin{equation}
            g(\alpha)=\sum_{n=0}^\infty\frac{(rT)^n\exp(\alpha n)}{n!}\exp(-rT).
            \label{equ:1.47}
        \end{equation}      
        The $k$th derivative of g with respect to $\alpha$,evaluated at the point $\alpha=0$, is
        \begin{equation}
            \frac{dg}{d\alpha^k}|_{\alpha=0}=\sum_{n=0}^\infty\frac{n^k(rT)^n}{n!}\exp(-rT),
            \label{equ:1.48}
        \end{equation}        
    so once we have computed $g$,we need to calculate only its first and second derivative to determine the sums we need. Rearranging the terms a bit, and recalling that $\exp(z)=\sum z^n/n!$, we find\\        
    \begin{equation}
        g(\alpha)=\exp(-rT)\sum_{n=0}^\infty\frac{(rT\exp(\alpha))^n}{n!}=\exp(-rT)\exp(rTe^\alpha).
        \label{equ:1.49}
    \end{equation}
    The derivatives are then \\
    \begin{equation}
        \frac{dg}{d\alpha}=rTe^\alpha \exp(-rT)\exp(rTe^\alpha)
        \label{equ:1.50}
    \end{equation}
    and\\
    \begin{equation}
    \small    \frac{d^g}{d\alpha^2}=(rTe^\alpha)^2\exp(-rT)\exp(rTe^\alpha)+rTe^\alpha \exp(-rT)\exp(rTe^\alpha).
        \label{equ:1.51}
    \end{equation}
    Evaluating these at $\alpha=0$and putting the results into Equation \ref{equ:1.45} and \ref{equ:1.46} gives the result $\langle n\rangle=rT$ and $$\sigma_n^2(T)=(rT)^2+rT-(rT)^2=rT.$$
    \end{proof}
\end{coro}

\begin{defn}[\emph{Fano factor}]
    % (Fano factor)The ratio of the variance and mean of the spike count ,$\sigma^2_n/\langle n\rangle$,is called the Fano factor.
    The ratio of the variance and mean of the spike count,
   $     \sigma^2_n/\langle n\rangle$, is called the Fano factor.            
\end{defn}

\begin{exm}
    The Fano factor takes the value $1$ for a homogeneous Poisson process, independent of the time interval $T$.
\end{exm}

\begin{lem}
    % For a homogeneous Poisson process,the probability of an interspike intervalfalling between $\tau$ and $\tau + \Delta t$ is $$P[\tau\leq t_{i+1}-t_{i}<\tau +\Delta t]=r\Delta t\ \exp(-r\tau)$$.
    The probability of an interspike intervalfalling between $\tau$ and $\tau + \Delta t$ is 
    \begin{equation}
        P[\tau\leq t_{i+1}-t_{i}<\tau +\Delta t]=r\Delta t\ \exp(-r\tau).
        \label{equ:1.31}
    \end{equation}
    \begin{proof}
        Suppose that a spike occurs at a time $t_i$ for some value of $i$. The probability of a homogeneous Poisson process generating the next spike somewhere in the interval $$t_i+\tau \leq t_{i+1} \leq t_i + \tau +\Delta t,$$ for small $\Delta t$, is the probabilities that no spike is fired for a time $\tau$, times the probability, $r\Delta t$, of  generating a spike within the following small interval $\Delta t$. From Equation \ref{equ:1.29}, with $n=0$, the probability of not firing a spike for period $\tau$ is $\exp(-r\tau)$. So the probability of an interspike interval falling between $\tau$ and $\tau+\Delta t$ is $$  P[\tau\leq t_{i+1}-t_{i}<\tau +\Delta t]=r\Delta t\ \exp(-r\tau).$$
    \end{proof}
\end{lem}

\begin{thm}
    From the interspike interval distribution of a homogeneous Poisson spike train,  we can compute the mean interspike interval, 
    \begin{equation}
        \langle \tau \rangle =\int^{\infty}_{0}\tau r\ \exp(-r\tau)d\tau  = \frac{1}{r}
        \label{equ:1.32}         
    \end{equation}
    and the variance of the interspike intervals, 
    \begin{equation}
        \sigma^2_\tau =\int^{\infty}_{0}\tau^2 r\ \exp(-r\tau)d\tau - \langle \tau \rangle^2 = \frac{1}{r^2}.
        \label{equ:1.33}         
    \end{equation}
\end{thm}

\begin{defn}
    % [\emph{coefficient of variation}]
    The ratio of the standard deviation and the mean of interspike interval distribution.
    \begin{equation}
        C_V=\frac{\sigma_\tau}{\langle \tau  \rangle},
        \label{equ:1.34}
    \end{equation} is the \emph{the coefficient of variation}
\end{defn}

\begin{rem}
    The coefficient of variation takes the value $1$ for a homogeneous Poisson process. This is a necessary,  though not sufficient, condition to identify a Poisson spike train. Recall that the Fano factor for a Poisson process is also $1$. For any renewal process, the Fano factor evaluated over long time intervals approaches the value $C^2_V$.
\end{rem}

\subsection{The Spike-Train Autocorrelation Funciton}

\begin{defn}
        The spike-train autocorrelation function,
        \begin{equation}
            Q_{\rho\rho}(\tau)=\frac{1}{T}\int^T_0 \langle (\rho(t)-\langle r \rangle)(\rho(t+\tau)-\langle r\rangle)\rangle dt,
            \label{equ:1.35}
        \end{equation} is the autocorrelation of the neural response function of Equation \ref{equ:1.1} with its average over time and trials substracted out. 
\end{defn}

\begin{thm}
    The autocorrelation function for a Poisson spike train generated at a constant rate $\langle r \rangle =r$ is 
    \begin{equation}
        Q_{\rho\rho}(\tau)=r\delta(\tau)
    \end{equation}
    \begin{proof}
        The spike-train auto correlation function is constructed from data in the form of a histogram by dividing time into bins. The value of the histogram for a bin labeled with a positive or negative integer $m$ is computed by determining the number of the times that any two spikes in the train are separated by a time interval lying between $(m-1/2)\Delta t$ and $(m+1/2)\Delta $ with $\Delta t$ the bin size.  This includes all pairings, even  between a spike and itself. We call this number $N_m$. If the intervals between the $n^2$ spike pairs in the train were uniformly distributed over the range from $0$ to $T$, there would be $n^2\Delta t/T$ intervals in each bin. This uniform term is removed from the autocorrelation histogram by subtracting $n^2\Delta t /T$ from $N_m$ for all $m$. The spike-train autocorrelation histogram is then defined by dividing the resulting numbers by $T$, so the value of the histogram in bin m is $H_m=N_m/T-n^2\Delta /T^2$. For small bin sizes, the $m = 0$ term in the histogram counts the average number of spikes,  that is $N_m = \langle n \rangle $ and in the limit $\Delta t \to 0,\ H_0=\langle n \rangle /T$ is the average firing rate $\langle r \rangle$. Because other bins have $H_m$ of order $\Delta t$, large $m = 0$ term is often removed from histogram plots. The spike-train autocorrlation function is defined as $H_m/\Delta t$ in the limit $\Delta t \to 0$, and it has the units of a firing rate squared. In this limit,  the $m = 0$ bin becomes a $\delta $funcitn, $H_0/\Delta t\to \langle r\rangle \delta (\tau)$.\\
        As we can seen, the distribution of interspikde intervals for adjacent spikes in a homogeneous Poisson spike train is exponential(Equation \ref{equ:1.31}). By contrast, the intervals between any two spikes(not necessarily adjacent) in such a train are uniformly distributed. As a result,  the subtraction procedure outlined above gives $H_m=0$ for all bins except for the $m=0$ bin that contains the contribution of the zero intervals between spikes and themselves. The autocorrlation function for a Poisson spike train generated at a constant rate $\langle r\rangle = r$ is 
        $$        Q_{\rho\rho}(\tau)=r\delta(\tau).$$
    \end{proof}
\end{thm}

\begin{defn}
The spike-train correlation function ,  
\begin{equation}
    Q_{\rho_1 \rho_2}(\tau)=\frac{1}{T}\int^T_0 \langle (\rho_1(t)-\langle r_1 \rangle)(\rho_2(t+\tau)-\langle r_2\rangle)\rangle dt, 
    \label{equ:1.35}
\end{equation}
    is the correlation of different neural response function $\rho_1(t)$ and $\rho_2(t)$ with their average over time and trials which are $r_1$ and $r_2$ substracted out.
    % (problem)
\end{defn}

\begin{rem}
    The spike-train autocorrelation function is an even function of $\tau$, $ Q_{\rho\rho}(\tau)=Q_{\rho\rho}(-\tau)$, but the cross-correlation function is not necessarily even.
\end{rem}

\begin{exm}
    Asymmetric shifts in this peak away from 0 result from fixed delays between the firing of the twoneurons, and they indicate nonsynchronous but phase-locked firing.    
    Periodic structure in either an autocorrelation or a cross-correlation function or histogram indicates that the firing probability oscillates. Such periodic structure is seen in the histograms of the following firgue, showing 40 Hz oscillations in neurons of catprimary visual cortex that are roughly synchronized between the two cerebral hemispheres.
\end{exm}
% (problem)
\begin{center}
    \label{fig:1.12A}    
    \includegraphics[scale = 0.36]{png/Figure1-12-A.png}
% \end{center}
% \begin{center}
    \label{fig:1.12B} 
    \includegraphics[scale = 0.36]{png/Figure1-12-B.png}\\
\end{center}

\subsection{The Inhomogeneous Poisson Process}
\begin{thm}

The probability density of the inhomogeneous Poisson Process for $n$ spike times is 
    \begin{equation}
    p[t_1, t_2, ..., t_n]=\exp\left(-\int^T_0r(t)dt\right)\prod^n_{i=1}r(t_i),
        \label{equ:1.37}
    \end{equation}
    The spike times are ordered $0\leq t_1 \leq t_2\leq ... \leq t_n \leq T.$
    \begin{proof}
    The probability density for a particular spike sequence with spike times $t_i$ for $i = 1, 2, ..., n$ is obtained from the corresponding probability distribution by multiplying the probability that the spikes occur when they do by the probability that no other spikes occur.We begin by computing the probability that no spikes are generated during the time interval from $t_i$ to $t_{i+1}$ between two adjacent spikes. We determine this by dividing the interval into M bins of size $\Delta t$ and setting $M\Delta t=t_{i+1}-t_i$. We will ultimately take the limit $\Delta t\to 0$. The firing rate during bin $m$ within this interval is $r(t_i+m\Delta t)$. Because the probability of firing a spike in this bin is $r(t_i+m\Delta t)\Delta t$, the probabilities of not firing a spike is $1-r(t_i+m\Delta t)\Delta t$. To have no spikes during the entire interval, we must string together $M$ such bins,  and the probability of this occurring is the product of the individual probabilities, 
            \begin{equation}
            P[\text{no spikes}]=\prod_{m=1}^M(1-r(t_i+m\Delta t)\Delta t).
            \label{equ:1.52}
            \end{equation}
    We evaluate this expression by taking its logarithm,             
            \begin{equation}
            \ln P[\text{no spikes}]=\sum_{m=1}^M\ln(1-r(t_i+m\Delta t)\Delta t),
            \label{equ:1.53}
            \end{equation}
    using the fact that the logarithm of a product is the sum of the logarithms of the multiplied terms. Using the approximation $\ln (1-r(t_i+m\Delta t)\Delta t)\approx -r(t_i+m\Delta t)\Delta t$,  valid for small $\Delta t$, we can simplify this to 
            \begin{equation}
            \ln P[\text{no spikes}]=-\sum_{m=1}^Mr(t_i+m\Delta t)\Delta t.
            \label{equ:1.54}
            \end{equation}
    In the limit $\Delta t \to 0$, the approximation becomes exact and this sum becomes the  integral of $r(t)$ from $t_i$ to $t_{i+1}$, 
            \begin{equation}
            \ln P[\text{no spikes}]=-\int_{t_i}^{t_{i+1}}r(t)dt.
            \label{equ:1.55}
            \end{equation}
    Exponentiating this Equation gives the result we need,             
            \begin{equation}
            P[\text{no spikes}]=\exp\left(-\int_{t_i}^{t_{i+1}}r(t)dt\right).
            \label{equ:1.56}
            \end{equation}
    The probability density $p[t_1, t_2, ..., t_n]$is the product of the densities for the individual spikes and the probabilities of not generating spikes during the interspikde intervals, between time $0$ and the first spike,  and between the time of the last spike and the end of the trial period:            
            \begin{equation}
            \begin{aligned}
            p[t_1, t_2, ...t_n]=\exp\left(-\int_0^{t_1}r(t)dt\right)\exp\left(-\int_{t_n}^Tr(t)dt\right)\times \\  r(t_n)\prod_{i=1}^{n-1}r(t_i)\exp\left(-\int_{t_i}^{t_{i+1}}r(t)dt\right).
            \end{aligned}
            \label{equ:1.57}
            \end{equation}
    The exponentials in this expression all combine because the product of exponentials is the exponential of the sum, so the different integrals in this sum add up to form a single integral:            
            \begin{equation}
                \small
            \begin{aligned}
            &\exp\left(-\int_0^{t_1}r(t)dt)\right)\exp\left(-\int_{t_n}^Tr(t)dt\right)\prod_{i=1}^{n-1}\exp\left(-\int_{t_i}^{t_{i+1}}r(t)dt\right)\\
            &=\exp\left(-\left(\int_0^{t_1}r(t)dt+\sum_{i=1}^{n-1}\int_{t_i}^{t_{i+1}}r(t)dt+\int_{t_n}^Tr(t)dt\right)\right)\\
            &=\exp\left(-\int_0^Tr(t)dt\right) .
            \end{aligned}
            \label{equ:1.58}
            \end{equation}
            Substituting this into Equation \ref{equ:1.57} gives the result in Equation \ref{equ:1.37}
    \end{proof}
\end{thm}

\begin{rem}
The eqution \ref{equ:1.26} is a special case of Equation \ref{equ:1.37}.
\end{rem}

\subsection{The Poisson Spike Generator}

\begin{rul}[\emph{Estimated firing rate}]
    Spike sequences can be simulated by using some estimate of the firing rate, $r_\text{est}(t)$, predicted from knowledge of the stimulus,  to drive a Poisson process.
\end{rul}

\begin{alg}
    The program progresses through time in small steps of size $\Delta t$ and generates, at each time step, a random number $x_{\text{rand}}$ chosen uniformly in the range between $0$ and $1$. If $r_{\text{est}}(t)\Delta t > x_{\text{rand}}$ at that time step, a spike is fired; otherwise it is not.
    % is $r_{est}(t)\Delta t$.
\end{alg}

\begin{alg}
    For a constant firing rate, it is faster to compute spike times $t_i$ for $i=1,2,...,n$ iteratively by generating interspike intervals from an exponential probability density(Equation \ref{equ:1.31}). Thus  we can generate spike times iteratively from the formula $t_{i+1}= t_i-\ln(x_\text{rand}/r)$.
    
 \end{alg}
\begin{rem}
    If $x_\text{rand}$ is uniformly distributed over the range between $0$ and $1$, the negative of its logarithm is exponentially distributed.
\end{rem}
 \begin{alg}[\emph{Spike thinning}]
    The thinning technique requires a bound $r_\text{max}$ on the estimated firing rate such that $r_{\text{est}}(t) \leq r_\text{max}$    at all times. We first generate a spike sequence corresponding to the constant rate $r_{max}$ by iterating the rule $t_{i+1} = t_i - \ln(x_{\text{rand}})/r_\text{max}$. The spike are then thinned by generating another $x_{\text{rand}}$ for each $i$ and removing the spike at time $t_i$ from the train if $r_{\text{est}(t_i)}/r_{\text{max}} < x_{\text{rand}}$. If $r_\text{est}(t_i) / r_{\text{max}} \geq x_{\text{rand}}$, spike $i$ is retained. Thinning corrects for the difference between the estimated timedependent rate and the maximum rate.
    \end{alg} 

\begin{exm}
    The following figures shows an example of a model of an orientation-selective V1 neuron constructed by  Spike thinning. In this model,  the estimated firing rate is determined from the response tuning curve
    \begin{equation}
        r_{est}(t)=f(s(t))=r_{max}\exp\left(-\frac{1}{2}\left(\frac{s(t)-s_{max}}{\sigma_f}\right)^2\right).
        \label{equ:1.38}
    \end{equation}
    % (problem)
\end{exm}

\begin{center}
    \label{fig:1.13A}    
    \includegraphics[scale = 0.2]{png/Figure1-13-A.png}\\
\end{center}

\begin{center}
    \label{fig:1.13B}    
    \includegraphics[scale = 0.2]{png/Figure1-13-B.png}\\
\end{center}

\begin{center}
    \label{fig:1.13C}    
    \includegraphics[scale = 0.2]{png/Figure1-13-C.png}\\
\end{center}
This figure Model of an orientation-selective neuron. The orientation angle (top
panel) was increased from an initial value of $-40^\circ$  by $20^\circ $  every $100$ ms. The firing
rate (middle panel) was used to generate spikes (bottom panel) using a Poisson
spike generator. The bottom panel shows spike sequences generated on five different trials.

\subsection{Comparison with Data}
\begin{rem}
    The Poisson process is simple and useful, but does it match data on neural response variability? To address this question,  we examine Fano factors, interspike interval distributions,  and coefficients of variation.
\end{rem}

\begin{prop}
    The Fano factor describes the relationship between the mean spike count over a given interval and the spike-count variance. Mean spike counts $\langle n \rangle $ and variances $\sigma^2_n$ from a wide variety of neuronal recordings have been fitted to the Equation $\sigma^2_n = A\langle n\rangle^B $, and the \emph{multiplier} $A$ and exponent B have been determined. The values of both $A$ and $B$ typically lie between 
    $1.0$ and $1.5.$
\end{prop}

\begin{rem}
    Because the Poisson model predicts $A = B = 1$, this indicates
that the data show a higher degree of variability than the Poisson model
would predict. However, many of these experiments involve anesthetized
animals, and it is known that response variability is higher in anesthetized
than in alert animals.
\end{rem}


\begin{exm}[\emph{comparison of the Fano factor}]
    The following figures shows data for spike-count means and variances extracted
from recordings of MT neurons in alert macaque monkeys using a number of different stimuli. The MT (medial temporal) area is a visual region of the primate cortex where many neurons are sensitive to image motion.
The individual means and variances are scattered in figure A,  but they
cluster around the diagonal which is the Poisson prediction. Similarly,  the
results show A and B values close to $1$,  the Poisson values (figure B).
Of course,  many neural responses cannot be described by Poisson statistics,  but it is reassuring to see a case where the Poisson model seems a
reasonable approximation. As mentioned previously,  when spike trains
are not described very accurately by a Poisson model,  refractory effects
are often the primary reason.
\end{exm}
\begin{center}
    \label{fig:1.14A}    
    \includegraphics[scale = 0.36]{png/Figure1-14-A.png}\\
\end{center}

\begin{center}
    \label{fig:1.14B}    
    \includegraphics[scale = 0.36]{png/Figure1-14-B.png}\\
\end{center}

\begin{center}
    \label{fig:1.14C}    
    \includegraphics[scale = 0.36]{png/Figure1-14-C.png}\\
\end{center}

\begin{alg}
    Interspike interval distributions are extracted from data as interspike histograms by counting the number of intervals falling in discrete time bins.
\end{alg}

\begin{exm}[\emph{the Poisson model of interspike interval}]
    The following figure presents an example from the responses of a nonbursting cell in area MT of a monkey in response to images consisting of randomly moving dots with a variable amount of coherence imposed on
    their motion (see chapter $3$ for a more detailed description). 
\end{exm}

\begin{center}
    \label{fig:1.15A}    
    % \includegraphics[scale = 0.36]{png/Figure1-15-A.png}\\
    \includegraphics[trim=30 0 0 60,clip,scale = 0.36]{png/Figure1-15-A.png}\\        
\end{center}
For interspike intervals longer than about 10 ms, the shape of this histogram is exponential, in agreement with Equation \ref{equ:1.31}. However, for shorter intervals there is a discrepancy. While the homogeneous Poisson distribution of Equation \ref{equ:1.31} rises for short interspike intervals, the experimental results show a rapid decrease. This is the result of refractoriness making short interspike intervals less likely than the Poisson model would predict.
\begin{rem}
\end{rem}
\begin{prop}
    The data of the Poisson model of interspike interval with a stochastic refractory period can be fitted more accurately by a gamma distribution, 
    \begin{equation}
        p[\tau] = \frac{r(r\tau)^k\exp(-r\tau)}{k!}
        \label{equ:1.39}
    \end{equation}
    with $k>0$, than by the exponential distribution of the Poisson model, which has $k = 0$.
\end{prop}

\begin{exm}[\emph{the Poisson model of interspike interval with a stochastic refractory period}]
    The following figure shows a theoretical histogram obtained by adding a refractory period of variable duration to the Poisson model. Spiking was prohibited during the refractory period,  and then was described once again by a homogeneous Poisson process. The refractory period was randomly chosen from a Gaussian distribution with a mean of $5$ ms and a standard
deviation of $2$ ms (only random draws that generated positive refractory periods were included). The resulting interspike interval distribution of figure \ref{fig:1.15B} agrees quite well with the data.
\end{exm}

\begin{center}
    \label{fig:1.15B}    
    % \includegraphics[scale = 0.36]{png/Figure1-15-B.png}\\
    \includegraphics[trim=10 10 30 30,clip,scale = 0.36]{png/Figure1-15-B.png}\\    
\end{center}

\begin{exm}[\emph{comparion of the coefficients of variation}]
    $C_V$ values extracted from the spike trains of neurons recorded in monkeys from area MT and primary visual cortex(V1) are shown in this figure. The data have been divided into groups based on the mean interspike interval,  and the coefficient of variation is plotted as a function of the mean interval,  equivalent to $1/\langle r\rangle$. Except for short mean interspike intervals, the values are near $1$, although they tend to cluster slightly lower than $1$, the Poisson value. The small $C_V$ values for short interspike intervals are due to the refractory period. The solid curve is the prediction of a Poisson  model with refractoriness.
\end{exm}

\begin{center}
    \label{fig:1.16}    
    \includegraphics[scale = 0.36]{png/Figure1-16.png}\\
\end{center}

\begin{rem}
    However,  there are cases in which the accuracy in the timing and numbers of spikes fired by a neuron is considerably higher than would be implied by Poisson statistics. 
    Furthermore,  even when it successfully describes data,  the Poisson model does not provide a mechanistic explanation of neuronal response variability.
\end{rem}

\begin{exm}
    The following figure compares the response of V1 cells to constant current injection in vivo and in vitro. The in vitro response is a regular and reproducible spike train(left panel). The same current injection paradigm applied in vivo produces a highly irregular pattern of firing(center panel) similar to the response to a moving bar stimulus(right panel).
\end{exm}

\begin{center}
    \label{fig:1.17}    
    \includegraphics[scale = 0.25]{png/Figure1-17.png}\\
\end{center}
Although some of the basic statistical properties of firing variability may be captured by the Poisson model of spike generation,  the spike generating mechanism itself in real neurons is clearly not responsible for the variability. We explore ideas about possible sources of spike-train variability in chapter $5$.

\begin{rem}
    Some neurons fire action potentials in clusters or bursts of spikes that can not be described by a Poisson process with a fixed rate. Bursting can be included in a Poisson model by allowing the firing rate to fluctuate in order to describe the high rate of firing during a burst. Sometimes the distribution of bursts themselves can be described by a Poisson process (such a doubly stochastic process is called a Cox process).    
\end{rem}





% problem: 推导确认一下
% problem: 那些例子如何修改
% problem: 概念如何改变
% problem: tex规格 ok
% problem: 修改exp  ok
% problem: 具体的排版  ok
% problem: 以及一些公式 ok
% problem: 图像无法正确索引 ok


\end{multicols}

\chapter{ Neural Encoding II: Reverse Correlation and Visual Receptive Fields}
\label{cha:Neural Encoding II}

\begin{multicols}{2}
\setlength{\columnseprule}{0.2pt}  

\section{Introduction}
\subsection{The Explanation of Some Terms}
\rem \emph{Neurons} are highly specialized for generating electrical signals in response to chemical and other inputs, and transmitting them to other cells.
\rem \emph{Dendrites} receives information inputs from other neurons.
\rem \emph{Axon} carries the neuronal output to other cells.
\begin{center}
    \label{fig:1.1}
    \includegraphics[scale = 0.35]{png/Figure1-1}\\
\end{center}

\rem \emph{Ion channels} control the flow of ions across the cell membrane by opening and closing in response to voltage changes and to both internal and external signals.
\begin{center}
  \label{fig:1.2}
  \includegraphics[scale = 0.55]{png/Figure1-2}\\
\end{center}
  
\con [Membrane Potential]The potential difference between two solutions separated by membranes, generally refers to the electrical phenomenon accompanying the life activities of cells, which exists on both side of cells.
\rem Under resting conditions,the potential inside the cell membrane(mainly K$^+$) is negative, outside the cell membrane(mainly Na$^+$) is positive, and the cell is said to be \emph{polarized}.
\defn [Action Potential] \emph{Action potential} is the characteristic electrical pulses or, more simply, spikes that can travel down nerve fibers.
\con [Hyperpolarization]Current in the form of positively charged ions flowing out of the cell (or negatively charged ions flowing into the cell) through open channels makes the membrane potential more negative, a process called \emph{hyperpolarization}.
\con [Depolarization]Current flowing into the cell changes the membrane potential to less negative or even positive values. This is called \emph{depolarization}.
\rem If a neuron is depolarized sufficiently to raise the membrane potential above a threshold level, a positive feedback process is initiated, and the neuron generates an \emph{action potential}.
\con [Absolute Refractory Period]For a few milliseconds just after an action potential has been fired, it may be virtually impossible to initiate another spike.
\con [Relative Refractory Period]After the absolute refractory period, the excitability of cells gradually recovers. After stimulation, excitement can occur, but the stimulation must be greater than the original threshold intensity.
\rem \emph{Absolute refractory period} and \emph{relative refractory period} are two basic phenomena in the process of neural response.
\subsection{Recording Neuronal Responses}
\exm Intracellular and extracellular methods for recording neuronal responses electrically
\begin{center}
    \label{fig:1.3}
    \includegraphics[scale = 0.35]{png/Figure1-3}\\
\end{center}

\begin{enumerate}[(i)]
  \item The top trace represents a recording from an intracellular electrode connected to the soma of the neuron.
  \item The middle trace is a simulated extracellular recording.
  \item The bottom trace represents a recording from an intracellular electrode connected to the axon some distance away from the soma.
\end{enumerate}




\subsection{From Stimulus to Response}
\rem  Neurons typically respond by producing complex spike sequences that reflect both the intrinsic dynamics of the neuron and the temporal characteristics of the stimulus.
\defn Neural encoding refers to the map from stimulus to response.
\exm We can catalog how neurons respond to a wide variety of stimuli, and then construct models that attempt to predict responses to other stimuli.
\defn Neural decoding refers to the reverse map, from response to stimulus.
\rem The complexity and trial-to-trial variability of action potential sequences make it unlikely that we can describe and predict the timing of each spike deterministically. Instead, we seek a model that can account for the probabilities that different spike sequences are evoked by a specific stimulus.


%%% Local Variables:
%%% mode: latex
%%% TeX-master: t
%%% End:


\section{Discrimination}
\label{sec:Single-cell decoding}
\begin{exm}
  \label{exm:mokey experiment}
  In the experiments performed by Britten et al. (1992).,
a monkey was trained to discriminate between two directions of motion
of a visual stimulus which was a pattern of dots on a video monitor. The percentage of dots that move together in the fixed direction is
called the coherence level. By varying the degree of coherence shown by
pictures, the task of detecting the movement direction can be made more or less
diffcult.
\begin{center} 
 \includegraphics[scale = 0.3]{./png/3-1}
\end{center}
\end{exm}
\begin{defn}[plus and minus]
  \label{defn:plus and minus}
  The preferred direction was called \emph{plus} (\rm{or} $+$ ) 
direction that produced the maximum response
in that neuron, and  its opposite direction is called the \emph{minus}
 (\rm{or} $-$ ) direction.
\end{defn}
\begin{exm}
  \label{fig:random-dot motion-
discrimination task}
  During the same experiment in Example \ref{exm:mokey experiment}, the judgment
  accuracy of the monkey and the optic nerve coding signal activity in
  the MT area were recorded. The experimental results show that: first,
  the coding of MT neural activity is basically sufficient for judging
  the direction; second, at high coherence levels, the fring-rate distributions
corresponding to the two directions are fairly well separated, while
at low coherence levels, they merge.
\begin{center}
  \centering
 \includegraphics[scale = 0.4]{png/3-2AB}
  \label{fig:3.2A}
\end{center}

\end{exm}

\begin{rem}
  Although spike count rates take only discrete values, it is more
  convenient to treat $r$ as a continuous variable for our
  discussion. Treated as probability densities, these two
  distributions are approximately Gaussian with the same variance, $\sigma_{r}^{2}$, but different means,
$\left\langle r \right\rangle_{+}$ for the plus direction and $\left\langle r \right\rangle_{-}$ for the minus direction.
\end{rem}

\begin{defn}[discriminability]
  \label{defn:discriminability}
  A convenient
measure of the separation between the distributions is the
\emph{discriminability}
\begin{equation}
  \label{eq:3.4}
  d'=\frac{\left\langle r \right\rangle_{+}-\left\langle r \right\rangle_{-}}{\sigma_{r}}.
\end{equation}
\end{defn}


\begin{rem}
 
Decoding involves using the neural
response to determine in which of the two possible directions the
stimulus moved for Example \ref{exm:mokey experiment}. A simple decoding procedure in this case is to determine the fring rate
$r$ during a trial and compare it to a threshold number $z$. If $r
\geq z$, we report “plus”; otherwise we report “minus”.
\end{rem}

\begin{defn}[size and power]
  \label{defn:size and power}
    Below are the probabilities of answering plus for both given the conditions:
  \begin{enumerate}[(i)]
  \item The probability that it will give the answer “plus”
when the stimulus is moving in the plus direction is the conditional probability that $r\geq z$ given a plus
stimulus, $\alpha(z)=P[r\geq z|+]$, called \emph{size} or \emph{false
  alarm} rate of the test.
\item The probability that it will give the answer “plus”
when the stimulus is actually moving in the minus direction (called a false
alarm) is similarly $\beta(z)=P[r\geq z|-]$, called \emph{power} or \emph{hit} rate of the test.
\end{enumerate}

These two probabilities completely determine the performance of the decoding procedure because the probabilities for the other two cases
\begin{center}
  \begin{tabular}[h]{|c|cc|}
\hline
         & \multicolumn{2}{c|}{probablity}          \\ \hline
stimulus & \multicolumn{1}{c|}{correct} & incorrect \\ \hline
$+$        & \multicolumn{1}{c|}{$\beta$}        &$1-\beta$       \\ \hline
$-$       & \multicolumn{1}{c|}{$1-\alpha$}     & $\alpha$         \\ \hline
\end{tabular}
\end{center}
\end{defn}


\subsection{ROC Curves}
\begin{defn}
  \label{def:ROC curves}
  The \emph{receiver operating characteristic} (\textbf{ROC}) curve is
  traced out as a function of the threshold $z$. Each point on an ROC
  curve corresponds to a different value of $z$. The $x$ coordinate of
  the point is $\alpha$, the size of the test for this value of $z$
 and the $y$ coordinate is $\beta$. ROC curve provides a way of
 evaluating how test performance depends on the choice of  $z$ and
 indicates how the size and power of a test trade off as the threshold
 is varied.
\end{defn}

\begin{exm}
  \label{exm:ROC curves}
The figure shows ROC curves computed by Britten et al. for several different values of the stimulus coherence.
\begin{enumerate}[(i)]
\item At high coherence levels, when
the task is easy, the ROC curve rises rapidly from $\alpha(z)=0$,
$\beta(z)=0$ as the threshold is lowered from a high value, and the
probability $\beta(z)$ of a correct “plus” answer quickly approaches
$1$ without a concomitant increase in $\alpha(z)$. As the threshold is
lowered further, the probability of giving the answer “plus” when
the correct answer is “minus” also rises, and $\alpha(z)$ increases.
\item At lower high coherence levels, when the task is difficult, the
  curve rises more slowly as $z$ is lowered.
\item At quite low coherence levels, the task is impossible, in that
  the test merely gives random answers, the curve will lie along the diagonal $\alpha=\beta$, because the probabilities of answers being correct and incorrect are equal.
\end{enumerate}
\begin{center}
    \includegraphics[scale = 0.5]{png/3-3}
 \end{center}
\end{exm}

\begin{rem}
  Examination of Example \ref{exm:ROC curves} suggests a relationship between
  the area under the ROC curve and the level of performance on the
  task. When the ROC curve in Example \ref{exm:ROC curves} lies along the diagonal, the
  area underneath it is $1/2$, which is the probability of a correct
  answer in this case (given any threshold). When the task is easy and
  the ROC curve hugs the left axis and upper limit, the area under it approaches $1$, which is again the
  probability of a correct answer (given an appropriate threshold). The
  area underneath the ROC curve is the probability of a correct answer
  in the most cases (given an appropriate threshold).

  However, the precise relationship between task performance and the area under
the ROC curve is complicated by the fact that different threshold values can be
used. This ambiguity can be removed by considering a slightly different
task, called \emph{two-alternative forced choice}.
\end{rem}

\begin{ntn}
  \label{ntn:two-alternative forced choice}
  For \emph{two-alternative forced choice}, the stimulus is presented
twice, once with motion in the plus direction and once in the minus di-
rection. The task is to decide which presentation corresponded to the plus
direction, given the fring rates on both trials, $r_1$ and $r_2$. A natural exten-
sion of the test procedure we have been discussing is to answer trial 1 if
$r_1 \geq r_2 $ and otherwise answer trial 2. This removes the threshold variable
from consideration.
\end{ntn}

\begin{prop}
  \label{prop:two-alternative correct probability}
 In the two-alternative force-choice task, the value of $r$ on one trial serves
as the threshold for the other trial. Then the probability of getting the correct answer
\begin{equation}
  \label{eq:3.6}
  P[\rm{correct}]=\int_{0}^{\infty}{p[z|-]\beta(z)dz}.
\end{equation}
 \begin{proof}
 For example, if the order of stimulus presentation is plus, then minus, the comparison procedure we have
outlined will report the correct answer if $r_1 \geq z$ where $z=r_2$,
and this has probability $P[r_1\geq z|+]=\beta(z)$ with $z=r_2$. For
small $\Delta z$, the probability that $r_2$ takes a value in the range
between $z$ and $z+\Delta z$ when the second trial has a minus
stimulus is $p[z|-]\Delta z$, where $p[z|-]$ is the conditional
fring-rate probability density for a fring rate $r=z$. Integrating over all values of $z$
gives the answer.
 \end{proof}
\end{prop}


\begin{prop}
  \label{prop:size}
The probability of getting the correct answer in the
Equation \ref{eq:3.6} can be transformed into
\begin{equation}
  \label{eq:3.5}
  P[\rm{correct}]=\int_0^1\beta d\alpha.
\end{equation}

 \begin{proof} 
   $\alpha(z)$ mentioned in definition \ref{defn:size and power}, can
be written as an integral of the conditional fring-rate probability density
$p[r|-]$,
\begin{equation}
  \label{eq:3.7}
  \alpha(z)=\int_{z}^{\infty}{p[r|-]dr}.
\end{equation}
   Taking the derivative of this equation with respect to $z$, we find
   that
   \begin{equation*}
       \label{eq:3.8}
       \frac{d\alpha}{dz}=-p[z|-].
     \end{equation*}
     This allows us to make the replacement $p[z|-]dz\rightarrow
     -d\alpha$ in the integral of Equation (\ref{eq:3.6}) and to change
     the integration variable from $z$ to $\alpha$. Noting that
     $\alpha=1$ when $z=0$ and $\alpha=0$ when $z=\infty$, we infer it.
\end{proof}
\end{prop}
\begin{rem}
  The ROC curve is just $\beta$ plotted as a function of $\alpha$, so
this integral is the area under the ROC curve. Thus, the area under
the ROC curve is the probability of responding correctly in the two-alternative forced-choice test.
\end{rem}

\begin{exc}
  Prove that suppose that $p[r|+]$ and $p[r|-]$ are both Gaussian functions with means
$\left\langle r \right\rangle_{+}$ and $\left\langle r
\right\rangle_{-}$, and a common variance $\sigma_r^{2}$. The reader
is invited to show that, in this case,
\begin{equation}
  \label{eq:3.10}
  P[\text{correct}]=\frac{1}{2}\text{erfc} \Big(\frac{
    \left<r\right>_{+}-\left<r\right>_{-}}{2\sigma_{r}}\Big)=\frac{1}{2}\text{erfc} \Big( -\frac{d'}{2} \Big),
\end{equation}
where $d'$ is the discriminability defined in equation (\ref{eq:3.4})
and $\rm{erfc}(x)$ is the complementary error function (which is an integral of a Gaussian distribution) defined as
\begin{equation*}
  \label{eq:3.11}
  \text{erfc}(x)=\frac{2}{\sqrt{\pi}}\int_x^{\infty}\text{exp}(-y^{2})dy.
\end{equation*}
\end{exc}
\begin{rem}
  $P[\rm{correct}]$ and $d'$ are positively correlated, that is to
  say,  the greater the difference in their firing rates, the greater
  the probability of accurate judgment. And in the case where the
  distributions are equalvariance Gaussians, the relationship between
  the discriminability and the area under the ROC curve is invertible because the complementary error function is monotonic.
\end{rem}

\subsection{ROC Analysis of Motion Discrimination}
\begin{rem}
  To interpret the experiment as a two-alternative forced-choice task, Brit
ten et al. imagined that, in addition to being given the fring rate of the
recorded neuron during stimulus presentation, the observer is given the
fring rate of a hypothetical “anti-neuron” having response
characteristics exactly opposite from the recorded neuron.
 In reality, the responses of this
anti-neuron to a plus stimulus were just those of the recorded neuron to a
minus stimulus, and vice versa. The idea of using the responses of a single
neuron to opposite stimuli as if they were the simultaneous responses of
two different neurons will also reappear in our discussion of spike-train decoding. An observer predicting motion directions on the basis of just these
two neurons at a level equal to the area under the ROC curve is termed an
ideal observer.
\end{rem}
\begin{rem}
   The figure A in Example \ref{fig:random-dot motion-
discrimination task} shows a typical result for the performance of an ideal observer
using one recorded neuron and its anti-neuron partner. The open circles in
figure were obtained by calculating the areas under the ROC curves
for this neuron. Amazingly, the ability of the ideal observer to perform
the discrimination task using a single neuron/anti-neuron pair is equal to
the ability of the monkey to do the task. This seems
remarkable because the monkey presumably has access to a large
population of neurons, while the ideal observer uses only two. 
\end{rem}



\subsection{The Likelihood Ratio Test}
\begin{lem}
  The discrimination test we have considered compares the fring rate
  to a fixed threshold value. The Neyman-Pearson lemma shows that it is
  optimal to choose the test function the ratio of
  probability densities (or probabilities),which also can be seen function of the fring rate
  \begin{equation}
    \label{eq:3.12}
    l(r)=\frac{p[r|+]}{p[r|-]},
  \end{equation}
  which is known as the \emph{likelihood ratio}.
  
  \begin{proof}
Consider the difference $\beta$ in the power of two tests that have identical
sizes $\alpha$. One uses the likelihood ratio $l(r)$, and the other uses a different
test function $h(r)$. For the test $h(r)$ using the threshold $z_{h}$,
\begin{equation}
  \label{eq:3.61}
  \begin{aligned}
    \alpha_h(z_h)&=\int {p[r|-]\Theta(h(r)-z_h)dr},\\
    \beta_h(z_h)&=\int {p[r|+]\Theta(h(r)-z_h)dr}.
  \end{aligned}
\end{equation}
Similar equations hold for the $\alpha_l(z_{l})$ and $\beta_l(z_l)$
values for the test $l(r)$ using
the threshold $z_{l}$. We use the $\Theta$  function, which is $1$ for positive and $0$ for
negative values of its argument, to impose the condition that the test is
greater than the threshold. Comparing the $\beta$ values for the two tests, we
find
\begin{equation}
  \label{eq:3.62}
  \begin{aligned}
    \nabla\beta&=\beta_l(z_l)-\beta_h(z_h)\\
                 &=\int{p[r|+]\Theta(l(r)-z_l)dr}-int {p[r|+]\Theta(h(r)-z_h)dr}.
  \end{aligned}
\end{equation}
The range of integration where $l(r)\geq z_l$ and also  $h(r)\geq z_{h}$  cancels between
these two integrals and use the definition $ l(r)=p[r|+]/{p[r|-]}$,
we can replace ${p[r|+]}$ with $l(r){p[r|-]}$ in this equation, giving
\begin{equation}
  \label{eq:3.64}
  \begin{split}
     \nabla\beta=\int{
    l(r)p[r|-]\left(\Theta(l(r)-z_l)\Theta(z_{h}-h(r))\right)dr}\\
    -\int {l(r)p[r|-]\left(\Theta(z_{l}-l(r))\Theta(h(r)-z_h) \right)dr}.
  \end{split}
  \end{equation}
  Then, due to the conditions imposed on $l(r)$ by the $\Theta$ functions within the
integrals, replacing $l(r)$ by $z$ can neither decrease the value of the integral
resulting from the first term in the large parentheses, nor increase the value
arising from the second. This leads to the inequality
\begin{equation}
  \label{eq:3.65}
  \begin{split}
    \nabla\beta\geq z\int {p[r|-]
    \Theta(l(r)-z_l)\Theta(z_{h}-h(r))dr}\\
    -z\int {p[r|-]\Theta(z_{l}-l(r))\Theta(h(r)-z_h)dr}.
  \end{split}
  \end{equation}
  Putting back the region of integration that cancels between these two
  terms (for which $l(r)\geq z_{l}$ and $h(r)\geq z_{h}$), we find
  \begin{equation}
    \label{eq:3.66}
    \nabla\beta\geq z\Big[\int {p[r|-]\Theta(l(r)-z_l)dr}-\int {p[r|-]\Theta(h(r)-z_h)dr}\Big].
  \end{equation}
  By definition, these integrals are the sizes of the two tests, which are equal
by hypothesis. Thus $\beta\geq 0$, showing
likelihood ratio $l(r)$, at least in the sense of maximizing the power for a
given size.
\end{proof}\qedhere
\end{lem}

\begin{rem}
  The test function $r$ used above is
not equal to the likelihood ratio. However, if the likelihood is a
monotonically increasing function of $r$, the fring-rate threshold test
is equivalent to using the likelihood ratio and is also indeed
optimal. Similarly, any monotonic function of the likelihood ratio will
provide as good a test as the likelihood itself, and the logarithm is frequently used.
\end{rem}



\begin{prop}
   There is a direct relationship between the likelihood ratio and the ROC
  curve. As in Equation \ref{eq:3.7} and (\ref{eq:3.8}), we can
  write
  \begin{equation}
  \label{eq:3.13}
  \beta(z)=\int_z^{\infty}p[r|+]dr \quad\text{so} \quad \frac{d\beta}{dz}=-p[z|+].
\end{equation}
Combining this result with Equation \ref{eq:3.8}, we find that
\begin{equation}
  \label{eq:3.14}
  \frac{d\beta}{d\alpha}=\frac{d\beta}{dz}\frac{dz}{d\alpha}=l(z),
\end{equation}
so the slope of the ROC curve is equal to the likelihood ratio.
\end{prop}

\begin{rem}
  Another way of seeing that comparing the likelihood ratio to a threshold
value is an optimal decoding procedure for discrimination uses a \emph{Bayesian
approach} based on associating a cost or penalty with getting the wrong answer.
\end{rem}

\begin{defn}
  The penalty associated with answering “minus” when
the correct answer is “plus” is quantifed by the \emph{loss parameter} $L_{-}$. Similarly, quantify the loss for answering “plus” when the correct answer is
“minus” as $L_{+}$.
\end{defn}

\begin{thm}
  The probabilities that the correct answer is
“plus” or “minus”, given the fring rate $r$, are $P[+|r]$ and $P[-|r]$ respectively. These probabilities are related to the conditional fring-rate probability densities by Bayes theorem,
\begin{equation}
  \label{eq:3.15}
  P[+|r]=\frac{p[r|+]P[+]}{p[r]}\quad \text{and}\quad P[-|r]=\frac{p[r|-]P[-]}{p[r]}.
\end{equation}
\end{thm}


\begin{prop}
  The average loss expected for a “plus” answer when the fring rate is $r$ is
the loss associated with being wrong times the probability of being wrong,
$\rm{Loss}_+=L_{+}P[-|r]$. Similarly, the expected loss when answering “minus”
is $\rm{Loss}_-=L_{-}P[+|r]$. A reasonable strategy is to cut the losses, answering
“plus” if $\rm{Loss}_{+}\le \rm{Loss}_{-} $ and “minus”
otherwise. Using Equation
\ref{eq:3.15}, we find that this strategy gives the response
“plus” if
\begin{equation}
  \label{eq:3.16}
  l(r)=\frac{p[r|+]}{p[r|-]}\geq \frac{L_+P[-]}{L_-P[+]}.
\end{equation}
This shows that the strategy of comparing the likelihood ratio to a threshold is a way of minimizing the expected loss.
\end{prop}


\begin{exm}
  If the conditional probability densities $p[r|+]$ and $p[r|-]$ are Gaussians
with means $r_{+}$ and $r_{-}$ and identical variances $\sigma_r^2$ ,
and $P[+]=P[-]=1/2$, the probability $P[+|r]$ is a sigmoidal function
of $r$,
\begin{equation}
  \label{eq:3.17}
  P[+|r]=\frac{1}{1+\exp(-d'(r-r_{ave})/\sigma_r)},
\end{equation}
where $r_{\rm{ave}}=(r_++r_-)/2$.
\end{exm}

\begin{exm}
  We have thus far considered discriminating between two quite distinct
stimulus values, plus and minus. Often we are interested in discriminating
between two stimulus values $s+\nabla s$ and $s$ that are very close to one another.
In this case, the likelihood ratio is
\begin{equation}
  \begin{aligned}
    \frac{p[r|s+\nabla s]}{p[r|s]} &\approx \frac{p[r|s]+\nabla
                                     s\partial p[r|s]/\partial s}{p[r|s]}
                                     &=1+\nabla \frac{\partial \ln
                                       p[r|s]}{\partial s}.
  \end{aligned}
\end{equation}
For small $\nabla s$, a test that compares
\begin{equation}
  \label{eq:3.19}
  Z(r)=\frac{\partial\ln p[r|s]}{\partial s},
\end{equation}
to a threshold $(z-1)/s$ is equivalent to the likelihood ratio test.
\end{exm}


%%% Local Variables:
%%% mode: latex
%%% TeX-master: "../notesOnFluidMechanics"
%%% TeX-master: t
%%% End:


\section{Entropy and Information for Spike Trains}
\label{sec:Entropy and Information for Spike Trains}


\begin{rem}
  Computing the entropy or information content of a neuronal response characterized by spike times is much more difficult than computing these quantities for responses described by firing rates. Nevertheless, these computations are important, because firing rates are incomplete descriptions that can lead to serious underestimates of the entropy and information.
\end{rem}

\begin{fac}
  \label{fac:spikeTrainInformation}
  Spike-train entropy calculations are typically based on the study of long-duration recordings consisting of many action potentials. The longer the total length of a spike train, the more information it contains.
\end{fac}

\begin{rem}
  By Fact \ref{fac:spikeTrainInformation}, the entropy and mutual information of spike trains are reported as entropy or information rates.
\end{rem}

\begin{defn}
  \label{def:entropyInformationRates}
  The \emph{entropy rate} and \emph{information rate} are defined as the total entropy and information divided by the duration of the spike train, respectively. Alternatively, entropy and mutual information can be divided by the total number of action potentials and reported as bits per spike rather than bits per second.
\end{defn}

\begin{ntn}
  We write the entropy rate as $\dot{H}$ rather than $H$.
\end{ntn}

\begin{fac}
  The temporal pattern of a group of action potentials can be specified by listing either the individual spike times or the sequence of intervals between successive spikes.
\end{fac}

\begin{rem}
  The entropy and mutual information calculations we present are based on a spike-time description, but as an initial example we consider an approximate computation of entropy using interspike intervals.
\end{rem}
\subsection{Based on Interspike Intervals}
\begin{rem}
  The interspike interval is a continuous variable.
\end{rem}
\begin{ntn}
  The probability of an interspike interval falling in the range between $\tau$ and $\tau+\Delta\tau$ is given in terms of the interspike interval probability density by $p[\tau]\Delta\tau$, where $\Delta\tau$ is the resolution.
\end{ntn}

\begin{prop}
  \label{prop:independentInterspike}
  If the different interspike intervals are statistically independent and identically distributed, the entropy associated with the interspike intervals in a spike train of average rate $\left<r\right>$ and of duration $T$ is
  \begin{displaymath}
    H = -\left<r\right>T\int_0^{\infty}p[\tau]\log_2(p[\tau]\Delta\tau)d\tau,
  \end{displaymath}
  where $\left<r\right>T$ is the number of intervals. In this case, the entropy rate is
  \begin{displaymath}
    \dot{H} = -\left<r\right>\int_0^{\infty}p[\tau]\log_2(p[\tau]\Delta\tau)d\tau.
  \end{displaymath}
\end{prop}
\begin{proof}
  These are directly from definitions of the entropy and entropy rate.
\end{proof}

\begin{exm}
  If a spike train is described by a homogeneous Poisson process with rate $\left<r\right>$, we have
  \begin{displaymath}
    p[\tau] = \left<r\right>e^{-\left<r\right>\tau}
  \end{displaymath}
  and the interspikes are statistically independent (Chapter \ref{cha:Neural Encoding I}). Thus,
  \begin{equation}
    \label{equ:4.53}
    \dot{H} = \frac{\left<r\right>}{\ln 2}(1-\ln\left<r\right>\Delta\tau).
  \end{equation}
  In fact,
  \begin{displaymath}
    \begin{aligned}
      \dot{H} &= -\left<r\right>\int_0^{\infty}\left<r\right>e^{-\left<r\right>\tau}\log_2(\left<r\right>e^{-\left<r\right>\tau}\Delta\tau)d\tau\\
      &= -\left<r\right>\int_0^{\infty}e^{-\tau}\log_2(\left<r\right>e^{-\tau}\Delta\tau)d\tau\\
      &= -\left<r\right>\left(\log_2(\left<r\right>\Delta\tau) + \int_0^{\infty}\frac{-e^{-\tau}}{\ln 2}d\tau\right) \\
      &= \frac{\left<r\right>}{\ln 2}(1-\ln\left<r\right>\Delta\tau),
    \end{aligned}
  \end{displaymath}
  where the second step follows from the variable substitution $\tau = \left<r\right> \tau$ and the third step from the integration by parts.
\end{exm}

\begin{defn}
  \label{PossionEntropyRate}
  Equation \ref{equ:4.53} is called the \emph{Poisson entropy rate}.
\end{defn}

\begin{thm}
  In general, the entropy rate $\dot{H}$ for a spike train with interspike interval distribution $p[\tau]$ and average rate $\left<r\right>$ satisfies
  \begin{equation}
    \label{equ:4.52}
    \dot{H} \leq -\left<r\right>\int_0^{\infty}p[\tau]\log_2(p[\tau]\Delta\tau)d\tau.
  \end{equation}
\end{thm}
\begin{proof}
  Correlations between different interspike intervals reduce the total entropy, so the result obtained by assuming independent intervals provides an upper bound on the true entropy of a spike train.
\end{proof}

\subsection{General Computations}

\begin{frm}
  To make entropy calculations practical, a long spike train is broken into statistically independent subunits, and the total entropy is written as the sum of the entropies for the individual subunits.
\end{frm}

\begin{exm}
  In the case of Proposition \ref{prop:independentInterspike}, the subunit was the interspike interval.
\end{exm}

\begin{rem}
  If interspike intervals are not independent, and we wish to compute a result and not merely a bound, we must work with larger subunit descriptions.
\end{rem}

\begin{ntn}
  The variable $T_s$ is used below to denote the duration of the spike sequence being considered, while $T$, which is much larger than $T_s$, is the duration of the entire spike train.
\end{ntn}

\begin{frm}
  Denote these basic subunits by spike sequences of duration $T_s$. A spike sequence can be obtained as follows.
  \begin{enumerate}[(i)]
  \item Divide time $T_{s}$ into discrete bins of size $\Delta t$, which is small enough so that not more than one spike appears in a bin.
  \item Label each bin by a 0 (no spike) or a 1 (spike), depending on whether or not a spike occurred within it.
  \item Represent a spike sequence defined over a block of duration $T_s$ by a string of $T_s/\Delta t$ zeros and ones.
  \end{enumerate}
  We denote such a sequence by $B(t)$, where $B$ is a $T_s/\Delta t$ bit binary number, and $t$ specifies the time of the first bin in the sequence being considered. Both $T_s$ and $t$ are integer multiples of the bin size $\Delta t$.
\end{frm}

\begin{ntn}
  The probability of a sequence $B$ occurring at any time during the entire response is denoted by $P[B]$.
\end{ntn}

\begin{rem}
  $P[B]$ can be obtained by counting the number of times the sequence $B$ occurs anywhere within the spike trains being analyzed (\emph{including overlapping cases}).
\end{rem}

\begin{prop}
  The spike-train entropy rate implied by the distribution that is characterized by $P[B]$ is
  \begin{equation}
    \label{equ:4.54}
    \dot{H} = -\frac{1}{T_{s}}\sum\limits_{B}P[B]\log_{2}P[B],
  \end{equation}
  where the sum is over all the sequences $B$ found in the data set, and we have divided by the duration $T_{s}$ of a single sequence to obtain an entropy rate.
\end{prop}
%\begin{proof}
 % Here $B$ is a possible value of the sequence over a block of duration $T_{s}$, which is the only one random variable involved.
%\end{proof}

\begin{prop}
  If the spike sequences in nonoverlapping intervals of duration $T_{s}$ are independent and identically distributed, the full spike-train entropy rate is also given by Equation \ref{equ:4.54}.
\end{prop}
\begin{proof}
  By the independence,
  \begin{displaymath}
    \begin{aligned}
      \dot{H} &= \frac{-T/T_{s}\sum\limits_{B}P[B]\log_{2}P[B]}{T} \\
      &= -\frac{1}{T_{s}}\sum\limits_{B}P[B]\log_{2}P[B],
    \end{aligned}
  \end{displaymath}
  which completes the proof.
\end{proof}

\begin{thm}
  For small $T_{s}$ such that the spike sequences are not independent, Equation \ref{equ:4.54} provides an upper bound on the true entropy rate, that is,
  \begin{equation}
    \label{eq:upper1}
    \dot{H} \leq -\frac{1}{T_{s}}\sum\limits_{B}P[B]\log_{2}P[B].
  \end{equation}
\end{thm}
\begin{proof}
  Any correlations between successive intervals (if $B(t+T_{s})$ is correlated with $B(t)$, for example) reduce the total spike-train entropy, causing Equation \ref{equ:4.54} to overestimate the true entropy rate.
\end{proof}

\begin{rem}
  If $T_{s}$ is too small, $B(t+T_{s})$ and $B(t)$ are likely to be correlated, and the overestimate may be severe. As $T_{s}$ increases, we expect the correlations to get smaller, and Equation \ref{equ:4.54} should provide a more accurate value.
\end{rem}

\begin{rem}
  For any finite data set, $T_{s}$ cannot be increased past a certain point, because there will not be enough spike sequences of duration $T_{s}$ in the data set to determine their probabilities. Thus, in practice, $T_{s}$ must be increased until the point where the extraction of probabilities becomes problematic, and some form of extrapolation to $T_{s}\to\infty$ must be made.
\end{rem}

\begin{asm}
  \label{asm:finite-true-relationship}
  Statistical mechanics arguments suggest that the difference between the entropy rate for finite $T_{s}$ and the true entropy rate for $T_{s}\to\infty$ should be proportional to $1/T_{s}$ for large $T_{s}$.
\end{asm}

\begin{prop}
  The true entropy rate can be estimated by linearly extrapolating a plot of the entropy rate versus $1/T_{s}$ to the point $1/T_{s} = 0$.
\end{prop}
\begin{proof}
  This is directly from Assumption \ref{asm:finite-true-relationship}.
\end{proof}

\begin{rem}
  To compute the mutual information rate for a spike train, we must subtract the full noise entropy rate from the full spike-train entropy rate.
\end{rem}

\begin{ntn}
  $P[B(t)]$ is the probability of finding a given sequence $B$ at time $t$ within a set of spike trains obtained on trials using the same stimulus. In contrast, $P[B]$, used in the spike-train entropy rate calculation, is the probability of finding the sequence $B$ at any time within these trains.
\end{ntn}

\begin{lem}
  \label{lem:noiseEntropyRate-t}
  If the same stimulus is used in repeated trials, the noise entropy rate at time $t$ satisfies
  \begin{displaymath}
    \dot{H}_{t} = -\frac{1}{T_{s}}\sum\limits_{B}P[B(t)]\log_{2}P[B(t)].
  \end{displaymath}
\end{lem}
\begin{proof}
  %Definition \ref{def:noiseEntropyRate-t} makes sense.
  The noise entropy rate is determined from the probabilities of finding various sequences $B$, given that they were evoked by the same stimulus. %This is done by considering sequences $B(t)$ that start at a fixed time $t$.
  If the same stimulus is used in repeated trials, sequences $B(t)$ that begin at time $t$ in every trial are generated by the same stimulus. Therefore, the conditional probability of the response, given the stimulus, is in this case the distribution $P[B(t)]$ for response sequences beginning at time $t$. This is obtained by determining the fraction of trials on which $B(t)$ was evoked.
\end{proof}

\begin{rem}
  Determining $P[B(t)]$ for a sufficient number of spike sequences may take a large number of trials using the same stimulus.
\end{rem}

\begin{prop}
  \label{prop:fullNoiseEntropyRate}
  The full noise entropy rate can be computed by averaging the noise entropy rate at time $t$ over all $t$ values, that is,
  \begin{equation}
    \label{equ:4.55}
    \dot{H}_{noise} = -\frac{\Delta t}{T}\sum\limits_{t}\left(\frac{1}{T_{s}}\sum\limits_{B}P[B(t)]\log_{2}P[B(t)]\right),
  \end{equation}
  where $T/\Delta t$ is the number of different $t$ values being summed.
\end{prop}
\begin{proof}
  In this case, the average over $t$ plays the role of the average over stimuli in Equation \ref{equ:4.6}. Then, Lemma \ref{lem:noiseEntropyRate-t} completes the proof.
\end{proof}

\begin{thm}
  If Equation \ref{equ:4.55} is based on finite-length spike sequences, it provides an upper bound on the noise entropy rate, that is,
  \begin{equation}
    \label{equ:upper2}
    \dot{H}_{noise} \leq -\frac{\Delta t}{T}\sum\limits_{t}\left(\frac{1}{T_{s}}\sum\limits_{B}P[B(t)]\log_{2}P[B(t)]\right).
  \end{equation}
\end{thm}

\begin{prop}
  The true noise entropy rate is estimated by performing a linear extrapolation in $1/T_s$ to $1/T_s = 0$.
\end{prop}
\begin{proof}
  As was done for the spike-train entropy rate.
\end{proof}

\begin{exm}
  Entropy and noise entropy rates for the H1 visual neuron in the fly responding to a randomly moving visual image are shown in the following picture. (i) The filled circles in the upper trace show the full spike-train entropy rate computed for different values of $1/T_s$. The straight line is a linear extrapolation to $1/T_s = 0$, which corresponds to $T_s\to \infty$. (ii) The lower trace shows the spike train noise entropy rate for different values of $1/T_s$. The straight line is again an extrapolation to $1/T_s = 0$.
  \begin{center}
    \includegraphics[scale=0.45]{./png/entropyRateEst}
  \end{center}
  Both entropy rates increase as functions of $1/T_s$, and the true spike-train and noise entropy rates are overestimated at large values of $1/T_s$. At $1/T_s\approx 20/s$, there is a sudden shift in the dependence. This occurs when there is insufficient data to compute the spike sequence probabilities. 
  % The difference between the $y$ intercepts of the two straight lines plotted is the mutual information rate.
  By linearly extrapolating the linear part of the series of computed points spike trains had an approximate entropy rate of 157 bits/s and an appeoximate noise entropy rate of 79 bits/s when the resolution was $\Delta t = 3$ ms. The information rate is obtained by taking the difference between the extrapolated values for the spiketrain and noise entropy rates. The result is an information rate of 157 - 79 = 78 bits/s or 1.8 bits/spike.
\end{exm}

\begin{rem}
  Both the spike-train and noise entropy rates depend on $\Delta t$. The leading dependence, coming from the $\log_{2}\Delta t$ term discussed previously, cancels in the computation of the information rate, but the information can still depend on $\Delta t$ through nondivergent terms. This reflects the fact that more information can be extracted from accurately measured spike times than from poorly measured spike times. Thus, we expect the information rate to increase with decreasing $\Delta t$, at least over some range of $\Delta t$ values.  At some critical value of $\Delta t$ that matches the natural degree of noise jitter in the spike timings, we expect the information rate to stop increasing. This value of $\Delta t$ is interesting because it tells us about the degree of spike timing accuracy in neural encoding.
\end{rem}

\begin{rem}
  The information conveyed by spike trains can be used to compare responses to different stimuli and thereby reveal stimulus-specific aspects of neural encoding.
\end{rem}


 



%%% Local Variables:
%%% mode: latex
%%% TeX-master: "../notesOnFluidMechanics"
%%% End:




\begin{asm}
  \label{asm:surfaceForceViscous}
  From now on, 
  assume that
  \begin{equation}
    \label{equ:surfaceForceSigma}
    \text{force on $S$ per unit area} = -p(\mathbf{x}, t)\mathbf{n}+\mathbf{n}\cdot\boldsymbol\sigma(\mathbf{x}, t), 
  \end{equation}
  where $\boldsymbol\sigma$ is the \emph{(deviatoric) stress tensor} and
  $\mathbf{n}$ is the unit outward normal of $S$.
\end{asm}

%%% Local Variables:
%%% mode: latex
%%% TeX-master: "../notesOnFluidMechanics"
%%% End:
\newpage
\section{Spike-Train Statistics}
\label{sec:1.4}


\begin{rem}
        A complete description of the stochastic relationship between a stimulus and a response would require us to know the probabilities corresponding to every sequence of spikes that can be evoked by the stimulus.    
\end{rem}

\begin{lem}
    The probability that $z$ takes a value between $z$ and $z+ \Delta z$, for small $\Delta$(strictly speaking, as $\Delta z \to 0$), is equal to $p[z]\Delta z$, where $p[z]$ is called a probability density.
\end{lem}

\begin{ntn}    
    Throughout this book,  we use the notation $P$[\ ] to denote probabilities and $p$[\ ] to denote probability densities.
\end{ntn}    


\begin{thm}
    The probability of a spike sequence appearing is proportional to the probability density of spike times,  $p[t_1, t_2, ..., t_n]$. In other words, the probability $P[t_1,t_2,...,t_n]$ that a sequence of n spikes occurs with spike $i$ falling between times $t_i$ and $t_i+\Delta t$ for $i= $1,2,...,n is given in terms of this density by the relation 
    \begin{equation}
        P[t_1,t_2,...,t_n]=p[t_1,t_2,...,t_n](\Delta t)^n.        
    \end{equation}
    % This relationship is a special case of Equation \ref{equ:1.37} derived below.
    \begin{proof}
        \small
        $$P[t_1,t_2,...,t_n]=\int... \int p[s_1,s_2,...,s_n]dS\\$$
        $$=\int^{t_n+\Delta t/2}_{t_n-\Delta t/2} 
        \int^{t_{n-1}+\Delta t/2}_{t_{n-1}-\Delta t/2} ...\int^{t_1+\Delta t/2}_{t_1-\Delta t/2} p[s_1,s_2,...,s_n]ds_1 ...ds_{n-1}ds_{n}\\$$
  \\ According to the integral mean value theorem ( $\Delta t \to 0  $ )\\
        $\Rightarrow  P[t_1,t_2,...,t_n]=p[t_1,t_2,...,t_n](\Delta t)^n.        $
        
    \end{proof}
\end{thm}

\begin{defn}[\emph{point process}]
    A stochastic process that generates a sequence of events, such as action potentials ,is called a point process.     
\end{defn}

\begin{rem}
    In general, the probability of an event occurring at any given time could depend on the entire history of preceding events. 
\end{rem}

\begin{defn}[\emph{renewal process}]
    If this dependence extends only to the immediately preceding event, so that the intervals between successive events are independent, the point process is called a renewal process.
\end{defn}

\begin{defn}
    The Poisson process provides an extremely useful approximation of stochastic neuronal firing.
    To make the presentation easier to follow, we separate two cases, the homogeneous Poisson process, for which the firing rate is constant over time, and the inhomogeneous Poisson process, which involves a time-dependent firing rate.
\end{defn}

\subsection{The Homogeneous Poisson Process}

\begin{ntn}
    We denote the firing rate for a homogeneous Poisson process by r$(t)=$r, because it is independent of time.
\end{ntn}

\begin{defn}[\emph{probality of $n$ spikes occuring}]
     The probality that an arbitrary sequence of exactly $n$ spikes occurs within a trial of duration $T$ is $P_T[n]$.
\end{defn}

\begin{thm}
    For a homogeneous Poisson process, the Poisson distribution is 
    \begin{equation}
        P_T[n]=\frac{(rn)^n}{n}exp(-rT).
        \label{equ:1.29}
    \end{equation}
    \begin{proof}
        To compute $P_T[n]$, we divide the time T into M bins of size $\Delta t =T/M$. We assume that $\Delta t$ is small enough so that we never get two spikes within any one bin because, at the end of the calculation,we take the limit $\Delta t \to 0$.\\
        $P_T[n]$ is the product of three factors: \\
            (a)\ The probability of generating $n$ spikes within a  specified set of the $M$ bins,$\frac{M!}{(M-n)!n!}$;\\
            (b)\ The probability of not generating spikes in the remaining $M - n$ bins,$(r\Delta t)^n$;\\
            (c)\ A combinatorial factor equal to the number of ways of putting $n$ spikes into $M$ bins,$(1-r\Delta t)^{M-n}$; \\
            \text{    To sum up,}            
            \begin{equation}
            \label{equ:1.27}
            P_T[n]=\lim_{\Delta t \to 0}\frac{M!}{(M-n)!n!}(r\Delta t)^n(1-r\Delta t)^{M-n}.
        \end{equation}
        As $\Delta t \to 0, M$ grows without bound because $ M\Delta t=T$. Because n is fixed, we can write $M-n\approx M=T/\Delta t$. Using this approximatin and defining $\epsilon=-r\Delta t$, we find that 
        \begin{equation}
            \lim_{\Delta t \to 0}(1-r\Delta t)^{M-n}=\lim_{\epsilon\to 0}(((1+\epsilon)^{\frac{1}{\epsilon}})^{-rT}=\exp(-rT)
        \end{equation}
        For large $M,\ \frac{M!}{(M-n)!}\approx M^n=(T/\Delta t)^n$, so
        \begin{equation}            
            P_T[n]=\frac{(rn)^n}{n}exp(-rT).
        \end{equation}
    \end{proof}
\end{thm}

\begin{exm}
    The probabilities $P_T[n]$, for a few $n$ values, are plotted as a function of $rT$ in the following firgue. Note that as $n$ increase, the probability reaches its maximum at larger $T$ values and that large $n$ values are more likely than small ones for large $T$.
\end{exm}    

\begin{center}
    \label{fig:1.11}                
        \includegraphics[scale = 0.36]{png/Figure1-11-A}\\        
\end{center}

\begin{exm}
    The following figure shows the probabilities of various numbers of spikes occurring when the average number of spikes is $10$. For large $rT$, which corresponds to a large expected number of spikes, the Poisson distribution approaches a Gaussian distribution with mean and variance equal to $rT$. This figure shows that this approximation is already quite good for $rT = 10$.
\end{exm}    

\begin{center}
    \label{fig:1.12}            
    \includegraphics[scale = 0.36]{png/Figure1-11-B}\\    
\end{center}

\begin{thm}
    The probability $P[t_1,t_2,...,t_n]$ can be expressed in terms of another probability function $P_T[n]$, which is the probality that an arbitrary sequence of exactly $n$ spikes occurs within a trial of duration $T$. Assuming that the spike times are ordered $0\leq t_1\leq t_2\leq ...\leq t_n\leq T$, so that, the relationship is 
    \begin{equation}
        P[t_1,t_2,...,t_n]=n!{P_T[n]\left (\frac{\Delta t}{T}\right )^n}.
        \label{equ:1.26}
    \end{equation}
    \begin{proof}
        % represents
        The probability of docking is $ n!(\frac{\Delta t}{T})^n $ in a specific time order $(t_1,t_2,...,t_n).$  so,
       \begin{align}       
         &P[t_1,t_2,...,t_n]={P_T[n]}(n(\frac{\Delta t}{T})(n-1)(\frac{\Delta t}{T})...1(\frac{\Delta t}{T}))\\
        &=n!{P_T[n]\left(\frac{\Delta t}{T}\right)^n}
    \end{align}
    \end{proof}
\end{thm}

\begin{coro}
    We can compute the variance of spike counts produced by a Poisson process from the probabilities in Equation \ref{equ:1.29}. The spike count is 
    \begin{equation}
        \sigma^2_n = \langle n^2 \rangle -\langle n  \rangle ^2=rT.
    \end{equation}
    \begin{proof}
    The average number of spikes generated by a Poisson process with constasnt rate $r$ over a time $T$ is 
    \begin{equation}
        \langle n\rangle=\sum_{n=0}^\infty nP_T[n]=\sum_{n=0}^\infty\frac{n(rT)^n}{n!}\exp(-rT).
        \label{equ:1.45}
    \end{equation}
    and the variance in the spike count is
    \begin{equation}
        \sigma_n^2(T)=\sum_{n=0}^\infty n^2P_T[n]-\langle n\rangle^2=\sum_{n=0}^\infty\frac{n^2(rT)^n}{n!}\exp(-rT)-\langle n\rangle^2.
        \label{equ:1.46}
        \end{equation}
        To compute the quantities,we need to calculate the two sums appearing in these Equations.A good way to do this is to compute the moment-generating function
        \begin{equation}
            g(\alpha)=\sum_{n=0}^\infty\frac{(rT)^n\exp(\alpha n)}{n!}\exp(-rT).
            \label{equ:1.47}
        \end{equation}      
        The $k$th derivative of g with respect to $\alpha$,evaluated at the point $\alpha=0$, is
        \begin{equation}
            \frac{dg}{d\alpha^k}|_{\alpha=0}=\sum_{n=0}^\infty\frac{n^k(rT)^n}{n!}\exp(-rT),
            \label{equ:1.48}
        \end{equation}        
    so once we have computed $g$,we need to calculate only its first and second derivative to determine the sums we need. Rearranging the terms a bit, and recalling that $\exp(z)=\sum z^n/n!$, we find\\        
    \begin{equation}
        g(\alpha)=\exp(-rT)\sum_{n=0}^\infty\frac{(rT\exp(\alpha))^n}{n!}=\exp(-rT)\exp(rTe^\alpha).
        \label{equ:1.49}
    \end{equation}
    The derivatives are then \\
    \begin{equation}
        \frac{dg}{d\alpha}=rTe^\alpha \exp(-rT)\exp(rTe^\alpha)
        \label{equ:1.50}
    \end{equation}
    and\\
    \begin{equation}
    \small    \frac{d^g}{d\alpha^2}=(rTe^\alpha)^2\exp(-rT)\exp(rTe^\alpha)+rTe^\alpha \exp(-rT)\exp(rTe^\alpha).
        \label{equ:1.51}
    \end{equation}
    Evaluating these at $\alpha=0$and putting the results into Equation \ref{equ:1.45} and \ref{equ:1.46} gives the result $\langle n\rangle=rT$ and $$\sigma_n^2(T)=(rT)^2+rT-(rT)^2=rT.$$
    \end{proof}
\end{coro}

\begin{defn}[\emph{Fano factor}]
    % (Fano factor)The ratio of the variance and mean of the spike count ,$\sigma^2_n/\langle n\rangle$,is called the Fano factor.
    The ratio of the variance and mean of the spike count,
   $     \sigma^2_n/\langle n\rangle$, is called the Fano factor.            
\end{defn}

\begin{exm}
    The Fano factor takes the value $1$ for a homogeneous Poisson process, independent of the time interval $T$.
\end{exm}

\begin{lem}
    % For a homogeneous Poisson process,the probability of an interspike intervalfalling between $\tau$ and $\tau + \Delta t$ is $$P[\tau\leq t_{i+1}-t_{i}<\tau +\Delta t]=r\Delta t\ \exp(-r\tau)$$.
    The probability of an interspike intervalfalling between $\tau$ and $\tau + \Delta t$ is 
    \begin{equation}
        P[\tau\leq t_{i+1}-t_{i}<\tau +\Delta t]=r\Delta t\ \exp(-r\tau).
        \label{equ:1.31}
    \end{equation}
    \begin{proof}
        Suppose that a spike occurs at a time $t_i$ for some value of $i$. The probability of a homogeneous Poisson process generating the next spike somewhere in the interval $$t_i+\tau \leq t_{i+1} \leq t_i + \tau +\Delta t,$$ for small $\Delta t$, is the probabilities that no spike is fired for a time $\tau$, times the probability, $r\Delta t$, of  generating a spike within the following small interval $\Delta t$. From Equation \ref{equ:1.29}, with $n=0$, the probability of not firing a spike for period $\tau$ is $\exp(-r\tau)$. So the probability of an interspike interval falling between $\tau$ and $\tau+\Delta t$ is $$  P[\tau\leq t_{i+1}-t_{i}<\tau +\Delta t]=r\Delta t\ \exp(-r\tau).$$
    \end{proof}
\end{lem}

\begin{thm}
    From the interspike interval distribution of a homogeneous Poisson spike train,  we can compute the mean interspike interval, 
    \begin{equation}
        \langle \tau \rangle =\int^{\infty}_{0}\tau r\ \exp(-r\tau)d\tau  = \frac{1}{r}
        \label{equ:1.32}         
    \end{equation}
    and the variance of the interspike intervals, 
    \begin{equation}
        \sigma^2_\tau =\int^{\infty}_{0}\tau^2 r\ \exp(-r\tau)d\tau - \langle \tau \rangle^2 = \frac{1}{r^2}.
        \label{equ:1.33}         
    \end{equation}
\end{thm}

\begin{defn}
    % [\emph{coefficient of variation}]
    The ratio of the standard deviation and the mean of interspike interval distribution.
    \begin{equation}
        C_V=\frac{\sigma_\tau}{\langle \tau  \rangle},
        \label{equ:1.34}
    \end{equation} is the \emph{the coefficient of variation}
\end{defn}

\begin{rem}
    The coefficient of variation takes the value $1$ for a homogeneous Poisson process. This is a necessary,  though not sufficient, condition to identify a Poisson spike train. Recall that the Fano factor for a Poisson process is also $1$. For any renewal process, the Fano factor evaluated over long time intervals approaches the value $C^2_V$.
\end{rem}

\subsection{The Spike-Train Autocorrelation Funciton}

\begin{defn}
        The spike-train autocorrelation function,
        \begin{equation}
            Q_{\rho\rho}(\tau)=\frac{1}{T}\int^T_0 \langle (\rho(t)-\langle r \rangle)(\rho(t+\tau)-\langle r\rangle)\rangle dt,
            \label{equ:1.35}
        \end{equation} is the autocorrelation of the neural response function of Equation \ref{equ:1.1} with its average over time and trials substracted out. 
\end{defn}

\begin{thm}
    The autocorrelation function for a Poisson spike train generated at a constant rate $\langle r \rangle =r$ is 
    \begin{equation}
        Q_{\rho\rho}(\tau)=r\delta(\tau)
    \end{equation}
    \begin{proof}
        The spike-train auto correlation function is constructed from data in the form of a histogram by dividing time into bins. The value of the histogram for a bin labeled with a positive or negative integer $m$ is computed by determining the number of the times that any two spikes in the train are separated by a time interval lying between $(m-1/2)\Delta t$ and $(m+1/2)\Delta $ with $\Delta t$ the bin size.  This includes all pairings, even  between a spike and itself. We call this number $N_m$. If the intervals between the $n^2$ spike pairs in the train were uniformly distributed over the range from $0$ to $T$, there would be $n^2\Delta t/T$ intervals in each bin. This uniform term is removed from the autocorrelation histogram by subtracting $n^2\Delta t /T$ from $N_m$ for all $m$. The spike-train autocorrelation histogram is then defined by dividing the resulting numbers by $T$, so the value of the histogram in bin m is $H_m=N_m/T-n^2\Delta /T^2$. For small bin sizes, the $m = 0$ term in the histogram counts the average number of spikes,  that is $N_m = \langle n \rangle $ and in the limit $\Delta t \to 0,\ H_0=\langle n \rangle /T$ is the average firing rate $\langle r \rangle$. Because other bins have $H_m$ of order $\Delta t$, large $m = 0$ term is often removed from histogram plots. The spike-train autocorrlation function is defined as $H_m/\Delta t$ in the limit $\Delta t \to 0$, and it has the units of a firing rate squared. In this limit,  the $m = 0$ bin becomes a $\delta $funcitn, $H_0/\Delta t\to \langle r\rangle \delta (\tau)$.\\
        As we can seen, the distribution of interspikde intervals for adjacent spikes in a homogeneous Poisson spike train is exponential(Equation \ref{equ:1.31}). By contrast, the intervals between any two spikes(not necessarily adjacent) in such a train are uniformly distributed. As a result,  the subtraction procedure outlined above gives $H_m=0$ for all bins except for the $m=0$ bin that contains the contribution of the zero intervals between spikes and themselves. The autocorrlation function for a Poisson spike train generated at a constant rate $\langle r\rangle = r$ is 
        $$        Q_{\rho\rho}(\tau)=r\delta(\tau).$$
    \end{proof}
\end{thm}

\begin{defn}
The spike-train correlation function ,  
\begin{equation}
    Q_{\rho_1 \rho_2}(\tau)=\frac{1}{T}\int^T_0 \langle (\rho_1(t)-\langle r_1 \rangle)(\rho_2(t+\tau)-\langle r_2\rangle)\rangle dt, 
    \label{equ:1.35}
\end{equation}
    is the correlation of different neural response function $\rho_1(t)$ and $\rho_2(t)$ with their average over time and trials which are $r_1$ and $r_2$ substracted out.
    % (problem)
\end{defn}

\begin{rem}
    The spike-train autocorrelation function is an even function of $\tau$, $ Q_{\rho\rho}(\tau)=Q_{\rho\rho}(-\tau)$, but the cross-correlation function is not necessarily even.
\end{rem}

\begin{exm}
    Asymmetric shifts in this peak away from 0 result from fixed delays between the firing of the twoneurons, and they indicate nonsynchronous but phase-locked firing.    
    Periodic structure in either an autocorrelation or a cross-correlation function or histogram indicates that the firing probability oscillates. Such periodic structure is seen in the histograms of the following firgue, showing 40 Hz oscillations in neurons of catprimary visual cortex that are roughly synchronized between the two cerebral hemispheres.
\end{exm}
% (problem)
\begin{center}
    \label{fig:1.12A}    
    \includegraphics[scale = 0.36]{png/Figure1-12-A.png}
% \end{center}
% \begin{center}
    \label{fig:1.12B} 
    \includegraphics[scale = 0.36]{png/Figure1-12-B.png}\\
\end{center}

\subsection{The Inhomogeneous Poisson Process}
\begin{thm}

The probability density of the inhomogeneous Poisson Process for $n$ spike times is 
    \begin{equation}
    p[t_1, t_2, ..., t_n]=\exp\left(-\int^T_0r(t)dt\right)\prod^n_{i=1}r(t_i),
        \label{equ:1.37}
    \end{equation}
    The spike times are ordered $0\leq t_1 \leq t_2\leq ... \leq t_n \leq T.$
    \begin{proof}
    The probability density for a particular spike sequence with spike times $t_i$ for $i = 1, 2, ..., n$ is obtained from the corresponding probability distribution by multiplying the probability that the spikes occur when they do by the probability that no other spikes occur.We begin by computing the probability that no spikes are generated during the time interval from $t_i$ to $t_{i+1}$ between two adjacent spikes. We determine this by dividing the interval into M bins of size $\Delta t$ and setting $M\Delta t=t_{i+1}-t_i$. We will ultimately take the limit $\Delta t\to 0$. The firing rate during bin $m$ within this interval is $r(t_i+m\Delta t)$. Because the probability of firing a spike in this bin is $r(t_i+m\Delta t)\Delta t$, the probabilities of not firing a spike is $1-r(t_i+m\Delta t)\Delta t$. To have no spikes during the entire interval, we must string together $M$ such bins,  and the probability of this occurring is the product of the individual probabilities, 
            \begin{equation}
            P[\text{no spikes}]=\prod_{m=1}^M(1-r(t_i+m\Delta t)\Delta t).
            \label{equ:1.52}
            \end{equation}
    We evaluate this expression by taking its logarithm,             
            \begin{equation}
            \ln P[\text{no spikes}]=\sum_{m=1}^M\ln(1-r(t_i+m\Delta t)\Delta t),
            \label{equ:1.53}
            \end{equation}
    using the fact that the logarithm of a product is the sum of the logarithms of the multiplied terms. Using the approximation $\ln (1-r(t_i+m\Delta t)\Delta t)\approx -r(t_i+m\Delta t)\Delta t$,  valid for small $\Delta t$, we can simplify this to 
            \begin{equation}
            \ln P[\text{no spikes}]=-\sum_{m=1}^Mr(t_i+m\Delta t)\Delta t.
            \label{equ:1.54}
            \end{equation}
    In the limit $\Delta t \to 0$, the approximation becomes exact and this sum becomes the  integral of $r(t)$ from $t_i$ to $t_{i+1}$, 
            \begin{equation}
            \ln P[\text{no spikes}]=-\int_{t_i}^{t_{i+1}}r(t)dt.
            \label{equ:1.55}
            \end{equation}
    Exponentiating this Equation gives the result we need,             
            \begin{equation}
            P[\text{no spikes}]=\exp\left(-\int_{t_i}^{t_{i+1}}r(t)dt\right).
            \label{equ:1.56}
            \end{equation}
    The probability density $p[t_1, t_2, ..., t_n]$is the product of the densities for the individual spikes and the probabilities of not generating spikes during the interspikde intervals, between time $0$ and the first spike,  and between the time of the last spike and the end of the trial period:            
            \begin{equation}
            \begin{aligned}
            p[t_1, t_2, ...t_n]=\exp\left(-\int_0^{t_1}r(t)dt\right)\exp\left(-\int_{t_n}^Tr(t)dt\right)\times \\  r(t_n)\prod_{i=1}^{n-1}r(t_i)\exp\left(-\int_{t_i}^{t_{i+1}}r(t)dt\right).
            \end{aligned}
            \label{equ:1.57}
            \end{equation}
    The exponentials in this expression all combine because the product of exponentials is the exponential of the sum, so the different integrals in this sum add up to form a single integral:            
            \begin{equation}
                \small
            \begin{aligned}
            &\exp\left(-\int_0^{t_1}r(t)dt)\right)\exp\left(-\int_{t_n}^Tr(t)dt\right)\prod_{i=1}^{n-1}\exp\left(-\int_{t_i}^{t_{i+1}}r(t)dt\right)\\
            &=\exp\left(-\left(\int_0^{t_1}r(t)dt+\sum_{i=1}^{n-1}\int_{t_i}^{t_{i+1}}r(t)dt+\int_{t_n}^Tr(t)dt\right)\right)\\
            &=\exp\left(-\int_0^Tr(t)dt\right) .
            \end{aligned}
            \label{equ:1.58}
            \end{equation}
            Substituting this into Equation \ref{equ:1.57} gives the result in Equation \ref{equ:1.37}
    \end{proof}
\end{thm}

\begin{rem}
The eqution \ref{equ:1.26} is a special case of Equation \ref{equ:1.37}.
\end{rem}

\subsection{The Poisson Spike Generator}

\begin{rul}[\emph{Estimated firing rate}]
    Spike sequences can be simulated by using some estimate of the firing rate, $r_\text{est}(t)$, predicted from knowledge of the stimulus,  to drive a Poisson process.
\end{rul}

\begin{alg}
    The program progresses through time in small steps of size $\Delta t$ and generates, at each time step, a random number $x_{\text{rand}}$ chosen uniformly in the range between $0$ and $1$. If $r_{\text{est}}(t)\Delta t > x_{\text{rand}}$ at that time step, a spike is fired; otherwise it is not.
    % is $r_{est}(t)\Delta t$.
\end{alg}

\begin{alg}
    For a constant firing rate, it is faster to compute spike times $t_i$ for $i=1,2,...,n$ iteratively by generating interspike intervals from an exponential probability density(Equation \ref{equ:1.31}). Thus  we can generate spike times iteratively from the formula $t_{i+1}= t_i-\ln(x_\text{rand}/r)$.
    
 \end{alg}
\begin{rem}
    If $x_\text{rand}$ is uniformly distributed over the range between $0$ and $1$, the negative of its logarithm is exponentially distributed.
\end{rem}
 \begin{alg}[\emph{Spike thinning}]
    The thinning technique requires a bound $r_\text{max}$ on the estimated firing rate such that $r_{\text{est}}(t) \leq r_\text{max}$    at all times. We first generate a spike sequence corresponding to the constant rate $r_{max}$ by iterating the rule $t_{i+1} = t_i - \ln(x_{\text{rand}})/r_\text{max}$. The spike are then thinned by generating another $x_{\text{rand}}$ for each $i$ and removing the spike at time $t_i$ from the train if $r_{\text{est}(t_i)}/r_{\text{max}} < x_{\text{rand}}$. If $r_\text{est}(t_i) / r_{\text{max}} \geq x_{\text{rand}}$, spike $i$ is retained. Thinning corrects for the difference between the estimated timedependent rate and the maximum rate.
    \end{alg} 

\begin{exm}
    The following figures shows an example of a model of an orientation-selective V1 neuron constructed by  Spike thinning. In this model,  the estimated firing rate is determined from the response tuning curve
    \begin{equation}
        r_{est}(t)=f(s(t))=r_{max}\exp\left(-\frac{1}{2}\left(\frac{s(t)-s_{max}}{\sigma_f}\right)^2\right).
        \label{equ:1.38}
    \end{equation}
    % (problem)
\end{exm}

\begin{center}
    \label{fig:1.13A}    
    \includegraphics[scale = 0.2]{png/Figure1-13-A.png}\\
\end{center}

\begin{center}
    \label{fig:1.13B}    
    \includegraphics[scale = 0.2]{png/Figure1-13-B.png}\\
\end{center}

\begin{center}
    \label{fig:1.13C}    
    \includegraphics[scale = 0.2]{png/Figure1-13-C.png}\\
\end{center}
This figure Model of an orientation-selective neuron. The orientation angle (top
panel) was increased from an initial value of $-40^\circ$  by $20^\circ $  every $100$ ms. The firing
rate (middle panel) was used to generate spikes (bottom panel) using a Poisson
spike generator. The bottom panel shows spike sequences generated on five different trials.

\subsection{Comparison with Data}
\begin{rem}
    The Poisson process is simple and useful, but does it match data on neural response variability? To address this question,  we examine Fano factors, interspike interval distributions,  and coefficients of variation.
\end{rem}

\begin{prop}
    The Fano factor describes the relationship between the mean spike count over a given interval and the spike-count variance. Mean spike counts $\langle n \rangle $ and variances $\sigma^2_n$ from a wide variety of neuronal recordings have been fitted to the Equation $\sigma^2_n = A\langle n\rangle^B $, and the \emph{multiplier} $A$ and exponent B have been determined. The values of both $A$ and $B$ typically lie between 
    $1.0$ and $1.5.$
\end{prop}

\begin{rem}
    Because the Poisson model predicts $A = B = 1$, this indicates
that the data show a higher degree of variability than the Poisson model
would predict. However, many of these experiments involve anesthetized
animals, and it is known that response variability is higher in anesthetized
than in alert animals.
\end{rem}


\begin{exm}[\emph{comparison of the Fano factor}]
    The following figures shows data for spike-count means and variances extracted
from recordings of MT neurons in alert macaque monkeys using a number of different stimuli. The MT (medial temporal) area is a visual region of the primate cortex where many neurons are sensitive to image motion.
The individual means and variances are scattered in figure A,  but they
cluster around the diagonal which is the Poisson prediction. Similarly,  the
results show A and B values close to $1$,  the Poisson values (figure B).
Of course,  many neural responses cannot be described by Poisson statistics,  but it is reassuring to see a case where the Poisson model seems a
reasonable approximation. As mentioned previously,  when spike trains
are not described very accurately by a Poisson model,  refractory effects
are often the primary reason.
\end{exm}
\begin{center}
    \label{fig:1.14A}    
    \includegraphics[scale = 0.36]{png/Figure1-14-A.png}\\
\end{center}

\begin{center}
    \label{fig:1.14B}    
    \includegraphics[scale = 0.36]{png/Figure1-14-B.png}\\
\end{center}

\begin{center}
    \label{fig:1.14C}    
    \includegraphics[scale = 0.36]{png/Figure1-14-C.png}\\
\end{center}

\begin{alg}
    Interspike interval distributions are extracted from data as interspike histograms by counting the number of intervals falling in discrete time bins.
\end{alg}

\begin{exm}[\emph{the Poisson model of interspike interval}]
    The following figure presents an example from the responses of a nonbursting cell in area MT of a monkey in response to images consisting of randomly moving dots with a variable amount of coherence imposed on
    their motion (see chapter $3$ for a more detailed description). 
\end{exm}

\begin{center}
    \label{fig:1.15A}    
    % \includegraphics[scale = 0.36]{png/Figure1-15-A.png}\\
    \includegraphics[trim=30 0 0 60,clip,scale = 0.36]{png/Figure1-15-A.png}\\        
\end{center}
For interspike intervals longer than about 10 ms, the shape of this histogram is exponential, in agreement with Equation \ref{equ:1.31}. However, for shorter intervals there is a discrepancy. While the homogeneous Poisson distribution of Equation \ref{equ:1.31} rises for short interspike intervals, the experimental results show a rapid decrease. This is the result of refractoriness making short interspike intervals less likely than the Poisson model would predict.
\begin{rem}
\end{rem}
\begin{prop}
    The data of the Poisson model of interspike interval with a stochastic refractory period can be fitted more accurately by a gamma distribution, 
    \begin{equation}
        p[\tau] = \frac{r(r\tau)^k\exp(-r\tau)}{k!}
        \label{equ:1.39}
    \end{equation}
    with $k>0$, than by the exponential distribution of the Poisson model, which has $k = 0$.
\end{prop}

\begin{exm}[\emph{the Poisson model of interspike interval with a stochastic refractory period}]
    The following figure shows a theoretical histogram obtained by adding a refractory period of variable duration to the Poisson model. Spiking was prohibited during the refractory period,  and then was described once again by a homogeneous Poisson process. The refractory period was randomly chosen from a Gaussian distribution with a mean of $5$ ms and a standard
deviation of $2$ ms (only random draws that generated positive refractory periods were included). The resulting interspike interval distribution of figure \ref{fig:1.15B} agrees quite well with the data.
\end{exm}

\begin{center}
    \label{fig:1.15B}    
    % \includegraphics[scale = 0.36]{png/Figure1-15-B.png}\\
    \includegraphics[trim=10 10 30 30,clip,scale = 0.36]{png/Figure1-15-B.png}\\    
\end{center}

\begin{exm}[\emph{comparion of the coefficients of variation}]
    $C_V$ values extracted from the spike trains of neurons recorded in monkeys from area MT and primary visual cortex(V1) are shown in this figure. The data have been divided into groups based on the mean interspike interval,  and the coefficient of variation is plotted as a function of the mean interval,  equivalent to $1/\langle r\rangle$. Except for short mean interspike intervals, the values are near $1$, although they tend to cluster slightly lower than $1$, the Poisson value. The small $C_V$ values for short interspike intervals are due to the refractory period. The solid curve is the prediction of a Poisson  model with refractoriness.
\end{exm}

\begin{center}
    \label{fig:1.16}    
    \includegraphics[scale = 0.36]{png/Figure1-16.png}\\
\end{center}

\begin{rem}
    However,  there are cases in which the accuracy in the timing and numbers of spikes fired by a neuron is considerably higher than would be implied by Poisson statistics. 
    Furthermore,  even when it successfully describes data,  the Poisson model does not provide a mechanistic explanation of neuronal response variability.
\end{rem}

\begin{exm}
    The following figure compares the response of V1 cells to constant current injection in vivo and in vitro. The in vitro response is a regular and reproducible spike train(left panel). The same current injection paradigm applied in vivo produces a highly irregular pattern of firing(center panel) similar to the response to a moving bar stimulus(right panel).
\end{exm}

\begin{center}
    \label{fig:1.17}    
    \includegraphics[scale = 0.25]{png/Figure1-17.png}\\
\end{center}
Although some of the basic statistical properties of firing variability may be captured by the Poisson model of spike generation,  the spike generating mechanism itself in real neurons is clearly not responsible for the variability. We explore ideas about possible sources of spike-train variability in chapter $5$.

\begin{rem}
    Some neurons fire action potentials in clusters or bursts of spikes that can not be described by a Poisson process with a fixed rate. Bursting can be included in a Poisson model by allowing the firing rate to fluctuate in order to describe the high rate of firing during a burst. Sometimes the distribution of bursts themselves can be described by a Poisson process (such a doubly stochastic process is called a Cox process).    
\end{rem}





% problem: 推导确认一下
% problem: 那些例子如何修改
% problem: 概念如何改变
% problem: tex规格 ok
% problem: 修改exp  ok
% problem: 具体的排版  ok
% problem: 以及一些公式 ok
% problem: 图像无法正确索引 ok


\end{multicols}

\chapter{ Neural Decoding}
\label{cha:Neural Decoding}

\begin{multicols}{2}
\setlength{\columnseprule}{0.2pt}  

\section{Introduction}
\subsection{The Explanation of Some Terms}
\rem \emph{Neurons} are highly specialized for generating electrical signals in response to chemical and other inputs, and transmitting them to other cells.
\rem \emph{Dendrites} receives information inputs from other neurons.
\rem \emph{Axon} carries the neuronal output to other cells.
\begin{center}
    \label{fig:1.1}
    \includegraphics[scale = 0.35]{png/Figure1-1}\\
\end{center}

\rem \emph{Ion channels} control the flow of ions across the cell membrane by opening and closing in response to voltage changes and to both internal and external signals.
\begin{center}
  \label{fig:1.2}
  \includegraphics[scale = 0.55]{png/Figure1-2}\\
\end{center}
  
\con [Membrane Potential]The potential difference between two solutions separated by membranes, generally refers to the electrical phenomenon accompanying the life activities of cells, which exists on both side of cells.
\rem Under resting conditions,the potential inside the cell membrane(mainly K$^+$) is negative, outside the cell membrane(mainly Na$^+$) is positive, and the cell is said to be \emph{polarized}.
\defn [Action Potential] \emph{Action potential} is the characteristic electrical pulses or, more simply, spikes that can travel down nerve fibers.
\con [Hyperpolarization]Current in the form of positively charged ions flowing out of the cell (or negatively charged ions flowing into the cell) through open channels makes the membrane potential more negative, a process called \emph{hyperpolarization}.
\con [Depolarization]Current flowing into the cell changes the membrane potential to less negative or even positive values. This is called \emph{depolarization}.
\rem If a neuron is depolarized sufficiently to raise the membrane potential above a threshold level, a positive feedback process is initiated, and the neuron generates an \emph{action potential}.
\con [Absolute Refractory Period]For a few milliseconds just after an action potential has been fired, it may be virtually impossible to initiate another spike.
\con [Relative Refractory Period]After the absolute refractory period, the excitability of cells gradually recovers. After stimulation, excitement can occur, but the stimulation must be greater than the original threshold intensity.
\rem \emph{Absolute refractory period} and \emph{relative refractory period} are two basic phenomena in the process of neural response.
\subsection{Recording Neuronal Responses}
\exm Intracellular and extracellular methods for recording neuronal responses electrically
\begin{center}
    \label{fig:1.3}
    \includegraphics[scale = 0.35]{png/Figure1-3}\\
\end{center}

\begin{enumerate}[(i)]
  \item The top trace represents a recording from an intracellular electrode connected to the soma of the neuron.
  \item The middle trace is a simulated extracellular recording.
  \item The bottom trace represents a recording from an intracellular electrode connected to the axon some distance away from the soma.
\end{enumerate}




\subsection{From Stimulus to Response}
\rem  Neurons typically respond by producing complex spike sequences that reflect both the intrinsic dynamics of the neuron and the temporal characteristics of the stimulus.
\defn Neural encoding refers to the map from stimulus to response.
\exm We can catalog how neurons respond to a wide variety of stimuli, and then construct models that attempt to predict responses to other stimuli.
\defn Neural decoding refers to the reverse map, from response to stimulus.
\rem The complexity and trial-to-trial variability of action potential sequences make it unlikely that we can describe and predict the timing of each spike deterministically. Instead, we seek a model that can account for the probabilities that different spike sequences are evoked by a specific stimulus.


%%% Local Variables:
%%% mode: latex
%%% TeX-master: t
%%% End:


\section{Discrimination}
\label{sec:Single-cell decoding}
\begin{exm}
  \label{exm:mokey experiment}
  In the experiments performed by Britten et al. (1992).,
a monkey was trained to discriminate between two directions of motion
of a visual stimulus which was a pattern of dots on a video monitor. The percentage of dots that move together in the fixed direction is
called the coherence level. By varying the degree of coherence shown by
pictures, the task of detecting the movement direction can be made more or less
diffcult.
\begin{center} 
 \includegraphics[scale = 0.3]{./png/3-1}
\end{center}
\end{exm}
\begin{defn}[plus and minus]
  \label{defn:plus and minus}
  The preferred direction was called \emph{plus} (\rm{or} $+$ ) 
direction that produced the maximum response
in that neuron, and  its opposite direction is called the \emph{minus}
 (\rm{or} $-$ ) direction.
\end{defn}
\begin{exm}
  \label{fig:random-dot motion-
discrimination task}
  During the same experiment in Example \ref{exm:mokey experiment}, the judgment
  accuracy of the monkey and the optic nerve coding signal activity in
  the MT area were recorded. The experimental results show that: first,
  the coding of MT neural activity is basically sufficient for judging
  the direction; second, at high coherence levels, the fring-rate distributions
corresponding to the two directions are fairly well separated, while
at low coherence levels, they merge.
\begin{center}
  \centering
 \includegraphics[scale = 0.4]{png/3-2AB}
  \label{fig:3.2A}
\end{center}

\end{exm}

\begin{rem}
  Although spike count rates take only discrete values, it is more
  convenient to treat $r$ as a continuous variable for our
  discussion. Treated as probability densities, these two
  distributions are approximately Gaussian with the same variance, $\sigma_{r}^{2}$, but different means,
$\left\langle r \right\rangle_{+}$ for the plus direction and $\left\langle r \right\rangle_{-}$ for the minus direction.
\end{rem}

\begin{defn}[discriminability]
  \label{defn:discriminability}
  A convenient
measure of the separation between the distributions is the
\emph{discriminability}
\begin{equation}
  \label{eq:3.4}
  d'=\frac{\left\langle r \right\rangle_{+}-\left\langle r \right\rangle_{-}}{\sigma_{r}}.
\end{equation}
\end{defn}


\begin{rem}
 
Decoding involves using the neural
response to determine in which of the two possible directions the
stimulus moved for Example \ref{exm:mokey experiment}. A simple decoding procedure in this case is to determine the fring rate
$r$ during a trial and compare it to a threshold number $z$. If $r
\geq z$, we report “plus”; otherwise we report “minus”.
\end{rem}

\begin{defn}[size and power]
  \label{defn:size and power}
    Below are the probabilities of answering plus for both given the conditions:
  \begin{enumerate}[(i)]
  \item The probability that it will give the answer “plus”
when the stimulus is moving in the plus direction is the conditional probability that $r\geq z$ given a plus
stimulus, $\alpha(z)=P[r\geq z|+]$, called \emph{size} or \emph{false
  alarm} rate of the test.
\item The probability that it will give the answer “plus”
when the stimulus is actually moving in the minus direction (called a false
alarm) is similarly $\beta(z)=P[r\geq z|-]$, called \emph{power} or \emph{hit} rate of the test.
\end{enumerate}

These two probabilities completely determine the performance of the decoding procedure because the probabilities for the other two cases
\begin{center}
  \begin{tabular}[h]{|c|cc|}
\hline
         & \multicolumn{2}{c|}{probablity}          \\ \hline
stimulus & \multicolumn{1}{c|}{correct} & incorrect \\ \hline
$+$        & \multicolumn{1}{c|}{$\beta$}        &$1-\beta$       \\ \hline
$-$       & \multicolumn{1}{c|}{$1-\alpha$}     & $\alpha$         \\ \hline
\end{tabular}
\end{center}
\end{defn}


\subsection{ROC Curves}
\begin{defn}
  \label{def:ROC curves}
  The \emph{receiver operating characteristic} (\textbf{ROC}) curve is
  traced out as a function of the threshold $z$. Each point on an ROC
  curve corresponds to a different value of $z$. The $x$ coordinate of
  the point is $\alpha$, the size of the test for this value of $z$
 and the $y$ coordinate is $\beta$. ROC curve provides a way of
 evaluating how test performance depends on the choice of  $z$ and
 indicates how the size and power of a test trade off as the threshold
 is varied.
\end{defn}

\begin{exm}
  \label{exm:ROC curves}
The figure shows ROC curves computed by Britten et al. for several different values of the stimulus coherence.
\begin{enumerate}[(i)]
\item At high coherence levels, when
the task is easy, the ROC curve rises rapidly from $\alpha(z)=0$,
$\beta(z)=0$ as the threshold is lowered from a high value, and the
probability $\beta(z)$ of a correct “plus” answer quickly approaches
$1$ without a concomitant increase in $\alpha(z)$. As the threshold is
lowered further, the probability of giving the answer “plus” when
the correct answer is “minus” also rises, and $\alpha(z)$ increases.
\item At lower high coherence levels, when the task is difficult, the
  curve rises more slowly as $z$ is lowered.
\item At quite low coherence levels, the task is impossible, in that
  the test merely gives random answers, the curve will lie along the diagonal $\alpha=\beta$, because the probabilities of answers being correct and incorrect are equal.
\end{enumerate}
\begin{center}
    \includegraphics[scale = 0.5]{png/3-3}
 \end{center}
\end{exm}

\begin{rem}
  Examination of Example \ref{exm:ROC curves} suggests a relationship between
  the area under the ROC curve and the level of performance on the
  task. When the ROC curve in Example \ref{exm:ROC curves} lies along the diagonal, the
  area underneath it is $1/2$, which is the probability of a correct
  answer in this case (given any threshold). When the task is easy and
  the ROC curve hugs the left axis and upper limit, the area under it approaches $1$, which is again the
  probability of a correct answer (given an appropriate threshold). The
  area underneath the ROC curve is the probability of a correct answer
  in the most cases (given an appropriate threshold).

  However, the precise relationship between task performance and the area under
the ROC curve is complicated by the fact that different threshold values can be
used. This ambiguity can be removed by considering a slightly different
task, called \emph{two-alternative forced choice}.
\end{rem}

\begin{ntn}
  \label{ntn:two-alternative forced choice}
  For \emph{two-alternative forced choice}, the stimulus is presented
twice, once with motion in the plus direction and once in the minus di-
rection. The task is to decide which presentation corresponded to the plus
direction, given the fring rates on both trials, $r_1$ and $r_2$. A natural exten-
sion of the test procedure we have been discussing is to answer trial 1 if
$r_1 \geq r_2 $ and otherwise answer trial 2. This removes the threshold variable
from consideration.
\end{ntn}

\begin{prop}
  \label{prop:two-alternative correct probability}
 In the two-alternative force-choice task, the value of $r$ on one trial serves
as the threshold for the other trial. Then the probability of getting the correct answer
\begin{equation}
  \label{eq:3.6}
  P[\rm{correct}]=\int_{0}^{\infty}{p[z|-]\beta(z)dz}.
\end{equation}
 \begin{proof}
 For example, if the order of stimulus presentation is plus, then minus, the comparison procedure we have
outlined will report the correct answer if $r_1 \geq z$ where $z=r_2$,
and this has probability $P[r_1\geq z|+]=\beta(z)$ with $z=r_2$. For
small $\Delta z$, the probability that $r_2$ takes a value in the range
between $z$ and $z+\Delta z$ when the second trial has a minus
stimulus is $p[z|-]\Delta z$, where $p[z|-]$ is the conditional
fring-rate probability density for a fring rate $r=z$. Integrating over all values of $z$
gives the answer.
 \end{proof}
\end{prop}


\begin{prop}
  \label{prop:size}
The probability of getting the correct answer in the
Equation \ref{eq:3.6} can be transformed into
\begin{equation}
  \label{eq:3.5}
  P[\rm{correct}]=\int_0^1\beta d\alpha.
\end{equation}

 \begin{proof} 
   $\alpha(z)$ mentioned in definition \ref{defn:size and power}, can
be written as an integral of the conditional fring-rate probability density
$p[r|-]$,
\begin{equation}
  \label{eq:3.7}
  \alpha(z)=\int_{z}^{\infty}{p[r|-]dr}.
\end{equation}
   Taking the derivative of this equation with respect to $z$, we find
   that
   \begin{equation*}
       \label{eq:3.8}
       \frac{d\alpha}{dz}=-p[z|-].
     \end{equation*}
     This allows us to make the replacement $p[z|-]dz\rightarrow
     -d\alpha$ in the integral of Equation (\ref{eq:3.6}) and to change
     the integration variable from $z$ to $\alpha$. Noting that
     $\alpha=1$ when $z=0$ and $\alpha=0$ when $z=\infty$, we infer it.
\end{proof}
\end{prop}
\begin{rem}
  The ROC curve is just $\beta$ plotted as a function of $\alpha$, so
this integral is the area under the ROC curve. Thus, the area under
the ROC curve is the probability of responding correctly in the two-alternative forced-choice test.
\end{rem}

\begin{exc}
  Prove that suppose that $p[r|+]$ and $p[r|-]$ are both Gaussian functions with means
$\left\langle r \right\rangle_{+}$ and $\left\langle r
\right\rangle_{-}$, and a common variance $\sigma_r^{2}$. The reader
is invited to show that, in this case,
\begin{equation}
  \label{eq:3.10}
  P[\text{correct}]=\frac{1}{2}\text{erfc} \Big(\frac{
    \left<r\right>_{+}-\left<r\right>_{-}}{2\sigma_{r}}\Big)=\frac{1}{2}\text{erfc} \Big( -\frac{d'}{2} \Big),
\end{equation}
where $d'$ is the discriminability defined in equation (\ref{eq:3.4})
and $\rm{erfc}(x)$ is the complementary error function (which is an integral of a Gaussian distribution) defined as
\begin{equation*}
  \label{eq:3.11}
  \text{erfc}(x)=\frac{2}{\sqrt{\pi}}\int_x^{\infty}\text{exp}(-y^{2})dy.
\end{equation*}
\end{exc}
\begin{rem}
  $P[\rm{correct}]$ and $d'$ are positively correlated, that is to
  say,  the greater the difference in their firing rates, the greater
  the probability of accurate judgment. And in the case where the
  distributions are equalvariance Gaussians, the relationship between
  the discriminability and the area under the ROC curve is invertible because the complementary error function is monotonic.
\end{rem}

\subsection{ROC Analysis of Motion Discrimination}
\begin{rem}
  To interpret the experiment as a two-alternative forced-choice task, Brit
ten et al. imagined that, in addition to being given the fring rate of the
recorded neuron during stimulus presentation, the observer is given the
fring rate of a hypothetical “anti-neuron” having response
characteristics exactly opposite from the recorded neuron.
 In reality, the responses of this
anti-neuron to a plus stimulus were just those of the recorded neuron to a
minus stimulus, and vice versa. The idea of using the responses of a single
neuron to opposite stimuli as if they were the simultaneous responses of
two different neurons will also reappear in our discussion of spike-train decoding. An observer predicting motion directions on the basis of just these
two neurons at a level equal to the area under the ROC curve is termed an
ideal observer.
\end{rem}
\begin{rem}
   The figure A in Example \ref{fig:random-dot motion-
discrimination task} shows a typical result for the performance of an ideal observer
using one recorded neuron and its anti-neuron partner. The open circles in
figure were obtained by calculating the areas under the ROC curves
for this neuron. Amazingly, the ability of the ideal observer to perform
the discrimination task using a single neuron/anti-neuron pair is equal to
the ability of the monkey to do the task. This seems
remarkable because the monkey presumably has access to a large
population of neurons, while the ideal observer uses only two. 
\end{rem}



\subsection{The Likelihood Ratio Test}
\begin{lem}
  The discrimination test we have considered compares the fring rate
  to a fixed threshold value. The Neyman-Pearson lemma shows that it is
  optimal to choose the test function the ratio of
  probability densities (or probabilities),which also can be seen function of the fring rate
  \begin{equation}
    \label{eq:3.12}
    l(r)=\frac{p[r|+]}{p[r|-]},
  \end{equation}
  which is known as the \emph{likelihood ratio}.
  
  \begin{proof}
Consider the difference $\beta$ in the power of two tests that have identical
sizes $\alpha$. One uses the likelihood ratio $l(r)$, and the other uses a different
test function $h(r)$. For the test $h(r)$ using the threshold $z_{h}$,
\begin{equation}
  \label{eq:3.61}
  \begin{aligned}
    \alpha_h(z_h)&=\int {p[r|-]\Theta(h(r)-z_h)dr},\\
    \beta_h(z_h)&=\int {p[r|+]\Theta(h(r)-z_h)dr}.
  \end{aligned}
\end{equation}
Similar equations hold for the $\alpha_l(z_{l})$ and $\beta_l(z_l)$
values for the test $l(r)$ using
the threshold $z_{l}$. We use the $\Theta$  function, which is $1$ for positive and $0$ for
negative values of its argument, to impose the condition that the test is
greater than the threshold. Comparing the $\beta$ values for the two tests, we
find
\begin{equation}
  \label{eq:3.62}
  \begin{aligned}
    \nabla\beta&=\beta_l(z_l)-\beta_h(z_h)\\
                 &=\int{p[r|+]\Theta(l(r)-z_l)dr}-int {p[r|+]\Theta(h(r)-z_h)dr}.
  \end{aligned}
\end{equation}
The range of integration where $l(r)\geq z_l$ and also  $h(r)\geq z_{h}$  cancels between
these two integrals and use the definition $ l(r)=p[r|+]/{p[r|-]}$,
we can replace ${p[r|+]}$ with $l(r){p[r|-]}$ in this equation, giving
\begin{equation}
  \label{eq:3.64}
  \begin{split}
     \nabla\beta=\int{
    l(r)p[r|-]\left(\Theta(l(r)-z_l)\Theta(z_{h}-h(r))\right)dr}\\
    -\int {l(r)p[r|-]\left(\Theta(z_{l}-l(r))\Theta(h(r)-z_h) \right)dr}.
  \end{split}
  \end{equation}
  Then, due to the conditions imposed on $l(r)$ by the $\Theta$ functions within the
integrals, replacing $l(r)$ by $z$ can neither decrease the value of the integral
resulting from the first term in the large parentheses, nor increase the value
arising from the second. This leads to the inequality
\begin{equation}
  \label{eq:3.65}
  \begin{split}
    \nabla\beta\geq z\int {p[r|-]
    \Theta(l(r)-z_l)\Theta(z_{h}-h(r))dr}\\
    -z\int {p[r|-]\Theta(z_{l}-l(r))\Theta(h(r)-z_h)dr}.
  \end{split}
  \end{equation}
  Putting back the region of integration that cancels between these two
  terms (for which $l(r)\geq z_{l}$ and $h(r)\geq z_{h}$), we find
  \begin{equation}
    \label{eq:3.66}
    \nabla\beta\geq z\Big[\int {p[r|-]\Theta(l(r)-z_l)dr}-\int {p[r|-]\Theta(h(r)-z_h)dr}\Big].
  \end{equation}
  By definition, these integrals are the sizes of the two tests, which are equal
by hypothesis. Thus $\beta\geq 0$, showing
likelihood ratio $l(r)$, at least in the sense of maximizing the power for a
given size.
\end{proof}\qedhere
\end{lem}

\begin{rem}
  The test function $r$ used above is
not equal to the likelihood ratio. However, if the likelihood is a
monotonically increasing function of $r$, the fring-rate threshold test
is equivalent to using the likelihood ratio and is also indeed
optimal. Similarly, any monotonic function of the likelihood ratio will
provide as good a test as the likelihood itself, and the logarithm is frequently used.
\end{rem}



\begin{prop}
   There is a direct relationship between the likelihood ratio and the ROC
  curve. As in Equation \ref{eq:3.7} and (\ref{eq:3.8}), we can
  write
  \begin{equation}
  \label{eq:3.13}
  \beta(z)=\int_z^{\infty}p[r|+]dr \quad\text{so} \quad \frac{d\beta}{dz}=-p[z|+].
\end{equation}
Combining this result with Equation \ref{eq:3.8}, we find that
\begin{equation}
  \label{eq:3.14}
  \frac{d\beta}{d\alpha}=\frac{d\beta}{dz}\frac{dz}{d\alpha}=l(z),
\end{equation}
so the slope of the ROC curve is equal to the likelihood ratio.
\end{prop}

\begin{rem}
  Another way of seeing that comparing the likelihood ratio to a threshold
value is an optimal decoding procedure for discrimination uses a \emph{Bayesian
approach} based on associating a cost or penalty with getting the wrong answer.
\end{rem}

\begin{defn}
  The penalty associated with answering “minus” when
the correct answer is “plus” is quantifed by the \emph{loss parameter} $L_{-}$. Similarly, quantify the loss for answering “plus” when the correct answer is
“minus” as $L_{+}$.
\end{defn}

\begin{thm}
  The probabilities that the correct answer is
“plus” or “minus”, given the fring rate $r$, are $P[+|r]$ and $P[-|r]$ respectively. These probabilities are related to the conditional fring-rate probability densities by Bayes theorem,
\begin{equation}
  \label{eq:3.15}
  P[+|r]=\frac{p[r|+]P[+]}{p[r]}\quad \text{and}\quad P[-|r]=\frac{p[r|-]P[-]}{p[r]}.
\end{equation}
\end{thm}


\begin{prop}
  The average loss expected for a “plus” answer when the fring rate is $r$ is
the loss associated with being wrong times the probability of being wrong,
$\rm{Loss}_+=L_{+}P[-|r]$. Similarly, the expected loss when answering “minus”
is $\rm{Loss}_-=L_{-}P[+|r]$. A reasonable strategy is to cut the losses, answering
“plus” if $\rm{Loss}_{+}\le \rm{Loss}_{-} $ and “minus”
otherwise. Using Equation
\ref{eq:3.15}, we find that this strategy gives the response
“plus” if
\begin{equation}
  \label{eq:3.16}
  l(r)=\frac{p[r|+]}{p[r|-]}\geq \frac{L_+P[-]}{L_-P[+]}.
\end{equation}
This shows that the strategy of comparing the likelihood ratio to a threshold is a way of minimizing the expected loss.
\end{prop}


\begin{exm}
  If the conditional probability densities $p[r|+]$ and $p[r|-]$ are Gaussians
with means $r_{+}$ and $r_{-}$ and identical variances $\sigma_r^2$ ,
and $P[+]=P[-]=1/2$, the probability $P[+|r]$ is a sigmoidal function
of $r$,
\begin{equation}
  \label{eq:3.17}
  P[+|r]=\frac{1}{1+\exp(-d'(r-r_{ave})/\sigma_r)},
\end{equation}
where $r_{\rm{ave}}=(r_++r_-)/2$.
\end{exm}

\begin{exm}
  We have thus far considered discriminating between two quite distinct
stimulus values, plus and minus. Often we are interested in discriminating
between two stimulus values $s+\nabla s$ and $s$ that are very close to one another.
In this case, the likelihood ratio is
\begin{equation}
  \begin{aligned}
    \frac{p[r|s+\nabla s]}{p[r|s]} &\approx \frac{p[r|s]+\nabla
                                     s\partial p[r|s]/\partial s}{p[r|s]}
                                     &=1+\nabla \frac{\partial \ln
                                       p[r|s]}{\partial s}.
  \end{aligned}
\end{equation}
For small $\nabla s$, a test that compares
\begin{equation}
  \label{eq:3.19}
  Z(r)=\frac{\partial\ln p[r|s]}{\partial s},
\end{equation}
to a threshold $(z-1)/s$ is equivalent to the likelihood ratio test.
\end{exm}


%%% Local Variables:
%%% mode: latex
%%% TeX-master: "../notesOnFluidMechanics"
%%% TeX-master: t
%%% End:


\section{Entropy and Information for Spike Trains}
\label{sec:Entropy and Information for Spike Trains}


\begin{rem}
  Computing the entropy or information content of a neuronal response characterized by spike times is much more difficult than computing these quantities for responses described by firing rates. Nevertheless, these computations are important, because firing rates are incomplete descriptions that can lead to serious underestimates of the entropy and information.
\end{rem}

\begin{fac}
  \label{fac:spikeTrainInformation}
  Spike-train entropy calculations are typically based on the study of long-duration recordings consisting of many action potentials. The longer the total length of a spike train, the more information it contains.
\end{fac}

\begin{rem}
  By Fact \ref{fac:spikeTrainInformation}, the entropy and mutual information of spike trains are reported as entropy or information rates.
\end{rem}

\begin{defn}
  \label{def:entropyInformationRates}
  The \emph{entropy rate} and \emph{information rate} are defined as the total entropy and information divided by the duration of the spike train, respectively. Alternatively, entropy and mutual information can be divided by the total number of action potentials and reported as bits per spike rather than bits per second.
\end{defn}

\begin{ntn}
  We write the entropy rate as $\dot{H}$ rather than $H$.
\end{ntn}

\begin{fac}
  The temporal pattern of a group of action potentials can be specified by listing either the individual spike times or the sequence of intervals between successive spikes.
\end{fac}

\begin{rem}
  The entropy and mutual information calculations we present are based on a spike-time description, but as an initial example we consider an approximate computation of entropy using interspike intervals.
\end{rem}
\subsection{Based on Interspike Intervals}
\begin{rem}
  The interspike interval is a continuous variable.
\end{rem}
\begin{ntn}
  The probability of an interspike interval falling in the range between $\tau$ and $\tau+\Delta\tau$ is given in terms of the interspike interval probability density by $p[\tau]\Delta\tau$, where $\Delta\tau$ is the resolution.
\end{ntn}

\begin{prop}
  \label{prop:independentInterspike}
  If the different interspike intervals are statistically independent and identically distributed, the entropy associated with the interspike intervals in a spike train of average rate $\left<r\right>$ and of duration $T$ is
  \begin{displaymath}
    H = -\left<r\right>T\int_0^{\infty}p[\tau]\log_2(p[\tau]\Delta\tau)d\tau,
  \end{displaymath}
  where $\left<r\right>T$ is the number of intervals. In this case, the entropy rate is
  \begin{displaymath}
    \dot{H} = -\left<r\right>\int_0^{\infty}p[\tau]\log_2(p[\tau]\Delta\tau)d\tau.
  \end{displaymath}
\end{prop}
\begin{proof}
  These are directly from definitions of the entropy and entropy rate.
\end{proof}

\begin{exm}
  If a spike train is described by a homogeneous Poisson process with rate $\left<r\right>$, we have
  \begin{displaymath}
    p[\tau] = \left<r\right>e^{-\left<r\right>\tau}
  \end{displaymath}
  and the interspikes are statistically independent (Chapter \ref{cha:Neural Encoding I}). Thus,
  \begin{equation}
    \label{equ:4.53}
    \dot{H} = \frac{\left<r\right>}{\ln 2}(1-\ln\left<r\right>\Delta\tau).
  \end{equation}
  In fact,
  \begin{displaymath}
    \begin{aligned}
      \dot{H} &= -\left<r\right>\int_0^{\infty}\left<r\right>e^{-\left<r\right>\tau}\log_2(\left<r\right>e^{-\left<r\right>\tau}\Delta\tau)d\tau\\
      &= -\left<r\right>\int_0^{\infty}e^{-\tau}\log_2(\left<r\right>e^{-\tau}\Delta\tau)d\tau\\
      &= -\left<r\right>\left(\log_2(\left<r\right>\Delta\tau) + \int_0^{\infty}\frac{-e^{-\tau}}{\ln 2}d\tau\right) \\
      &= \frac{\left<r\right>}{\ln 2}(1-\ln\left<r\right>\Delta\tau),
    \end{aligned}
  \end{displaymath}
  where the second step follows from the variable substitution $\tau = \left<r\right> \tau$ and the third step from the integration by parts.
\end{exm}

\begin{defn}
  \label{PossionEntropyRate}
  Equation \ref{equ:4.53} is called the \emph{Poisson entropy rate}.
\end{defn}

\begin{thm}
  In general, the entropy rate $\dot{H}$ for a spike train with interspike interval distribution $p[\tau]$ and average rate $\left<r\right>$ satisfies
  \begin{equation}
    \label{equ:4.52}
    \dot{H} \leq -\left<r\right>\int_0^{\infty}p[\tau]\log_2(p[\tau]\Delta\tau)d\tau.
  \end{equation}
\end{thm}
\begin{proof}
  Correlations between different interspike intervals reduce the total entropy, so the result obtained by assuming independent intervals provides an upper bound on the true entropy of a spike train.
\end{proof}

\subsection{General Computations}

\begin{frm}
  To make entropy calculations practical, a long spike train is broken into statistically independent subunits, and the total entropy is written as the sum of the entropies for the individual subunits.
\end{frm}

\begin{exm}
  In the case of Proposition \ref{prop:independentInterspike}, the subunit was the interspike interval.
\end{exm}

\begin{rem}
  If interspike intervals are not independent, and we wish to compute a result and not merely a bound, we must work with larger subunit descriptions.
\end{rem}

\begin{ntn}
  The variable $T_s$ is used below to denote the duration of the spike sequence being considered, while $T$, which is much larger than $T_s$, is the duration of the entire spike train.
\end{ntn}

\begin{frm}
  Denote these basic subunits by spike sequences of duration $T_s$. A spike sequence can be obtained as follows.
  \begin{enumerate}[(i)]
  \item Divide time $T_{s}$ into discrete bins of size $\Delta t$, which is small enough so that not more than one spike appears in a bin.
  \item Label each bin by a 0 (no spike) or a 1 (spike), depending on whether or not a spike occurred within it.
  \item Represent a spike sequence defined over a block of duration $T_s$ by a string of $T_s/\Delta t$ zeros and ones.
  \end{enumerate}
  We denote such a sequence by $B(t)$, where $B$ is a $T_s/\Delta t$ bit binary number, and $t$ specifies the time of the first bin in the sequence being considered. Both $T_s$ and $t$ are integer multiples of the bin size $\Delta t$.
\end{frm}

\begin{ntn}
  The probability of a sequence $B$ occurring at any time during the entire response is denoted by $P[B]$.
\end{ntn}

\begin{rem}
  $P[B]$ can be obtained by counting the number of times the sequence $B$ occurs anywhere within the spike trains being analyzed (\emph{including overlapping cases}).
\end{rem}

\begin{prop}
  The spike-train entropy rate implied by the distribution that is characterized by $P[B]$ is
  \begin{equation}
    \label{equ:4.54}
    \dot{H} = -\frac{1}{T_{s}}\sum\limits_{B}P[B]\log_{2}P[B],
  \end{equation}
  where the sum is over all the sequences $B$ found in the data set, and we have divided by the duration $T_{s}$ of a single sequence to obtain an entropy rate.
\end{prop}
%\begin{proof}
 % Here $B$ is a possible value of the sequence over a block of duration $T_{s}$, which is the only one random variable involved.
%\end{proof}

\begin{prop}
  If the spike sequences in nonoverlapping intervals of duration $T_{s}$ are independent and identically distributed, the full spike-train entropy rate is also given by Equation \ref{equ:4.54}.
\end{prop}
\begin{proof}
  By the independence,
  \begin{displaymath}
    \begin{aligned}
      \dot{H} &= \frac{-T/T_{s}\sum\limits_{B}P[B]\log_{2}P[B]}{T} \\
      &= -\frac{1}{T_{s}}\sum\limits_{B}P[B]\log_{2}P[B],
    \end{aligned}
  \end{displaymath}
  which completes the proof.
\end{proof}

\begin{thm}
  For small $T_{s}$ such that the spike sequences are not independent, Equation \ref{equ:4.54} provides an upper bound on the true entropy rate, that is,
  \begin{equation}
    \label{eq:upper1}
    \dot{H} \leq -\frac{1}{T_{s}}\sum\limits_{B}P[B]\log_{2}P[B].
  \end{equation}
\end{thm}
\begin{proof}
  Any correlations between successive intervals (if $B(t+T_{s})$ is correlated with $B(t)$, for example) reduce the total spike-train entropy, causing Equation \ref{equ:4.54} to overestimate the true entropy rate.
\end{proof}

\begin{rem}
  If $T_{s}$ is too small, $B(t+T_{s})$ and $B(t)$ are likely to be correlated, and the overestimate may be severe. As $T_{s}$ increases, we expect the correlations to get smaller, and Equation \ref{equ:4.54} should provide a more accurate value.
\end{rem}

\begin{rem}
  For any finite data set, $T_{s}$ cannot be increased past a certain point, because there will not be enough spike sequences of duration $T_{s}$ in the data set to determine their probabilities. Thus, in practice, $T_{s}$ must be increased until the point where the extraction of probabilities becomes problematic, and some form of extrapolation to $T_{s}\to\infty$ must be made.
\end{rem}

\begin{asm}
  \label{asm:finite-true-relationship}
  Statistical mechanics arguments suggest that the difference between the entropy rate for finite $T_{s}$ and the true entropy rate for $T_{s}\to\infty$ should be proportional to $1/T_{s}$ for large $T_{s}$.
\end{asm}

\begin{prop}
  The true entropy rate can be estimated by linearly extrapolating a plot of the entropy rate versus $1/T_{s}$ to the point $1/T_{s} = 0$.
\end{prop}
\begin{proof}
  This is directly from Assumption \ref{asm:finite-true-relationship}.
\end{proof}

\begin{rem}
  To compute the mutual information rate for a spike train, we must subtract the full noise entropy rate from the full spike-train entropy rate.
\end{rem}

\begin{ntn}
  $P[B(t)]$ is the probability of finding a given sequence $B$ at time $t$ within a set of spike trains obtained on trials using the same stimulus. In contrast, $P[B]$, used in the spike-train entropy rate calculation, is the probability of finding the sequence $B$ at any time within these trains.
\end{ntn}

\begin{lem}
  \label{lem:noiseEntropyRate-t}
  If the same stimulus is used in repeated trials, the noise entropy rate at time $t$ satisfies
  \begin{displaymath}
    \dot{H}_{t} = -\frac{1}{T_{s}}\sum\limits_{B}P[B(t)]\log_{2}P[B(t)].
  \end{displaymath}
\end{lem}
\begin{proof}
  %Definition \ref{def:noiseEntropyRate-t} makes sense.
  The noise entropy rate is determined from the probabilities of finding various sequences $B$, given that they were evoked by the same stimulus. %This is done by considering sequences $B(t)$ that start at a fixed time $t$.
  If the same stimulus is used in repeated trials, sequences $B(t)$ that begin at time $t$ in every trial are generated by the same stimulus. Therefore, the conditional probability of the response, given the stimulus, is in this case the distribution $P[B(t)]$ for response sequences beginning at time $t$. This is obtained by determining the fraction of trials on which $B(t)$ was evoked.
\end{proof}

\begin{rem}
  Determining $P[B(t)]$ for a sufficient number of spike sequences may take a large number of trials using the same stimulus.
\end{rem}

\begin{prop}
  \label{prop:fullNoiseEntropyRate}
  The full noise entropy rate can be computed by averaging the noise entropy rate at time $t$ over all $t$ values, that is,
  \begin{equation}
    \label{equ:4.55}
    \dot{H}_{noise} = -\frac{\Delta t}{T}\sum\limits_{t}\left(\frac{1}{T_{s}}\sum\limits_{B}P[B(t)]\log_{2}P[B(t)]\right),
  \end{equation}
  where $T/\Delta t$ is the number of different $t$ values being summed.
\end{prop}
\begin{proof}
  In this case, the average over $t$ plays the role of the average over stimuli in Equation \ref{equ:4.6}. Then, Lemma \ref{lem:noiseEntropyRate-t} completes the proof.
\end{proof}

\begin{thm}
  If Equation \ref{equ:4.55} is based on finite-length spike sequences, it provides an upper bound on the noise entropy rate, that is,
  \begin{equation}
    \label{equ:upper2}
    \dot{H}_{noise} \leq -\frac{\Delta t}{T}\sum\limits_{t}\left(\frac{1}{T_{s}}\sum\limits_{B}P[B(t)]\log_{2}P[B(t)]\right).
  \end{equation}
\end{thm}

\begin{prop}
  The true noise entropy rate is estimated by performing a linear extrapolation in $1/T_s$ to $1/T_s = 0$.
\end{prop}
\begin{proof}
  As was done for the spike-train entropy rate.
\end{proof}

\begin{exm}
  Entropy and noise entropy rates for the H1 visual neuron in the fly responding to a randomly moving visual image are shown in the following picture. (i) The filled circles in the upper trace show the full spike-train entropy rate computed for different values of $1/T_s$. The straight line is a linear extrapolation to $1/T_s = 0$, which corresponds to $T_s\to \infty$. (ii) The lower trace shows the spike train noise entropy rate for different values of $1/T_s$. The straight line is again an extrapolation to $1/T_s = 0$.
  \begin{center}
    \includegraphics[scale=0.45]{./png/entropyRateEst}
  \end{center}
  Both entropy rates increase as functions of $1/T_s$, and the true spike-train and noise entropy rates are overestimated at large values of $1/T_s$. At $1/T_s\approx 20/s$, there is a sudden shift in the dependence. This occurs when there is insufficient data to compute the spike sequence probabilities. 
  % The difference between the $y$ intercepts of the two straight lines plotted is the mutual information rate.
  By linearly extrapolating the linear part of the series of computed points spike trains had an approximate entropy rate of 157 bits/s and an appeoximate noise entropy rate of 79 bits/s when the resolution was $\Delta t = 3$ ms. The information rate is obtained by taking the difference between the extrapolated values for the spiketrain and noise entropy rates. The result is an information rate of 157 - 79 = 78 bits/s or 1.8 bits/spike.
\end{exm}

\begin{rem}
  Both the spike-train and noise entropy rates depend on $\Delta t$. The leading dependence, coming from the $\log_{2}\Delta t$ term discussed previously, cancels in the computation of the information rate, but the information can still depend on $\Delta t$ through nondivergent terms. This reflects the fact that more information can be extracted from accurately measured spike times than from poorly measured spike times. Thus, we expect the information rate to increase with decreasing $\Delta t$, at least over some range of $\Delta t$ values.  At some critical value of $\Delta t$ that matches the natural degree of noise jitter in the spike timings, we expect the information rate to stop increasing. This value of $\Delta t$ is interesting because it tells us about the degree of spike timing accuracy in neural encoding.
\end{rem}

\begin{rem}
  The information conveyed by spike trains can be used to compare responses to different stimuli and thereby reveal stimulus-specific aspects of neural encoding.
\end{rem}


 



%%% Local Variables:
%%% mode: latex
%%% TeX-master: "../notesOnFluidMechanics"
%%% End:




\begin{asm}
  \label{asm:surfaceForceViscous}
  From now on, 
  assume that
  \begin{equation}
    \label{equ:surfaceForceSigma}
    \text{force on $S$ per unit area} = -p(\mathbf{x}, t)\mathbf{n}+\mathbf{n}\cdot\boldsymbol\sigma(\mathbf{x}, t), 
  \end{equation}
  where $\boldsymbol\sigma$ is the \emph{(deviatoric) stress tensor} and
  $\mathbf{n}$ is the unit outward normal of $S$.
\end{asm}

%%% Local Variables:
%%% mode: latex
%%% TeX-master: "../notesOnFluidMechanics"
%%% End:
\newpage
\section{Spike-Train Statistics}
\label{sec:1.4}


\begin{rem}
        A complete description of the stochastic relationship between a stimulus and a response would require us to know the probabilities corresponding to every sequence of spikes that can be evoked by the stimulus.    
\end{rem}

\begin{lem}
    The probability that $z$ takes a value between $z$ and $z+ \Delta z$, for small $\Delta$(strictly speaking, as $\Delta z \to 0$), is equal to $p[z]\Delta z$, where $p[z]$ is called a probability density.
\end{lem}

\begin{ntn}    
    Throughout this book,  we use the notation $P$[\ ] to denote probabilities and $p$[\ ] to denote probability densities.
\end{ntn}    


\begin{thm}
    The probability of a spike sequence appearing is proportional to the probability density of spike times,  $p[t_1, t_2, ..., t_n]$. In other words, the probability $P[t_1,t_2,...,t_n]$ that a sequence of n spikes occurs with spike $i$ falling between times $t_i$ and $t_i+\Delta t$ for $i= $1,2,...,n is given in terms of this density by the relation 
    \begin{equation}
        P[t_1,t_2,...,t_n]=p[t_1,t_2,...,t_n](\Delta t)^n.        
    \end{equation}
    % This relationship is a special case of Equation \ref{equ:1.37} derived below.
    \begin{proof}
        \small
        $$P[t_1,t_2,...,t_n]=\int... \int p[s_1,s_2,...,s_n]dS\\$$
        $$=\int^{t_n+\Delta t/2}_{t_n-\Delta t/2} 
        \int^{t_{n-1}+\Delta t/2}_{t_{n-1}-\Delta t/2} ...\int^{t_1+\Delta t/2}_{t_1-\Delta t/2} p[s_1,s_2,...,s_n]ds_1 ...ds_{n-1}ds_{n}\\$$
  \\ According to the integral mean value theorem ( $\Delta t \to 0  $ )\\
        $\Rightarrow  P[t_1,t_2,...,t_n]=p[t_1,t_2,...,t_n](\Delta t)^n.        $
        
    \end{proof}
\end{thm}

\begin{defn}[\emph{point process}]
    A stochastic process that generates a sequence of events, such as action potentials ,is called a point process.     
\end{defn}

\begin{rem}
    In general, the probability of an event occurring at any given time could depend on the entire history of preceding events. 
\end{rem}

\begin{defn}[\emph{renewal process}]
    If this dependence extends only to the immediately preceding event, so that the intervals between successive events are independent, the point process is called a renewal process.
\end{defn}

\begin{defn}
    The Poisson process provides an extremely useful approximation of stochastic neuronal firing.
    To make the presentation easier to follow, we separate two cases, the homogeneous Poisson process, for which the firing rate is constant over time, and the inhomogeneous Poisson process, which involves a time-dependent firing rate.
\end{defn}

\subsection{The Homogeneous Poisson Process}

\begin{ntn}
    We denote the firing rate for a homogeneous Poisson process by r$(t)=$r, because it is independent of time.
\end{ntn}

\begin{defn}[\emph{probality of $n$ spikes occuring}]
     The probality that an arbitrary sequence of exactly $n$ spikes occurs within a trial of duration $T$ is $P_T[n]$.
\end{defn}

\begin{thm}
    For a homogeneous Poisson process, the Poisson distribution is 
    \begin{equation}
        P_T[n]=\frac{(rn)^n}{n}exp(-rT).
        \label{equ:1.29}
    \end{equation}
    \begin{proof}
        To compute $P_T[n]$, we divide the time T into M bins of size $\Delta t =T/M$. We assume that $\Delta t$ is small enough so that we never get two spikes within any one bin because, at the end of the calculation,we take the limit $\Delta t \to 0$.\\
        $P_T[n]$ is the product of three factors: \\
            (a)\ The probability of generating $n$ spikes within a  specified set of the $M$ bins,$\frac{M!}{(M-n)!n!}$;\\
            (b)\ The probability of not generating spikes in the remaining $M - n$ bins,$(r\Delta t)^n$;\\
            (c)\ A combinatorial factor equal to the number of ways of putting $n$ spikes into $M$ bins,$(1-r\Delta t)^{M-n}$; \\
            \text{    To sum up,}            
            \begin{equation}
            \label{equ:1.27}
            P_T[n]=\lim_{\Delta t \to 0}\frac{M!}{(M-n)!n!}(r\Delta t)^n(1-r\Delta t)^{M-n}.
        \end{equation}
        As $\Delta t \to 0, M$ grows without bound because $ M\Delta t=T$. Because n is fixed, we can write $M-n\approx M=T/\Delta t$. Using this approximatin and defining $\epsilon=-r\Delta t$, we find that 
        \begin{equation}
            \lim_{\Delta t \to 0}(1-r\Delta t)^{M-n}=\lim_{\epsilon\to 0}(((1+\epsilon)^{\frac{1}{\epsilon}})^{-rT}=\exp(-rT)
        \end{equation}
        For large $M,\ \frac{M!}{(M-n)!}\approx M^n=(T/\Delta t)^n$, so
        \begin{equation}            
            P_T[n]=\frac{(rn)^n}{n}exp(-rT).
        \end{equation}
    \end{proof}
\end{thm}

\begin{exm}
    The probabilities $P_T[n]$, for a few $n$ values, are plotted as a function of $rT$ in the following firgue. Note that as $n$ increase, the probability reaches its maximum at larger $T$ values and that large $n$ values are more likely than small ones for large $T$.
\end{exm}    

\begin{center}
    \label{fig:1.11}                
        \includegraphics[scale = 0.36]{png/Figure1-11-A}\\        
\end{center}

\begin{exm}
    The following figure shows the probabilities of various numbers of spikes occurring when the average number of spikes is $10$. For large $rT$, which corresponds to a large expected number of spikes, the Poisson distribution approaches a Gaussian distribution with mean and variance equal to $rT$. This figure shows that this approximation is already quite good for $rT = 10$.
\end{exm}    

\begin{center}
    \label{fig:1.12}            
    \includegraphics[scale = 0.36]{png/Figure1-11-B}\\    
\end{center}

\begin{thm}
    The probability $P[t_1,t_2,...,t_n]$ can be expressed in terms of another probability function $P_T[n]$, which is the probality that an arbitrary sequence of exactly $n$ spikes occurs within a trial of duration $T$. Assuming that the spike times are ordered $0\leq t_1\leq t_2\leq ...\leq t_n\leq T$, so that, the relationship is 
    \begin{equation}
        P[t_1,t_2,...,t_n]=n!{P_T[n]\left (\frac{\Delta t}{T}\right )^n}.
        \label{equ:1.26}
    \end{equation}
    \begin{proof}
        % represents
        The probability of docking is $ n!(\frac{\Delta t}{T})^n $ in a specific time order $(t_1,t_2,...,t_n).$  so,
       \begin{align}       
         &P[t_1,t_2,...,t_n]={P_T[n]}(n(\frac{\Delta t}{T})(n-1)(\frac{\Delta t}{T})...1(\frac{\Delta t}{T}))\\
        &=n!{P_T[n]\left(\frac{\Delta t}{T}\right)^n}
    \end{align}
    \end{proof}
\end{thm}

\begin{coro}
    We can compute the variance of spike counts produced by a Poisson process from the probabilities in Equation \ref{equ:1.29}. The spike count is 
    \begin{equation}
        \sigma^2_n = \langle n^2 \rangle -\langle n  \rangle ^2=rT.
    \end{equation}
    \begin{proof}
    The average number of spikes generated by a Poisson process with constasnt rate $r$ over a time $T$ is 
    \begin{equation}
        \langle n\rangle=\sum_{n=0}^\infty nP_T[n]=\sum_{n=0}^\infty\frac{n(rT)^n}{n!}\exp(-rT).
        \label{equ:1.45}
    \end{equation}
    and the variance in the spike count is
    \begin{equation}
        \sigma_n^2(T)=\sum_{n=0}^\infty n^2P_T[n]-\langle n\rangle^2=\sum_{n=0}^\infty\frac{n^2(rT)^n}{n!}\exp(-rT)-\langle n\rangle^2.
        \label{equ:1.46}
        \end{equation}
        To compute the quantities,we need to calculate the two sums appearing in these Equations.A good way to do this is to compute the moment-generating function
        \begin{equation}
            g(\alpha)=\sum_{n=0}^\infty\frac{(rT)^n\exp(\alpha n)}{n!}\exp(-rT).
            \label{equ:1.47}
        \end{equation}      
        The $k$th derivative of g with respect to $\alpha$,evaluated at the point $\alpha=0$, is
        \begin{equation}
            \frac{dg}{d\alpha^k}|_{\alpha=0}=\sum_{n=0}^\infty\frac{n^k(rT)^n}{n!}\exp(-rT),
            \label{equ:1.48}
        \end{equation}        
    so once we have computed $g$,we need to calculate only its first and second derivative to determine the sums we need. Rearranging the terms a bit, and recalling that $\exp(z)=\sum z^n/n!$, we find\\        
    \begin{equation}
        g(\alpha)=\exp(-rT)\sum_{n=0}^\infty\frac{(rT\exp(\alpha))^n}{n!}=\exp(-rT)\exp(rTe^\alpha).
        \label{equ:1.49}
    \end{equation}
    The derivatives are then \\
    \begin{equation}
        \frac{dg}{d\alpha}=rTe^\alpha \exp(-rT)\exp(rTe^\alpha)
        \label{equ:1.50}
    \end{equation}
    and\\
    \begin{equation}
    \small    \frac{d^g}{d\alpha^2}=(rTe^\alpha)^2\exp(-rT)\exp(rTe^\alpha)+rTe^\alpha \exp(-rT)\exp(rTe^\alpha).
        \label{equ:1.51}
    \end{equation}
    Evaluating these at $\alpha=0$and putting the results into Equation \ref{equ:1.45} and \ref{equ:1.46} gives the result $\langle n\rangle=rT$ and $$\sigma_n^2(T)=(rT)^2+rT-(rT)^2=rT.$$
    \end{proof}
\end{coro}

\begin{defn}[\emph{Fano factor}]
    % (Fano factor)The ratio of the variance and mean of the spike count ,$\sigma^2_n/\langle n\rangle$,is called the Fano factor.
    The ratio of the variance and mean of the spike count,
   $     \sigma^2_n/\langle n\rangle$, is called the Fano factor.            
\end{defn}

\begin{exm}
    The Fano factor takes the value $1$ for a homogeneous Poisson process, independent of the time interval $T$.
\end{exm}

\begin{lem}
    % For a homogeneous Poisson process,the probability of an interspike intervalfalling between $\tau$ and $\tau + \Delta t$ is $$P[\tau\leq t_{i+1}-t_{i}<\tau +\Delta t]=r\Delta t\ \exp(-r\tau)$$.
    The probability of an interspike intervalfalling between $\tau$ and $\tau + \Delta t$ is 
    \begin{equation}
        P[\tau\leq t_{i+1}-t_{i}<\tau +\Delta t]=r\Delta t\ \exp(-r\tau).
        \label{equ:1.31}
    \end{equation}
    \begin{proof}
        Suppose that a spike occurs at a time $t_i$ for some value of $i$. The probability of a homogeneous Poisson process generating the next spike somewhere in the interval $$t_i+\tau \leq t_{i+1} \leq t_i + \tau +\Delta t,$$ for small $\Delta t$, is the probabilities that no spike is fired for a time $\tau$, times the probability, $r\Delta t$, of  generating a spike within the following small interval $\Delta t$. From Equation \ref{equ:1.29}, with $n=0$, the probability of not firing a spike for period $\tau$ is $\exp(-r\tau)$. So the probability of an interspike interval falling between $\tau$ and $\tau+\Delta t$ is $$  P[\tau\leq t_{i+1}-t_{i}<\tau +\Delta t]=r\Delta t\ \exp(-r\tau).$$
    \end{proof}
\end{lem}

\begin{thm}
    From the interspike interval distribution of a homogeneous Poisson spike train,  we can compute the mean interspike interval, 
    \begin{equation}
        \langle \tau \rangle =\int^{\infty}_{0}\tau r\ \exp(-r\tau)d\tau  = \frac{1}{r}
        \label{equ:1.32}         
    \end{equation}
    and the variance of the interspike intervals, 
    \begin{equation}
        \sigma^2_\tau =\int^{\infty}_{0}\tau^2 r\ \exp(-r\tau)d\tau - \langle \tau \rangle^2 = \frac{1}{r^2}.
        \label{equ:1.33}         
    \end{equation}
\end{thm}

\begin{defn}
    % [\emph{coefficient of variation}]
    The ratio of the standard deviation and the mean of interspike interval distribution.
    \begin{equation}
        C_V=\frac{\sigma_\tau}{\langle \tau  \rangle},
        \label{equ:1.34}
    \end{equation} is the \emph{the coefficient of variation}
\end{defn}

\begin{rem}
    The coefficient of variation takes the value $1$ for a homogeneous Poisson process. This is a necessary,  though not sufficient, condition to identify a Poisson spike train. Recall that the Fano factor for a Poisson process is also $1$. For any renewal process, the Fano factor evaluated over long time intervals approaches the value $C^2_V$.
\end{rem}

\subsection{The Spike-Train Autocorrelation Funciton}

\begin{defn}
        The spike-train autocorrelation function,
        \begin{equation}
            Q_{\rho\rho}(\tau)=\frac{1}{T}\int^T_0 \langle (\rho(t)-\langle r \rangle)(\rho(t+\tau)-\langle r\rangle)\rangle dt,
            \label{equ:1.35}
        \end{equation} is the autocorrelation of the neural response function of Equation \ref{equ:1.1} with its average over time and trials substracted out. 
\end{defn}

\begin{thm}
    The autocorrelation function for a Poisson spike train generated at a constant rate $\langle r \rangle =r$ is 
    \begin{equation}
        Q_{\rho\rho}(\tau)=r\delta(\tau)
    \end{equation}
    \begin{proof}
        The spike-train auto correlation function is constructed from data in the form of a histogram by dividing time into bins. The value of the histogram for a bin labeled with a positive or negative integer $m$ is computed by determining the number of the times that any two spikes in the train are separated by a time interval lying between $(m-1/2)\Delta t$ and $(m+1/2)\Delta $ with $\Delta t$ the bin size.  This includes all pairings, even  between a spike and itself. We call this number $N_m$. If the intervals between the $n^2$ spike pairs in the train were uniformly distributed over the range from $0$ to $T$, there would be $n^2\Delta t/T$ intervals in each bin. This uniform term is removed from the autocorrelation histogram by subtracting $n^2\Delta t /T$ from $N_m$ for all $m$. The spike-train autocorrelation histogram is then defined by dividing the resulting numbers by $T$, so the value of the histogram in bin m is $H_m=N_m/T-n^2\Delta /T^2$. For small bin sizes, the $m = 0$ term in the histogram counts the average number of spikes,  that is $N_m = \langle n \rangle $ and in the limit $\Delta t \to 0,\ H_0=\langle n \rangle /T$ is the average firing rate $\langle r \rangle$. Because other bins have $H_m$ of order $\Delta t$, large $m = 0$ term is often removed from histogram plots. The spike-train autocorrlation function is defined as $H_m/\Delta t$ in the limit $\Delta t \to 0$, and it has the units of a firing rate squared. In this limit,  the $m = 0$ bin becomes a $\delta $funcitn, $H_0/\Delta t\to \langle r\rangle \delta (\tau)$.\\
        As we can seen, the distribution of interspikde intervals for adjacent spikes in a homogeneous Poisson spike train is exponential(Equation \ref{equ:1.31}). By contrast, the intervals between any two spikes(not necessarily adjacent) in such a train are uniformly distributed. As a result,  the subtraction procedure outlined above gives $H_m=0$ for all bins except for the $m=0$ bin that contains the contribution of the zero intervals between spikes and themselves. The autocorrlation function for a Poisson spike train generated at a constant rate $\langle r\rangle = r$ is 
        $$        Q_{\rho\rho}(\tau)=r\delta(\tau).$$
    \end{proof}
\end{thm}

\begin{defn}
The spike-train correlation function ,  
\begin{equation}
    Q_{\rho_1 \rho_2}(\tau)=\frac{1}{T}\int^T_0 \langle (\rho_1(t)-\langle r_1 \rangle)(\rho_2(t+\tau)-\langle r_2\rangle)\rangle dt, 
    \label{equ:1.35}
\end{equation}
    is the correlation of different neural response function $\rho_1(t)$ and $\rho_2(t)$ with their average over time and trials which are $r_1$ and $r_2$ substracted out.
    % (problem)
\end{defn}

\begin{rem}
    The spike-train autocorrelation function is an even function of $\tau$, $ Q_{\rho\rho}(\tau)=Q_{\rho\rho}(-\tau)$, but the cross-correlation function is not necessarily even.
\end{rem}

\begin{exm}
    Asymmetric shifts in this peak away from 0 result from fixed delays between the firing of the twoneurons, and they indicate nonsynchronous but phase-locked firing.    
    Periodic structure in either an autocorrelation or a cross-correlation function or histogram indicates that the firing probability oscillates. Such periodic structure is seen in the histograms of the following firgue, showing 40 Hz oscillations in neurons of catprimary visual cortex that are roughly synchronized between the two cerebral hemispheres.
\end{exm}
% (problem)
\begin{center}
    \label{fig:1.12A}    
    \includegraphics[scale = 0.36]{png/Figure1-12-A.png}
% \end{center}
% \begin{center}
    \label{fig:1.12B} 
    \includegraphics[scale = 0.36]{png/Figure1-12-B.png}\\
\end{center}

\subsection{The Inhomogeneous Poisson Process}
\begin{thm}

The probability density of the inhomogeneous Poisson Process for $n$ spike times is 
    \begin{equation}
    p[t_1, t_2, ..., t_n]=\exp\left(-\int^T_0r(t)dt\right)\prod^n_{i=1}r(t_i),
        \label{equ:1.37}
    \end{equation}
    The spike times are ordered $0\leq t_1 \leq t_2\leq ... \leq t_n \leq T.$
    \begin{proof}
    The probability density for a particular spike sequence with spike times $t_i$ for $i = 1, 2, ..., n$ is obtained from the corresponding probability distribution by multiplying the probability that the spikes occur when they do by the probability that no other spikes occur.We begin by computing the probability that no spikes are generated during the time interval from $t_i$ to $t_{i+1}$ between two adjacent spikes. We determine this by dividing the interval into M bins of size $\Delta t$ and setting $M\Delta t=t_{i+1}-t_i$. We will ultimately take the limit $\Delta t\to 0$. The firing rate during bin $m$ within this interval is $r(t_i+m\Delta t)$. Because the probability of firing a spike in this bin is $r(t_i+m\Delta t)\Delta t$, the probabilities of not firing a spike is $1-r(t_i+m\Delta t)\Delta t$. To have no spikes during the entire interval, we must string together $M$ such bins,  and the probability of this occurring is the product of the individual probabilities, 
            \begin{equation}
            P[\text{no spikes}]=\prod_{m=1}^M(1-r(t_i+m\Delta t)\Delta t).
            \label{equ:1.52}
            \end{equation}
    We evaluate this expression by taking its logarithm,             
            \begin{equation}
            \ln P[\text{no spikes}]=\sum_{m=1}^M\ln(1-r(t_i+m\Delta t)\Delta t),
            \label{equ:1.53}
            \end{equation}
    using the fact that the logarithm of a product is the sum of the logarithms of the multiplied terms. Using the approximation $\ln (1-r(t_i+m\Delta t)\Delta t)\approx -r(t_i+m\Delta t)\Delta t$,  valid for small $\Delta t$, we can simplify this to 
            \begin{equation}
            \ln P[\text{no spikes}]=-\sum_{m=1}^Mr(t_i+m\Delta t)\Delta t.
            \label{equ:1.54}
            \end{equation}
    In the limit $\Delta t \to 0$, the approximation becomes exact and this sum becomes the  integral of $r(t)$ from $t_i$ to $t_{i+1}$, 
            \begin{equation}
            \ln P[\text{no spikes}]=-\int_{t_i}^{t_{i+1}}r(t)dt.
            \label{equ:1.55}
            \end{equation}
    Exponentiating this Equation gives the result we need,             
            \begin{equation}
            P[\text{no spikes}]=\exp\left(-\int_{t_i}^{t_{i+1}}r(t)dt\right).
            \label{equ:1.56}
            \end{equation}
    The probability density $p[t_1, t_2, ..., t_n]$is the product of the densities for the individual spikes and the probabilities of not generating spikes during the interspikde intervals, between time $0$ and the first spike,  and between the time of the last spike and the end of the trial period:            
            \begin{equation}
            \begin{aligned}
            p[t_1, t_2, ...t_n]=\exp\left(-\int_0^{t_1}r(t)dt\right)\exp\left(-\int_{t_n}^Tr(t)dt\right)\times \\  r(t_n)\prod_{i=1}^{n-1}r(t_i)\exp\left(-\int_{t_i}^{t_{i+1}}r(t)dt\right).
            \end{aligned}
            \label{equ:1.57}
            \end{equation}
    The exponentials in this expression all combine because the product of exponentials is the exponential of the sum, so the different integrals in this sum add up to form a single integral:            
            \begin{equation}
                \small
            \begin{aligned}
            &\exp\left(-\int_0^{t_1}r(t)dt)\right)\exp\left(-\int_{t_n}^Tr(t)dt\right)\prod_{i=1}^{n-1}\exp\left(-\int_{t_i}^{t_{i+1}}r(t)dt\right)\\
            &=\exp\left(-\left(\int_0^{t_1}r(t)dt+\sum_{i=1}^{n-1}\int_{t_i}^{t_{i+1}}r(t)dt+\int_{t_n}^Tr(t)dt\right)\right)\\
            &=\exp\left(-\int_0^Tr(t)dt\right) .
            \end{aligned}
            \label{equ:1.58}
            \end{equation}
            Substituting this into Equation \ref{equ:1.57} gives the result in Equation \ref{equ:1.37}
    \end{proof}
\end{thm}

\begin{rem}
The eqution \ref{equ:1.26} is a special case of Equation \ref{equ:1.37}.
\end{rem}

\subsection{The Poisson Spike Generator}

\begin{rul}[\emph{Estimated firing rate}]
    Spike sequences can be simulated by using some estimate of the firing rate, $r_\text{est}(t)$, predicted from knowledge of the stimulus,  to drive a Poisson process.
\end{rul}

\begin{alg}
    The program progresses through time in small steps of size $\Delta t$ and generates, at each time step, a random number $x_{\text{rand}}$ chosen uniformly in the range between $0$ and $1$. If $r_{\text{est}}(t)\Delta t > x_{\text{rand}}$ at that time step, a spike is fired; otherwise it is not.
    % is $r_{est}(t)\Delta t$.
\end{alg}

\begin{alg}
    For a constant firing rate, it is faster to compute spike times $t_i$ for $i=1,2,...,n$ iteratively by generating interspike intervals from an exponential probability density(Equation \ref{equ:1.31}). Thus  we can generate spike times iteratively from the formula $t_{i+1}= t_i-\ln(x_\text{rand}/r)$.
    
 \end{alg}
\begin{rem}
    If $x_\text{rand}$ is uniformly distributed over the range between $0$ and $1$, the negative of its logarithm is exponentially distributed.
\end{rem}
 \begin{alg}[\emph{Spike thinning}]
    The thinning technique requires a bound $r_\text{max}$ on the estimated firing rate such that $r_{\text{est}}(t) \leq r_\text{max}$    at all times. We first generate a spike sequence corresponding to the constant rate $r_{max}$ by iterating the rule $t_{i+1} = t_i - \ln(x_{\text{rand}})/r_\text{max}$. The spike are then thinned by generating another $x_{\text{rand}}$ for each $i$ and removing the spike at time $t_i$ from the train if $r_{\text{est}(t_i)}/r_{\text{max}} < x_{\text{rand}}$. If $r_\text{est}(t_i) / r_{\text{max}} \geq x_{\text{rand}}$, spike $i$ is retained. Thinning corrects for the difference between the estimated timedependent rate and the maximum rate.
    \end{alg} 

\begin{exm}
    The following figures shows an example of a model of an orientation-selective V1 neuron constructed by  Spike thinning. In this model,  the estimated firing rate is determined from the response tuning curve
    \begin{equation}
        r_{est}(t)=f(s(t))=r_{max}\exp\left(-\frac{1}{2}\left(\frac{s(t)-s_{max}}{\sigma_f}\right)^2\right).
        \label{equ:1.38}
    \end{equation}
    % (problem)
\end{exm}

\begin{center}
    \label{fig:1.13A}    
    \includegraphics[scale = 0.2]{png/Figure1-13-A.png}\\
\end{center}

\begin{center}
    \label{fig:1.13B}    
    \includegraphics[scale = 0.2]{png/Figure1-13-B.png}\\
\end{center}

\begin{center}
    \label{fig:1.13C}    
    \includegraphics[scale = 0.2]{png/Figure1-13-C.png}\\
\end{center}
This figure Model of an orientation-selective neuron. The orientation angle (top
panel) was increased from an initial value of $-40^\circ$  by $20^\circ $  every $100$ ms. The firing
rate (middle panel) was used to generate spikes (bottom panel) using a Poisson
spike generator. The bottom panel shows spike sequences generated on five different trials.

\subsection{Comparison with Data}
\begin{rem}
    The Poisson process is simple and useful, but does it match data on neural response variability? To address this question,  we examine Fano factors, interspike interval distributions,  and coefficients of variation.
\end{rem}

\begin{prop}
    The Fano factor describes the relationship between the mean spike count over a given interval and the spike-count variance. Mean spike counts $\langle n \rangle $ and variances $\sigma^2_n$ from a wide variety of neuronal recordings have been fitted to the Equation $\sigma^2_n = A\langle n\rangle^B $, and the \emph{multiplier} $A$ and exponent B have been determined. The values of both $A$ and $B$ typically lie between 
    $1.0$ and $1.5.$
\end{prop}

\begin{rem}
    Because the Poisson model predicts $A = B = 1$, this indicates
that the data show a higher degree of variability than the Poisson model
would predict. However, many of these experiments involve anesthetized
animals, and it is known that response variability is higher in anesthetized
than in alert animals.
\end{rem}


\begin{exm}[\emph{comparison of the Fano factor}]
    The following figures shows data for spike-count means and variances extracted
from recordings of MT neurons in alert macaque monkeys using a number of different stimuli. The MT (medial temporal) area is a visual region of the primate cortex where many neurons are sensitive to image motion.
The individual means and variances are scattered in figure A,  but they
cluster around the diagonal which is the Poisson prediction. Similarly,  the
results show A and B values close to $1$,  the Poisson values (figure B).
Of course,  many neural responses cannot be described by Poisson statistics,  but it is reassuring to see a case where the Poisson model seems a
reasonable approximation. As mentioned previously,  when spike trains
are not described very accurately by a Poisson model,  refractory effects
are often the primary reason.
\end{exm}
\begin{center}
    \label{fig:1.14A}    
    \includegraphics[scale = 0.36]{png/Figure1-14-A.png}\\
\end{center}

\begin{center}
    \label{fig:1.14B}    
    \includegraphics[scale = 0.36]{png/Figure1-14-B.png}\\
\end{center}

\begin{center}
    \label{fig:1.14C}    
    \includegraphics[scale = 0.36]{png/Figure1-14-C.png}\\
\end{center}

\begin{alg}
    Interspike interval distributions are extracted from data as interspike histograms by counting the number of intervals falling in discrete time bins.
\end{alg}

\begin{exm}[\emph{the Poisson model of interspike interval}]
    The following figure presents an example from the responses of a nonbursting cell in area MT of a monkey in response to images consisting of randomly moving dots with a variable amount of coherence imposed on
    their motion (see chapter $3$ for a more detailed description). 
\end{exm}

\begin{center}
    \label{fig:1.15A}    
    % \includegraphics[scale = 0.36]{png/Figure1-15-A.png}\\
    \includegraphics[trim=30 0 0 60,clip,scale = 0.36]{png/Figure1-15-A.png}\\        
\end{center}
For interspike intervals longer than about 10 ms, the shape of this histogram is exponential, in agreement with Equation \ref{equ:1.31}. However, for shorter intervals there is a discrepancy. While the homogeneous Poisson distribution of Equation \ref{equ:1.31} rises for short interspike intervals, the experimental results show a rapid decrease. This is the result of refractoriness making short interspike intervals less likely than the Poisson model would predict.
\begin{rem}
\end{rem}
\begin{prop}
    The data of the Poisson model of interspike interval with a stochastic refractory period can be fitted more accurately by a gamma distribution, 
    \begin{equation}
        p[\tau] = \frac{r(r\tau)^k\exp(-r\tau)}{k!}
        \label{equ:1.39}
    \end{equation}
    with $k>0$, than by the exponential distribution of the Poisson model, which has $k = 0$.
\end{prop}

\begin{exm}[\emph{the Poisson model of interspike interval with a stochastic refractory period}]
    The following figure shows a theoretical histogram obtained by adding a refractory period of variable duration to the Poisson model. Spiking was prohibited during the refractory period,  and then was described once again by a homogeneous Poisson process. The refractory period was randomly chosen from a Gaussian distribution with a mean of $5$ ms and a standard
deviation of $2$ ms (only random draws that generated positive refractory periods were included). The resulting interspike interval distribution of figure \ref{fig:1.15B} agrees quite well with the data.
\end{exm}

\begin{center}
    \label{fig:1.15B}    
    % \includegraphics[scale = 0.36]{png/Figure1-15-B.png}\\
    \includegraphics[trim=10 10 30 30,clip,scale = 0.36]{png/Figure1-15-B.png}\\    
\end{center}

\begin{exm}[\emph{comparion of the coefficients of variation}]
    $C_V$ values extracted from the spike trains of neurons recorded in monkeys from area MT and primary visual cortex(V1) are shown in this figure. The data have been divided into groups based on the mean interspike interval,  and the coefficient of variation is plotted as a function of the mean interval,  equivalent to $1/\langle r\rangle$. Except for short mean interspike intervals, the values are near $1$, although they tend to cluster slightly lower than $1$, the Poisson value. The small $C_V$ values for short interspike intervals are due to the refractory period. The solid curve is the prediction of a Poisson  model with refractoriness.
\end{exm}

\begin{center}
    \label{fig:1.16}    
    \includegraphics[scale = 0.36]{png/Figure1-16.png}\\
\end{center}

\begin{rem}
    However,  there are cases in which the accuracy in the timing and numbers of spikes fired by a neuron is considerably higher than would be implied by Poisson statistics. 
    Furthermore,  even when it successfully describes data,  the Poisson model does not provide a mechanistic explanation of neuronal response variability.
\end{rem}

\begin{exm}
    The following figure compares the response of V1 cells to constant current injection in vivo and in vitro. The in vitro response is a regular and reproducible spike train(left panel). The same current injection paradigm applied in vivo produces a highly irregular pattern of firing(center panel) similar to the response to a moving bar stimulus(right panel).
\end{exm}

\begin{center}
    \label{fig:1.17}    
    \includegraphics[scale = 0.25]{png/Figure1-17.png}\\
\end{center}
Although some of the basic statistical properties of firing variability may be captured by the Poisson model of spike generation,  the spike generating mechanism itself in real neurons is clearly not responsible for the variability. We explore ideas about possible sources of spike-train variability in chapter $5$.

\begin{rem}
    Some neurons fire action potentials in clusters or bursts of spikes that can not be described by a Poisson process with a fixed rate. Bursting can be included in a Poisson model by allowing the firing rate to fluctuate in order to describe the high rate of firing during a burst. Sometimes the distribution of bursts themselves can be described by a Poisson process (such a doubly stochastic process is called a Cox process).    
\end{rem}





% problem: 推导确认一下
% problem: 那些例子如何修改
% problem: 概念如何改变
% problem: tex规格 ok
% problem: 修改exp  ok
% problem: 具体的排版  ok
% problem: 以及一些公式 ok
% problem: 图像无法正确索引 ok


\end{multicols}

%\input{sec/GePUPSpatialDiscretization}

\clearpage

\appendix


%\bibliography{bib/numericalApprox}
%\bibliographystyle{abbrv}
%\bibliographystyle{abbrvnat}
%\setcitestyle{authoryear,open={[},close={]}}

\end{document}


%%% Local Variables: 
%%% mode: latex
%%% TeX-master: t
%%% End: 
