\section{1.2}
\label{sec:1.2}

% \begin{rem}
%   The fundamental problem of tensor analysis is to
%   separate results related to geometric and physical objects themselves,
%   from the influence of the arbitrarily chosen coordinate system.
% \end{rem}

\begin{rem}
  Many physical laws are cumbersome
  when written in coordinate form
  but become more compact and attractive looking
  when written in tensorial form.
  For example,
  the incompressible Navier-Stokes equations in cylindrical coordinates are
  \begin{align*}
    \rho\left(\frac{Dv_r}{Dt}-\frac{v_{\theta}^2}{r}\right) &= \rho f_r-\frac{\partial p}{\partial r} + \mu\left(\Delta v_r-\frac{v_r}{r^2}-\frac{2}{r^2}\frac{\partial v_{\theta}}{\partial \theta}\right), \\
    \rho\left(\frac{Dv_{\theta}}{Dt}+\frac{v_rv_{\theta}}{r}\right) &= \rho f_{\theta}-\frac{1}{r}\frac{\partial p}{\partial\theta} + \mu\left(\Delta v_{\theta}+\frac{2}{r^2}\frac{\partial v_r}{\partial\theta}-\frac{v_{\theta}}{r^2}\right), \\
    \rho \frac{Dv_z}{Dt} &= \rho f_z-\frac{\partial p}{\partial z}+\mu\Delta v_z,
  \end{align*}
  where
  \begin{equation*}
    \Delta = \frac{1}{r}\frac{\partial}{\partial r}\left(r \frac{\partial}{\partial r}\right)+\frac{1}{r^2}\frac{\partial^2}{\partial \theta^2}+\frac{\partial^2}{\partial z^2},
  \end{equation*}
  and
  \begin{equation*}
    \frac{D}{Dt} = \frac{\partial}{\partial t}+v_r \frac{\partial}{\partial r}+\frac{v_{\theta}}{r}\frac{\partial}{\partial\theta}+v_z \frac{\partial}{\partial z}.
  \end{equation*}
\end{rem}

%%% Local Variables:
%%% mode: latex
%%% TeX-master: "../notesOnFluidMechanics"
%%% End:
